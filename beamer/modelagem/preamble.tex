\mode<presentation>
{
 \usetheme{JuanLesPins}
 \usefonttheme{serif}
 \usecolortheme{beaver}
 \setbeamercovered{invisible} \setbeamertemplate{blocks}[rounded][shadow=true] 
 \setbeamertemplate{navigation symbols}{} 
 \setbeamertemplate{footline}[frame number]
 \usecolortheme[RGB={122,4,24}]{structure}
}
%\usepackage[portuguese]{babel}
\usepackage[english,brazilian]{babel}
\selectlanguage{brazilian}

\usepackage[utf8]{inputenc}
\usepackage{steinmetz}
\usepackage{lmodern}
\usepackage{fancybox}
\usepackage{graphicx}
\usepackage{colortbl}
\usepackage{color}
\usepackage{courier}
\usepackage{textcomp}
\usepackage{amsmath,amssymb,amsfonts}
\usepackage{xcolor}
\usepackage{multirow}
\usepackage{calligra}
\usepackage{srcltx}
\usepackage{bm}
\usepackage{enumerate}
\usepackage{refcount}
%\usepackage[latin1]{inputenc}
\usepackage[T1]{fontenc}
\usepackage[thinc]{esdiff}
\usepackage[useregional]{datetime2}
\usepackage{mathrsfs}
\usepackage[mathscr]{euscript}
\let\euscr\mathscr \let\mathscr\relax
\usepackage[scr]{rsfso}
\usepackage{empheq}
\usepackage{commath}
%\usepackage{tdclock}

% MATLAB
\usepackage{xspace}
\newcommand{\MATLAB}{\textsc{Matlab}\xspace}

\setcounter{tocdepth}{1} 
\newcommand{\degree}{\ensuremath{^\circ}}
\newcommand{\xmark}{\ding{55}}%
\usepackage{tikz}
\usepackage{karnaugh-map}
\usepackage[american, siunitx]{circuitikz}

\usetikzlibrary{positioning, calc}

\usepackage{booktabs, xfrac, longtable}

\usepackage{cprotect}
\usepackage{verbatim}
\usepackage{siunitx}
\usepackage{bm}
\usepackage{esvect}
\usepackage{steinmetz}
\usepackage{chngcntr}
\counterwithin*{equation}{section}
\newcounter{saveenumi}
\newcommand{\saveenumerate}{%
  \stepcounter{saveenumi}%
  \label{saveenumi-\thesaveenumi}}
\newcommand{\restoreenumerate}{%
  \setcounterref{enumi}{saveenumi-\thesaveenumi}}
\sisetup{output-decimal-marker = {,}}

% \itemequation[label]{text before}{equation}
\makeatletter
\newcommand*{\itemequation}[3][]{%
  \item
  \begingroup
    \refstepcounter{equation}%
    \ifx\\#1\\%
    \else
      \label{#1}%
    \fi
    \sbox0{#2}%
    \sbox2{$\displaystyle#3\m@th$}%
    \sbox4{ \@eqnnum}%
    \dimen@=.5\dimexpr\linewidth-\wd2\relax
    % Warning for overlapping
    \let\CenterInSpace=N%
    \ifcase
    \ifdim\wd0>\dimen@
          \z@
        \else
          \ifdim\wd4>\dimen@
            \z@
          \else
            \@ne
          \fi
        \fi
      \let\CenterInSpace=Y%
    \fi
    \ifdim\dimexpr\wd0+\wd2+\wd4\relax>\linewidth
      \@latex@warning{Equation is too large}%
    \fi
    \noindent
    \rlap{\copy0}%
    \ifx\CenterInSpace Y%
      \rlap{\hbox to \linewidth{\kern\wd0\hss\copy2\hss\kern\wd4}}%
    \else
      \rlap{\hbox to \linewidth{\hfill\copy2\hfill}}%
    \fi
    \hbox to \linewidth{\hfill\copy4}%
    \hspace{0pt}% allow linebreak
  \endgroup
  \ignorespaces
}
\makeatother
%%%%%%%%%%%%%%%%%%%%%%%%%%%%%%%%%%%%%%%%%%%
%%%%%%%%%%%%%%%%%%%%%%%%%%%%%%%%%%%%%%%%%%%
%%%%%%%%%%%%%%%%%% RODAPÉ %%%%%%%%%%%%%%%%%

\setbeamercolor{footline}{fg=white}
\setbeamertemplate{footline}
{\begin{tikzpicture}
    \node [inner sep=0pt, anchor=east] (0,0) {\includegraphics[width=\paperwidth,height=1cm]{Figuras/Capa/macaefooter.png}};
    \node [inner sep=0pt, anchor=east] at (-2ex,-3ex) {\insertframenumber{} / \inserttotalframenumber};
\end{tikzpicture}}


%%%%%%%%%%%%%%%%%%%%%%%%%%%%%%%%%%%%%%%%%%%
%%%%%%%%%%%%%%%%%%%%%%%%%%%%%%%%%%%%%%%%%%%
%%%%%%%%%%% INFORMAÇÕES DO CURSO %%%%%%%%%%

\title[Modelagem de Sistemas Dinâmicos] {Modelagem de Sistemas Dinâmicos}

\subtitle {Notas de aula}

\author{Prof. Yago Pessanha Corrêa}

\institute[MSP/IFF] 
{
Grupo de Pesquisa e Desenvolvimento em Laboratórios de Automação e Controle \\
Instituto Federal de Educação, Ciência e Tecnologia Fluminense (IFFluminense) \\
Bacharelado em Engenharia de Controle e Automação \\
\vspace*{.1cm} {\tt \textbf{yago.correa@iff.edu.br}}\\
}

%\tddate


%%%%%%%%%%%%%%%%%%%%%%%%%%%%%%%%%%%%%%%%%%%
%%%%%%%%%%%%%%%%%%%%%%%%%%%%%%%%%%%%%%%%%%%
%%%%%%%%%%%%%%%%% SUMÁRIO %%%%%%%%%%%%%%%%%

\AtBeginSection[]
{
  \begin{frame}<beamer>{Sumário}
    \tableofcontents[currentsection]
  \end{frame}
}