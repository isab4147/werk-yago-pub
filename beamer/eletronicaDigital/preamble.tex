\mode<presentation>
{
 \usetheme{JuanLesPins}
 \usefonttheme{serif}
 \usecolortheme{beaver}
 \setbeamercovered{invisible} \setbeamertemplate{blocks}[rounded][shadow=true] 
 \setbeamertemplate{navigation symbols}{} 
 \setbeamertemplate{footline}[frame number]
 \usecolortheme[RGB={122,4,24}]{structure}
}

%\usepackage{fancybox}
%\usepackage{graphicx}
%\usepackage{colortbl}
%\usepackage{textcomp}
%\usepackage{multirow}
%\usepackage{calligra}
%\usepackage{srcltx}
%\usepackage{enumerate}
%\newcommand{\degree}{\ensuremath{^\circ}}

\setcounter{tocdepth}{1}

%%%%%%%%%%%%%%%%%%%%%%%%%%%%%%%%%%%%%%%%%%%
%%%%%%%%%%%%%%%%%%%%%%%%%%%%%%%%%%%%%%%%%%%
%%%%%%%%%%%%%%%%% PACKAGES %%%%%%%%%%%%%%%%%

% BASICAO

\usepackage{lmodern}
\usepackage[T1]{fontenc}
\usepackage[useregional]{datetime2}

% Para usar com PDFTEX
%\usepackage[brazilian]{babel}
%\usepackage[utf8]{inputenc}

% Para usar com XELATEX
\usepackage{polyglossia}
\setdefaultlanguage{brazil}


% FIGURAS
\usepackage{graphicx,float}
\usepackage{booktabs, longtable}


% TABELAS E DIAGRAMAÇÃO
\usepackage{enumerate}
\usepackage{multirow,multicol}
\usepackage{makecell}
\usepackage{array}
%\usepackage{fancybox}
%\usepackage{colortbl}
% Alterar tamanho das tabelas
\usepackage{adjustbox}

% Para usar linhas verticais
\aboverulesep=0ex
\belowrulesep=0ex
\renewcommand{\arraystretch}{1.2}

% ?????????????

% Fonte Courier?
%\usepackage{courier}

% Simbolos
%\usepackage{textcomp}

% Texto caligrafico
%\usepackage{calligra}

% Conexão entre pdfs e dvis
%\usepackage{srcltx}

% Para delimitar environments de tipo float
%\usepackage{cprotect}

% Para texto literal (programação)
%\usepackage{verbatim}

% Pegar números relacionados a referências para uso
%\usepackage{refcount}

% Para colocar relógios de ponteiro
%\usepackage{tdclock}

%%%%%%%%%%%%%%%%%%%%%%%%%%%%%%%%%%%%%%%%%%%
%%%%%%%%%%%%%%%%%%%%%%%%%%%%%%%%%%%%%%%%%%%
%%%%%%%%%%%%%%%%% COISAS MATEMATICAS %%%%%%%%%%%%%%%%%

\usepackage{amsmath,amssymb,amsfonts,xfrac,cancel,bm,mathtools}

% Aumentar tamanho da letra em mathmode
\usepackage{relsize}

% Notação de ângulos para complexos
%\usepackage{steinmetz}

% Notação de derivadas
%\usepackage[thinc]{esdiff}
%\usepackage{commath}

% Setinhas para notação de vetor
%\usepackage{esvect}

% Fonte para notação em rsfs
%\usepackage{mathrsfs}

% Para mapas de karnaugh
\usepackage{karnaugh-map}

% Para usar \hfill em mathmode
\makeatletter
\newcommand{\pushright}[1]{\ifmeasuring@#1\else\omit\hfill$\displaystyle#1$\fi\ignorespaces}
\newcommand{\pushleft}[1]{\ifmeasuring@#1\else\omit$\displaystyle#1$\hfill\fi\ignorespaces}
\makeatother


% setup do (circui)tikz
\usepackage[american, nooldvoltagedirection, siunitx, americanports]{circuitikz}

% Unidades (inclusa no circuitikz)
%\usepackage{siunitx}

\usepackage{tikz}
\usetikzlibrary{positioning, calc,
patterns, arrows.meta, decorations.pathmorphing, graphs, matrix}

\newcommand{\setmyunit}[1]{\tikzset{every picture/.style={x=#1, y=#1}}}

%\newlength{\ladderskip}
%\setlength{\ladderskip}{5\tikzcircuitssizeunit} % 5\tikzcircuitssizeunit = 35pt
%\newlength{\ladderrungsep}
%\setlength{\ladderrungsep}{.2\ladderskip}
%\def\ladderrungend#1{\pgftransformyshift{-#1\ladderskip-\ladderrungsep}}
%\newcommand{\powerrails}[1][0.7]{\draw let \p1=(laddertopright) in
%	(0,\y1+#1\ladderskip) -- (0,\ladderskip)
%	(\x1,\y1+#1\ladderskip) -- (\x1,\ladderskip);}

% Cores (inclusa no tikz)
%\usepackage[]{xcolor}


%%%%%%%%%%%%%%%%%%%%%%%%%%%%%%%%%%%%%%%%%%%
%%%%%%%%%%%%%%%%%%%%%%%%%%%%%%%%%%%%%%%%%%%
%%%%%%%%%%%%%%%%% COISAS DOIDAS %%%%%%%%%%%%%%%%%

% setup do siunitx
\sisetup{per-mode=symbol,output-decimal-marker={,},math-micro=\text{µ},text-micro=µ,exponent-product = \cdot}

% MATLAB
%\usepackage{xspace}
%\newcommand{\MATLAB}{\textsc{Matlab}\xspace}

% Notação de negação lógica
\newcommand{\notted}[1]{%
	\overline{#1}%
}

% Padrões de linhas que não funcionam no overleaf

%\pgfdeclarepatternformonly{south east lines}{\pgfqpoint{-0pt}{-0pt}}{\pgfqpoint{3pt}{3pt}}{\pgfqpoint{3pt}{3pt}}{
%	\pgfsetlinewidth{0.4pt}
%	\pgfpathmoveto{\pgfqpoint{0pt}{3pt}}
%	\pgfpathlineto{\pgfqpoint{3pt}{0pt}}
%	\pgfpathmoveto{\pgfqpoint{.2pt}{-.2pt}}
%	\pgfpathlineto{\pgfqpoint{-.2pt}{.2pt}}
%	\pgfpathmoveto{\pgfqpoint{3.2pt}{2.8pt}}
%	\pgfpathlineto{\pgfqpoint{2.8pt}{3.2pt}}
%	\pgfusepath{stroke}}
%
%\pgfdeclarepatternformonly{south west lines}{\pgfqpoint{-0pt}{-0pt}}{\pgfqpoint{3pt}{3pt}}{\pgfqpoint{3pt}{3pt}}{
%	\pgfsetlinewidth{0.4pt}
%	\pgfpathmoveto{\pgfqpoint{0pt}{0pt}}
%	\pgfpathlineto{\pgfqpoint{3pt}{3pt}}
%	\pgfpathmoveto{\pgfqpoint{2.8pt}{-.2pt}}
%	\pgfpathlineto{\pgfqpoint{3.2pt}{.2pt}}
%	\pgfpathmoveto{\pgfqpoint{-.2pt}{2.8pt}}
%	\pgfpathlineto{\pgfqpoint{.2pt}{3.2pt}}
%	\pgfusepath{stroke}}


% cor do fundo do block
%\definecolor{mWhite}{RGB}{239, 230, 231}

% Definições de funções trinométricas em pt-br
%\DeclareMathOperator{\sen}{sen}
%\DeclareMathOperator{\tg}{tg}
%\DeclareMathOperator{\cotg}{cotg}
%\DeclareMathOperator{\cossec}{cossec}

% QED branco
%\newcommand*{\QEDB}{\hfill\ensuremath{\square}}

% estilos para diagrama de blocos
%\newcommand{\deftkzbds}{
%	\tikzstyle{block} = [draw, fill=blue!20, rectangle, minimum height=3em, minimum width=6em]
%	\tikzstyle{sum} = [draw, fill=blue!20, circle, node distance=1cm]
%	\tikzstyle{input} = [coordinate]
%	\tikzstyle{output} = [coordinate]
%	\tikzstyle{pinstyle} = [pin edge={to-,thin,black}]
%}

% tipo de coluna matemática (COM espaçamento mínimo)
\usepackage{stackengine,collcell}
\let\endminwd\relax
%\newcolumntype{L}[1]{>{\collectcell\xminwd l{#1}$}l<{$\endminwd\endcollectcell}}
\newcolumntype{C}[1]{>{\collectcell\xminwd c{#1}$}c<{$\endminwd\endcollectcell}}
%\newcolumntype{R}[1]{>{\collectcell\xminwd r{#1}$}r<{$\endminwd\endcollectcell}}
\def\minwd#1#2#3\endminwd{\stackengine{0pt}{#3}{\rule{#2}{0pt}}{O}{#1}{F}{F}{L}}
\newcommand\xminwd[1]{\minwd#1}

% tipo de coluna matemática (SEM espaçamento mínimo)
%\newcolumntype{L}{>{$}l<{$}}
%\newcolumntype{C}{>{$}c<{$}}
%\newcolumntype{R}{>{$}r<{$}}

% Para adição longa
\newcommand*{\carry}[1][1]{\overset{#1}}
\newcolumntype{B}[1]{r*{#1}{@{\,}r}}

% Para multiplicação de binários
\usepackage{xparse}
\ExplSyntaxOn
\NewDocumentCommand{\binmult}{mm}
 {
  \david_binmult:nn { #1 } { #2 }
 }

\int_new:N \l_david_binmult_first_int
\int_new:N \l_david_binmult_second_int
\int_new:N \l_david_binmult_bits_int
\int_new:N \l__david_binmult_cycle_int

\tl_new:N \l_david_binmult_first_tl
\tl_new:N \l_david_binmult_second_tl
\tl_new:N \l_david_binmult_second_rev_tl
\tl_new:N \l_david_binmult_product_tl
\tl_new:N \l_david_binmult_tablebody_tl

\seq_new:N \l_david_binmult_partial_products_seq
\seq_new:N \l_david_binmult_partial_descr_seq

\cs_new_protected:Nn \__david_binmult_pad:Nn
 {% #1 should be a tl, #2 an integer denomination
  \prg_replicate:nn { \int_max:nn { #2 - \tl_count:N #1 } { 0 } }
   {
    \tl_put_left:Nn #1 { 0 }
   }
 }

\cs_new_protected:Nn \david_binmult:nn
 {
  % store the data
  \int_set:Nn \l_david_binmult_first_int { #1 }
  \int_set:Nn \l_david_binmult_second_int { #2 }
  \tl_set:Nx \l_david_binmult_first_tl { \int_to_bin:n { #1 } }
  \tl_set:Nx \l_david_binmult_second_tl { \int_to_bin:n { #2 } }
  \tl_set:Nx \l_david_binmult_product_tl { \int_to_bin:n { #1 * #2 } }
  % pad the factors
  \int_set:Nn \l_david_binmult_bits_int
   {
    \int_max:nn { \tl_count:N \l_david_binmult_first_tl } 
                { \tl_count:N \l_david_binmult_second_tl }
   }
  % pad the product
  \__david_binmult_pad:Nn \l_david_binmult_product_tl
   {
    2*\l_david_binmult_bits_int-1
   }
  \__david_binmult_pad:Nn \l_david_binmult_first_tl { \l_david_binmult_bits_int }
  \__david_binmult_pad:Nn \l_david_binmult_second_tl { \l_david_binmult_bits_int }
  \tl_set_eq:NN \l_david_binmult_second_rev_tl \l_david_binmult_second_tl
  \tl_reverse:N \l_david_binmult_second_rev_tl
  % compute the partial products
  \seq_clear:N \l_david_binmult_partial_products_seq
  \seq_clear:N \l_david_binmult_partial_descr_seq
  \int_zero:N \l__david_binmult_cycle_int
  \tl_map_inline:Nn \l_david_binmult_second_rev_tl
   {
    \int_compare:nTF { ##1 = 0 }
     {
      \seq_put_right:Nx \l_david_binmult_partial_products_seq
       {
        \prg_replicate:nn { \l_david_binmult_bits_int } { 0 }
        \prg_replicate:nn { \l__david_binmult_cycle_int } { \exp_not:N \hphantom {0} }
       }
     }
     {
      \seq_put_right:Nx \l_david_binmult_partial_products_seq
       {
        \tl_use:N \l_david_binmult_first_tl
        \prg_replicate:nn { \l__david_binmult_cycle_int } { \exp_not:N \hphantom {0} }
       }
     }
    \seq_put_right:Nx \l_david_binmult_partial_descr_seq
     {
      [
      $\tl_use:N \l_david_binmult_first_tl \times ##1$
      \int_case:nnF { \l__david_binmult_cycle_int }
       {
        {0}{}
        {1}{,~movido~para~a~esquerda}
       }
       {,~movido~para~a~esquerda~\david_binmult_number:n { \l__david_binmult_cycle_int }~posições}
      ]
     }
    \int_incr:N \l__david_binmult_cycle_int
   }
  % build the tabular
  \tl_clear:N \l_david_binmult_tablebody_tl
  \tl_put_right:Nn \l_david_binmult_tablebody_tl
   {
    & $\tl_use:N \l_david_binmult_first_tl$ & [$#1$~em~decimal] \\
    & ${\times}\;\tl_use:N \l_david_binmult_second_tl$ & [$#2$~em~decimal] \\
    \cmidrule(r){2-2}
   }
  \int_step_inline:nnnn { 1 } { 1 } { \seq_count:N \l_david_binmult_partial_products_seq }
   {
    \tl_put_right:Nx \l_david_binmult_tablebody_tl
     {
      $+$ &
      \seq_item:Nn \l_david_binmult_partial_products_seq { ##1 } &
      \seq_item:Nn \l_david_binmult_partial_descr_seq { ##1 } \exp_not:N \\
     }
   }
  \tl_put_right:Nn \l_david_binmult_tablebody_tl
   {
    \cmidrule(r){1-2}
    & $\tl_use:N \l_david_binmult_product_tl$ & [$\int_to_arabic:n { #1 * #2 }$~em~decimal]
   }
  % print the table 
  \begin{tabular}{@{} c r l @{}}
  \tl_use:N \l_david_binmult_tablebody_tl
  \end{tabular}
 }

\cs_new:Nn \david_binmult_number:n
 {
  \int_case:nn { #1 }
   {
    {0}{zero}
    {1}{uma}
    {2}{duas}
    {3}{três}
    {4}{quatro}
    {5}{cinco}
    {6}{seis}
    {7}{sete}
    {8}{oito}
   }
 }
\ExplSyntaxOff

% Troca de base
\usepackage{xintfrac, xinttools}
\usepackage{polexpr}[2018/02/16]% Pour \PolDecToString
\usepackage{babel}
\usepackage[autolanguage,np]{numprint}
\usepackage{geometry}
\usepackage{xint}
% TO BASE 10

% FROM https://tex.stackexchange.com/questions/400806/n-base-to-decimal-calculations-in-latex
% From https://tex.stackexchange.com/a/400571/4686
\newcommand{\tablenode}[2]
{\tikz[baseline=(#1.base),remember picture]\node[inner sep=0pt,name=#1]{#2};}

\makeatletter
% Works with bases up to 16, using ABCDEF as digits for 10, ..., 15
% (we use that TeX understand "A, ..., "F; for higher bases we would
% need macros converting from the base-b digit to the base-10 number)
\newcommand\ConvertFromBase[2]{%
	\begingroup
	\ttfamily
	\edef\my@base{#1}%   allow #1 to be a macro
	\edef\my@number{#2}% allow #2 to be a macro
	\gdef\my@total{0}%
	\gdef\my@power{1}%
	\edef\my@nbofdigits{\expandafter\xintLength\expandafter{\my@number}}%
	% make fun things with colors
	\definecolorseries{foo}{rgb}{last}{red}{blue}%
	\resetcolorseries[\my@nbofdigits]{foo}%
	% input number
	\global\let\my@nodeindex\my@nbofdigits
	\begin{tabular}{@{}*{\my@nbofdigits}{c@{}}c@{}}
		\xintFor* ##1 in {\my@number}\do{%
			\tablenode{A\my@nodeindex}{\textcolor{foo!!+}{##1}}%
			\xdef\my@nodeindex{\the\numexpr\my@nodeindex-1}% step by -1
			&}%
		${}_{\my@base}$
	\end{tabular} %<-- deliberate space
	% make fun things with colors
	\definecolorseries{foo}{rgb}{last}{blue}{red}%
	\resetcolorseries[\my@nbofdigits]{foo}%
	% output conversion
	% \my@nodeindex is at 0
	\begin{tabular}[t]{@{}l@{}c@{}r@{}}
		&\\
		\xintFor* ##1 in % \xintReverseOrder does not expand its argument
		% \xintReverseDigits does but it works only with 0..9 digits
		% we could use \xintReverseOrder{#2}, if we did not need
		% to allow #2 to be a macro itself.
		{\expandafter\xintReverseOrder\expandafter{\my@number}}\do{%
			\xdef\my@nodeindex{\the\numexpr\my@nodeindex+1}% step by +1
			\tablenode{B\my@nodeindex}
			{\textcolor{foo}{##1}${}\cdot\my@base^{\the\numexpr\my@nodeindex-1}$}%
			&${}={}$&%
			% use " notation to allow also A, B, C, D, E, F (only uppercase)
			\edef\my@partial{\xintiiMul{\the\numexpr"##1\relax}{\my@power}}%
			\xdef\my@total{\xintiiAdd{\my@total}{\my@partial}}%
			\xdef\my@power{\xintiiMul{\my@base}{\my@power}}%
			\textcolor{foo!!+}{$\my@partial$}\\}%
		\cline{3-3}
		&&\textcolor{red}{$\my@total$}
	\end{tabular}%<-- no space here
	% workaround the fact that foo!!+ from xcolor steps twice when used
	% with tikz's \draw (I guess once from the arrow, once from the arrow tip)
	\resetcolorseries[\numexpr2*\my@nbofdigits\relax]{foo}%
	% DRAWING THE ARROWS
	%
	% YOU NEED TO COMPILE AT LEAST TWICE FOR THE START AND END
	% NODE LOCATIONS TO STABILIZE
	%
	\begin{tikzpicture}[remember picture, overlay, >=stealth]
	% tikz natural language:
	% (this loop uses first the nodes for the least significant digits
	%  and ends up with the ones for the most significant digits)
	\foreach\my@nodeindex in {1, 2, ..., \my@nbofdigits}
	{\draw [->,very thick,{foo!!+}] 
		(A\my@nodeindex.south) to[out=270,in=180] (B\my@nodeindex.west);}%
	% or again with an \xintFor loop
	%  \xintFor* ##1 in {\xintSeq{1}{\my@nbofdigits}}\do{%
	%     \draw [->,very thick,{foo!!+}] 
	%           (A##1.south) to[out=270,in=180] (B##1.west);%
	%    }%
	\end{tikzpicture}%<-- no space here
	\endgroup
}
\makeatother


% INTEGERS

\newcount\total
\newcount\lasttotal
\newcount\targetbase

\def\basetenconversiontable#1#2{%
    \begin{tikzpicture}[every node/.style={minimum width=1cm, minimum height=0.5cm}, x=1cm,y=0.5cm]
    %
    \total=#1%
    \targetbase=#2
    \def\newnumber{}
    %
    \pgfmathloop
    \ifnum\total<1
    \else
        %
        \ifnum\pgfmathcounter>1
            \node at (\pgfmathcounter, -\pgfmathcounter+1) (tmp) {\the\targetbase};
            \draw (tmp.north west) |- (tmp.south east);
            %
            \node at (\pgfmathcounter-1, -\pgfmathcounter) (tmp) {\pgfmathparse{int(\total*\targetbase)}\pgfmathresult};
            \draw (tmp.south west) -- (tmp.south east);
            %
            \pgfmathparse{int(\lasttotal-\total*\targetbase)}%
            \let\digit=\pgfmathresult
            \node at (\pgfmathcounter-1, -\pgfmathcounter-1) [text=red] {\digit};
            \edef\newnumber{\digit\newnumber}
        \fi
        %
        \ifnum\total<\targetbase
                \edef\newnumber{\the\total\newnumber}
            \node at (\pgfmathcounter, -\pgfmathcounter) [text=red]  {\the\total};
        \else
            \node at (\pgfmathcounter, -\pgfmathcounter) {\the\total};
        \fi
        \lasttotal=\total
        \divide\total by\targetbase
    \repeatpgfmathloop    
    \draw [->] (\pgfmathcounter-1,-\pgfmathcounter-1) -- ++(-0.5,0); 
    \node [anchor=west] at (1, -\pgfmathcounter-2) {$#1=\newnumber_{\the\targetbase}$};
    \end{tikzpicture}   
}

% FRAC

\newcommand\MiniConvert[1]{\ifcase #1
	0\or 1\or 2\or 3\or 4\or 5\or 6\or 7\or 8\or 9\or A\or B\or C\or D\or E\or
	F\or G\or H\or I\or J\or K\or L\or M\or N\or O\or P\or Q\or R\or S\or T\or
	U\or V\or W\or X\or Y\or Z\else\ERROR\fi}%
\newcommand\ConvertitEnBaseB[3][25]{% #1 MUST BE OF THE 0.<decimal digits> type
	% (we can not use 1/5 because numprint's \np macro does not like the /)
	% the dot will be converted into a comma by \np macro
	% computes 25 digits by default. Abort earlier if all become zeros.
	% #3 = base < 36
	\def\ConvertiDots{\dots}%
	\noindent Número para converter para a base #3: \np{#2}.\par
	\def\Converti{0,}%<<<< LOCALIZE TO YOUR LANGUAGE
	\edef\ConvertitNombre{\xintRaw{#2}}%
	\xintiloop[1+1]
	\edef\ConvertitBFoisNombre{\xintMul{#3}{\ConvertitNombre}}%
	\edef\ConvertitBFoisNombrePartieInt
	{\xintTTrunc{\ConvertitBFoisNombre}}%
	\edef\ConvertitBFoisNombrePartieFrac
	{\xintTFrac{\ConvertitBFoisNombre}}%
	$#3\times\np{\PolDecToString{\ConvertitNombre}}
	= \boxed{\ConvertitBFoisNombrePartieInt} +
	\np{\PolDecToString{\ConvertitBFoisNombrePartieFrac}}$
	\hfill
	\llap{${}\longrightarrow{}$\MiniConvert\ConvertitBFoisNombrePartieInt}\par
	\edef\Converti{\Converti\MiniConvert{\ConvertitBFoisNombrePartieInt}}%
	\let\ConvertitNombre\ConvertitBFoisNombrePartieFrac
	\xintifZero{\ConvertitNombre}
	{\xintbreakiloopanddo\let\ConvertiDots\empty.}%
	{}%
	\ifnum#1>\xintiloopindex\space
	\repeat
	\noindent\mbox{}\hfill$\np{#2}=[$\Converti\ConvertiDots$]_{#3}$\par
}
\newcommand\ConvertitFracEnBaseB[3][25]{%
	% #1 MUST BE OR EXPAND TO A/B WITH 0 < A < B
	% computes 25 digits by default. Abort earlier if all become zeros.
	\def\ConvertiDots{\dots}%
	\edef\ConvertitNombre{\xintIrr{#2}}%
	\def\Converti{0,}%<<<< LOCALIZE TO YOUR LANGUAGE
	\noindent Número para converter para a base #3: \ConvertitNombre.\par
	\xintiloop[1+1]
	\edef\ConvertitBFoisNombre{\xintMul{#3}{\ConvertitNombre}}%
	\edef\ConvertitBFoisNombrePartieInt
	{\xintTTrunc{\ConvertitBFoisNombre}}%
	\edef\ConvertitBFoisNombrePartieFrac
	{\xintTFrac{\ConvertitBFoisNombre}}% does \xintREZ, not good for us
	$#3\times\xintFrac{\xintRawWithZeros\ConvertitNombre}
	= \boxed{\ConvertitBFoisNombrePartieInt} +
	\xintFrac{\xintRawWithZeros\ConvertitBFoisNombrePartieFrac}$\par
	\hfill
	\llap{${}\longrightarrow{}$\MiniConvert\ConvertitBFoisNombrePartieInt}\par
	\edef\Converti{\Converti\MiniConvert{\ConvertitBFoisNombrePartieInt}}%
	\let\ConvertitNombre\ConvertitBFoisNombrePartieFrac
	\xintifZero{\ConvertitNombre}
	{\xintbreakiloopanddo\let\ConvertiDots\empty.}%
	{}%
	\ifnum#1>\xintiloopindex\space
	\repeat
	\noindent\mbox{}\hfill$\xintFrac{#2}=[$\Converti\ConvertiDots$]_{#3}$\par}

% para marcar ponto na tela e desenhar sobre
\newcommand{\tikzmark}[1]{\tikz[baseline,remember picture] \coordinate (#1) {};}

% unidades uteis para siuntix

%\DeclareSIUnit{\lbf}{lbf}
%\DeclareSIUnit{\kgf}{kgf}
%\DeclareSIUnit{\kgfp}{\kgf \per \centi\meter\squared}
%\DeclareSIUnit{\mca}{mca}
%\DeclareSIUnit{\barp}{bar}
%\DeclareSIUnit{\pol}{pol}
%\DeclareSIUnit{\psip}{psi}
%\DeclareSIUnit{\atm}{atm}
%\DeclareSIUnit{\HP}{HP}
%\DeclareSIUnit{\psid}{\lbf\per\pol\squared}
%\DeclareSIUnit{}{}

% Usar para diminuir espaçamento antes/depois da equação (\useshortskip)
%\usepackage{nccmath}
%\usepackage{xpatch}
%\xpatchcmd{\NCC@ignorepar}{%
%	\abovedisplayskip\abovedisplayshortskip}
%{%
%	\abovedisplayskip\abovedisplayshortskip%
%	\belowdisplayskip\belowdisplayshortskip}
%{}{}

\usepackage{chngcntr}
\counterwithin*{equation}{section}
\newcounter{saveenumi}
\newcommand{\saveenumerate}{%
	\stepcounter{saveenumi}%
	\label{saveenumi-\thesaveenumi}}
\newcommand{\restoreenumerate}{%
	\setcounterref{enumi}{saveenumi-\thesaveenumi}}

% Símbolo para graus (usar \ang{degrees} do package siunitx)
%\newcommand{\degree}{\ensuremath{^\circ}}

% X de errado e certinho correspondente
\usepackage{pifont}
\newcommand{\cmark}{\ding{51}}%
\newcommand{\xmark}{\ding{55}}%

% \itemequation[label]{text before}{equation}

%\makeatletter
%\newcommand*{\itemequation}[3][]{%
%  \item
%  \begingroup
%    \refstepcounter{equation}%
%    \ifx\\#1\\%
%    \else
%      \label{#1}%
%    \fi
%    \sbox0{#2}%
%    \sbox2{$\displaystyle#3\m@th$}%
%    \sbox4{ \@eqnnum}%
%    \dimen@=.5\dimexpr\linewidth-\wd2\relax
%    % Warning for overlapping
%    \let\CenterInSpace=N%
%    \ifcase
%    \ifdim\wd0>\dimen@
%          \z@
%        \else
%          \ifdim\wd4>\dimen@
%            \z@
%          \else
%            \@ne
%          \fi
%        \fi
%      \let\CenterInSpace=Y%
%    \fi
%    \ifdim\dimexpr\wd0+\wd2+\wd4\relax>\linewidth
%      \@latex@warning{Equation is too large}%
%    \fi
%    \noindent
%    \rlap{\copy0}%
%    \ifx\CenterInSpace Y%
%      \rlap{\hbox to \linewidth{\kern\wd0\hss\copy2\hss\kern\wd4}}%
%    \else
%      \rlap{\hbox to \linewidth{\hfill\copy2\hfill}}%
%    \fi
%    \hbox to \linewidth{\hfill\copy4}%
%    \hspace{0pt}% allow linebreak
%  \endgroup
%  \ignorespaces
%}
%\makeatother


%%%%%%%%%%%%%%%%%%%%%%%%%%%%%%%%%%%%%%%%%%%
%%%%%%%%%%%%%%%%%%%%%%%%%%%%%%%%%%%%%%%%%%%
%%%%%%%%%%%%%%%%%% RODAPÉ %%%%%%%%%%%%%%%%%

\setbeamercolor{footline}{fg=white}
\setbeamertemplate{footline}
{\begin{tikzpicture}
    \node [inner sep=0pt, anchor=east] (0,0) {\includegraphics[width=\paperwidth,height=1cm]{Figuras/Capa/macaefooter.png}};
    \node [inner sep=0pt, anchor=east] at (-2ex,-3ex) {\insertframenumber{} / \inserttotalframenumber};
\end{tikzpicture}}


%%%%%%%%%%%%%%%%%%%%%%%%%%%%%%%%%%%%%%%%%%%
%%%%%%%%%%%%%%%%%%%%%%%%%%%%%%%%%%%%%%%%%%%
%%%%%%%%%%% INFORMAÇÕES DO CURSO %%%%%%%%%%

\title[Eletrônica Digital]
{Eletrônica Digital}

\subtitle {Notas de aula}

\author{Prof. Yago Pessanha Corrêa}

\institute[MSP/IFF] 
{
	Laboratório de Mecatrônica e Processamento de Sinais (MSP) \\
	Instituto Federal de Educação, Ciência e Tecnologia Fluminense (IFFluminense) \\
	Curso Técnico em Automação \\
	\vspace*{.1cm} {\tt \textbf{yago.correa@iff.edu.br}}\\
}

%\tddate


%%%%%%%%%%%%%%%%%%%%%%%%%%%%%%%%%%%%%%%%%%%
%%%%%%%%%%%%%%%%%%%%%%%%%%%%%%%%%%%%%%%%%%%
%%%%%%%%%%%%%%%%% SUMÁRIO %%%%%%%%%%%%%%%%%

\AtBeginSection[]
{
  \begin{frame}<beamer>{Sumário}
    \tableofcontents[currentsection]
  \end{frame}
}