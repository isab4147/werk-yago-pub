\section{Simplificação de expressões booleanas por mapa de Karnaugh}

\frame{
	\frametitle{Mapa de Karnaugh}
	\begin{block}{Mapeamento da tabela verdade}
		\begin{itemize}
			\item Circuito mínimo.
			\item Fácil utilização.
		\end{itemize}
	\end{block}
}

\frame{
	\frametitle{K-map - Duas variáveis}
	\begin{center}
	\begin{karnaugh-map}[2][2][1][$A$][$B$]
	\minterms{}
	\maxterms{}
	%\implicant{1}{1}
	%\implicant{2}{2}
	\end{karnaugh-map}
	\end{center}
}

\frame{
	\frametitle{K-map - Três variáveis}
	\begin{center}
	\begin{karnaugh-map}[4][2][1][$AB$][$C$]
	\minterms{}
	\maxterms{}
	\indeterminants{}
	%\implicant{3}{2}
	%\implicant{4}{5}
	\end{karnaugh-map}
	\end{center}
}

\frame{
	\frametitle{K-map - Quatro variáveis}
	\begin{center}
	\begin{karnaugh-map}[4][4][1][$AB$][$CD$]
	%\manualterms{0,0,0,0,0,0,0,0,0,0,0,0,0,0,0,0}
	%\implicant{0}{2}
	%\implicant{5}{15}
	%\implicantedge{1}{3}{9}{11}
	%\implicantcorner
	%\implicantedge{4}{12}{6}{14}
	\end{karnaugh-map}
	\end{center}
}

\frame{
	\frametitle{Mapa de Karnaugh x Tabela verdade}
	\begin{block}{Atenção à \textbf{ORDEM}}
		\begin{itemize}
			\item \textbf{Troca de 10 e 11}.
			\item Em células adjacentes apenas uma variável pode mudar de valor.
		\end{itemize}
	\end{block}
}

\frame{
	\frametitle{Como montar?}
	\begin{block}{Montagem do K-map}
		Podemos utilizar \alert{mintermos} ou \alert{maxtermos}.
		
		\bigskip
		
		Passo-a-passo:
		\begin{enumerate}
			\item Preencher as células de acordo com a tabela verdade.
			\item Agrupar os 1's (mintermos) ou 0's (maxtermos):
			      \begin{enumerate}
				      \item\normalsize Unir na horizontal ou vertical.
				      \item\normalsize \textbf{NUNCA} na diagonal.
				      \item\normalsize \textbf{PODE USAR} os cantos.
				      \item\normalsize Deve ser múltiplos da \textbf{BASE 2}.
				      \item\normalsize {\color{red} \textbf{MAIOR AGRUPAMENTO POSSÍVEL}}.
			      \end{enumerate}
		\end{enumerate}
	\end{block}
}

\frame{
	\frametitle{Como agrupar? - Adjacências}
	\begin{center}
	\begin{karnaugh-map}[2][2][1][$A$][$B$]
	\minterms{2,3}
	\maxterms{0,1}
	%\implicant{1}{1}
	%\implicant{2}{2}
	\end{karnaugh-map}
	\end{center}
}

\frame{
	\frametitle{Como agrupar? - Adjacências}
	\begin{center}
	\begin{karnaugh-map}[2][2][1][$A$][$B$]
	\minterms{2,3}
	\maxterms{0,1}
	\implicant{2}{3}
	%\implicant{2}{2}
	\end{karnaugh-map}
	\end{center}
}

\frame{
	\frametitle{Como agrupar? - Adjacências}
	\begin{center}
	\begin{karnaugh-map}[4][2][1][$AB$][$C$]
	\minterms{4,5,3,2}
	\maxterms{0,1,6,7}
	\indeterminants{}
	%\implicant{3}{2}
	%\implicant{4}{5}
	\end{karnaugh-map}
	\end{center}
}

\frame{
	\frametitle{Como agrupar? - Adjacências}
	\begin{center}
	\begin{karnaugh-map}[4][2][1][$AB$][$C$]
	\minterms{4,5,3,2}
	\maxterms{0,1,6,7}
	\indeterminants{}
	\implicant{4}{5}
	\implicant{3}{2}
	\end{karnaugh-map}
	\end{center}
}

\frame{
	\frametitle{Como agrupar? - Linhas ou colunas}
	\begin{center}
	\begin{karnaugh-map}[4][4][1][$AB$][$CD$]
	\manualterms{1,1,1,1,1,0,0,0,1,0,0,0,1,0,0,0}
	%\implicant{0}{2}
	%\implicant{5}{15}
	%\implicantedge{1}{9}{9}{11}
	%\implicantcorner
	%\implicantedge{4}{12}{6}{14}
	\end{karnaugh-map}
	\end{center}
}

\frame{
	\frametitle{Como agrupar? - Linhas ou colunas}
	\begin{center}
	\begin{karnaugh-map}[4][4][1][$AB$][$CD$]
	\manualterms{1,1,1,1,1,0,0,0,1,0,0,0,1,0,0,0}
	\implicant{0}{2}
	\implicant{0}{8}
	%\implicantedge{1}{9}{9}{11}
	%\implicantcorner
	%\implicantedge{4}{12}{6}{14}
	\end{karnaugh-map}
	\end{center}
}

\frame{
	\frametitle{Como agrupar? - Quadra}
	\begin{center}
	\begin{karnaugh-map}[4][4][1][$AB$][$CD$]
	\manualterms{1,1,0,0,1,1,0,0,0,0,1,1,0,0,1,1}
	%\implicant{0}{2}
	%\implicant{5}{15}
	%\implicantedge{1}{9}{9}{11}
	%\implicantcorner
	%\implicantedge{4}{12}{6}{14}
	\end{karnaugh-map}
	\end{center}
}

\frame{
	\frametitle{Como agrupar?}
	\begin{center}
	\begin{karnaugh-map}[4][4][1][$AB$][$CD$]
	\manualterms{1,1,0,0,1,1,0,0,0,0,1,1,0,0,1,1}
	\implicant{0}{5}
	\implicant{15}{10}
	%\implicantedge{1}{9}{9}{11}
	%\implicantcorner
	%\implicantedge{4}{12}{6}{14}
	\end{karnaugh-map}
	\end{center}
}

\frame{
	\frametitle{Como agrupar? - Quadra em anel}
	\begin{center}
	\begin{karnaugh-map}[4][4][1][$AB$][$CD$]
	\manualterms{1,0,1,0,0,0,0,0,1,0,1,0,0,0,0,0}
	%\implicant{0}{2}
	%\implicant{5}{15}
	%\implicantedge{1}{9}{9}{11}
	%\implicantcorner
	%\implicantedge{4}{12}{6}{14}
	\end{karnaugh-map}
	\end{center}
}

\frame{
	\frametitle{Como agrupar? - Quadra em anel}
	\begin{center}
	\begin{karnaugh-map}[4][4][1][$AB$][$CD$]
	\manualterms{1,0,1,0,0,0,0,0,1,0,1,0,0,0,0,0}
	%\implicant{1}{9}{4}{6}
	%\implicant{5}{15}
	%\implicantedge{1}{9}{9}{11}
	\implicantcorner
	%\implicantedge{4}{12}{6}{14}
	\end{karnaugh-map}
	\end{center}
}

\frame{
	\frametitle{Como agrupar? - Quadra em anel}
	\begin{center}
	\begin{karnaugh-map}[4][4][1][$AB$][$CD$]
	\manualterms{0,1,0,1,1,0,1,0,0,1,0,1,1,0,1,0}
	%\implicant{0}{2}
	%\implicant{5}{15}
	%\implicantedge{1}{9}{9}{11}
	%\implicantcorner
	%\implicantedge{4}{12}{6}{14}
	\end{karnaugh-map}
	\end{center}
}

\frame{
	\frametitle{Como agrupar? - Quadra em anel}
	\begin{center}
	\begin{karnaugh-map}[4][4][1][$AB$][$CD$]
	\manualterms{0,1,0,1,1,0,1,0,0,1,0,1,1,0,1,0}
	%\implicant{0}{2}
	%\implicant{5}{15}
	\implicantedge{1}{3}{9}{11}
	%\implicantcorner
	\implicantedge{4}{12}{6}{14}
	\end{karnaugh-map}
	\end{center}
}

\frame{
	\frametitle{Como agrupar? - Oitava}
	\begin{center}
	\begin{karnaugh-map}[4][4][1][$AB$][$CD$]
	\manualterms{0,1,0,1,1,1,1,1,0,1,0,1,1,1,1,1}
	%\implicant{0}{2}
	%\implicant{5}{15}
	%\implicantedge{1}{9}{9}{11}
	%\implicantcorner
	%\implicantedge{4}{12}{6}{14}
	\end{karnaugh-map}
	\end{center}
}

\frame{
	\frametitle{Como agrupar? - Oitava}
	\begin{center}
	\begin{karnaugh-map}[4][4][1][$AB$][$CD$]
	\manualterms{0,1,0,1,1,1,1,1,0,1,0,1,1,1,1,1}
	\implicant{4}{14}
	\implicant{1}{11}
	%\implicantedge{1}{9}{9}{11}
	%\implicantcorner
	%\implicantedge{4}{12}{6}{14}
	\end{karnaugh-map}
	\end{center}
}

\frame{
	\frametitle{Diagramas com condições irrelevantes}
	\begin{center}
	\begin{karnaugh-map}[4][4][1][$AB$][$CD$]
	\manualterms{X,0,1,X,1,0,1,1,0,1,X,0,0,X,0,X}
	%\implicant{0}{2}
	%\implicant{5}{15}
	%\implicantedge{1}{9}{9}{11}
	%\implicantcorner
	%\implicantedge{4}{12}{6}{14}
	\end{karnaugh-map}
	\end{center}
}

\frame{
	\frametitle{Diagramas com condições irrelevantes}
	\begin{center}
	\begin{karnaugh-map}[4][4][1][$AB$][$CD$]
	\manualterms{X,0,1,X,1,0,1,1,0,1,X,0,0,X,0,X}
	\implicant{13}{9}
	\implicant{3}{6}
	\implicantedge{0}{4}{2}{6}
	%\implicantcorner
	%\implicantedge{4}{12}{6}{14}
	\end{karnaugh-map}
	\end{center}
}

\frame{
\frametitle{Obtendo a expressão simplificada - Exemplo \#01}

\begin{block}{}
	\begin{itemize}
		\item A variável que \textbf{muda de estado }é \textbf{retirada}.
	\end{itemize}

	\[ S = A\cdot \notted{B} + A\cdot B + \notted{A}\cdot B \]
\end{block}

\centering
\begin{karnaugh-map}[2][2][1][$A$][$B$]
\minterms{1,2,3}
\maxterms{0}
\implicant{1}{3}
\implicant{2}{3}
\end{karnaugh-map}

\vspace{-0.5cm}

\begin{block}{Expressão simplificada}
	\[ S = {\color{red} A} + {\color{green} B} \]
\end{block}
}

\frame{
	\frametitle{Obtendo a expressão simplificada - Exemplo \#02}
	
	\begin{block}{}
		
		\[ S = A\cdot B\cdot \notted{C}\cdot  \notted{D} + A\cdot B\cdot \notted{C}\cdot D + A\cdot B\cdot C\cdot D + A\cdot B\cdot C\cdot \notted{D} \]
	\end{block}
	
	\centering
	\scalebox{0.9}{
		\begin{karnaugh-map}[4][4][1][$AB$][$CD$]
			
			\manualterms{0,0,0,0,0,0,0,0,0,0,0,0,1,1,1,1}
			\implicant{12}{14}
			%\implicant{3}{6}
			%\implicantedge{0}{4}{2}{6}
			%\implicantcorner
			%\implicantedge{4}{12}{6}{14}
		\end{karnaugh-map}
	}

	\vspace{-0.8cm}
	
	\begin{block}{Expressão simplificada}
		\[  S = {\color{red} A\cdot B} \]
	\end{block}
}

\frame{
	\frametitle{Obtendo a expressão simplificada - Exemplo \#03}
	
	\begin{block}{}
		
		\[ S = \notted{A}\cdot \notted{B}\cdot \notted{C}\cdot \notted{D} + \notted{A}\cdot  \notted{B}\cdot C\cdot \notted{D} + A\cdot \notted{B}\cdot \notted{C}\cdot \notted{D} + A\cdot \notted{B}\cdot C\cdot \notted{D} \]
	\end{block}
	
	\centering
	\scalebox{0.9}{
		\begin{karnaugh-map}[4][4][1][$AB$][$CD$]
			\manualterms{1,0,1,0,0,0,0,0,1,0,1,0,0,0,0,0}
			%\implicant{12}{15}
			%\implicant{3}{6}
			%\implicantedge{0}{2}{8}{10}
			\implicantcorner
			%\implicantedge{4}{12}{6}{14}
		\end{karnaugh-map}
	}

	\vspace{-0.8cm}
	
	\begin{block}{Expressão simplificada}
		\[ S = {\color{red} \notted{B}\cdot \notted{D}} \]
	\end{block}
}

\frame{
	\frametitle{Obtendo a expressão simplificada - Exemplo \#03}
	
	\begin{block}{}
		
		\[ S = \notted{A}\cdot \notted{B}\cdot \notted{C}\cdot \notted{D} + \notted{A}\cdot  \notted{B}\cdot C\cdot \notted{D} + A\cdot \notted{B}\cdot \notted{C}\cdot \notted{D} + A\cdot \notted{B}\cdot C\cdot \notted{D} \]
	\end{block}
	
	\centering
	\scalebox{0.9}{
		\begin{karnaugh-map}[4][4][1][$AB$][$CD$]
			\manualterms{1,0,1,0,0,0,0,0,1,0,1,0,0,0,0,0}
			%\implicant{12}{15}
			%\implicant{3}{6}
			%\implicantedge{0}{2}{8}{10}
			\implicantcorner
			%\implicantedge{4}{12}{6}{14}
		\end{karnaugh-map}
	}

	\vspace{-0.8cm}
	
	\begin{block}{Expressão simplificada}
		\[ S = {\color{red} \notted{B}\cdot \notted{D}} \]
	\end{block}
}

\frame{
	\frametitle{Obtendo a expressão simplificada - Exemplo \#04}
		\centering
		\begin{adjustbox}{totalheight=0.9\textheight-2\baselineskip}
			\begin{tabular}{cccc|c}
				\toprule
				A & B & C & D & S \\ \midrule
				0 & 0 & 0 & 0 & X \\
				0 & 0 & 0 & 1 & 0 \\
				0 & 0 & 1 & 0 & 1 \\
				0 & 0 & 1 & 1 & X \\
				0 & 1 & 0 & 0 & 1 \\
				0 & 1 & 0 & 1 & 0 \\
				0 & 1 & 1 & 0 & 1 \\
				0 & 1 & 1 & 1 & 1 \\
				1 & 0 & 0 & 0 & 0 \\
				1 & 0 & 0 & 1 & 1 \\
				1 & 0 & 1 & 0 & X \\
				1 & 0 & 1 & 1 & 0 \\
				1 & 1 & 0 & 0 & 0 \\
				1 & 1 & 0 & 1 & X \\
				1 & 1 & 1 & 0 & 0 \\
				1 & 1 & 1 & 1 & X \\ \bottomrule
			\end{tabular}
		\end{adjustbox}
}

\frame{
	\frametitle{Obtendo a expressão simplificada - Exemplo \#04}
	
	\centering
	\scalebox{0.9}{
		\begin{karnaugh-map}[4][4][1][$AB$][$CD$]
			\manualterms{X,1,0,0,0,0,1,X,1,1,X,0,X,1,0,X}
			\implicant{12}{9}
			\implicant{7}{6}
			\implicantedge{0}{1}{8}{9}
			%\implicantcorner
			%\implicantedge{4}{12}{6}{14}
		\end{karnaugh-map}
	}
	
	\begin{block}{Expressão simplificada}
		\[ S = {\color{red} \notted{A}\cdot C} + {\color{green} A\cdot \notted{C}\cdot D} + {\color{YellowOrange} \notted{A}\cdot \notted{D}} \]
	\end{block}
}

\frame{
	\frametitle{Obtendo a expressão simplificada - Exemplo \#05}
	
	\centering
	\scalebox{0.9}{
		\begin{karnaugh-map}[4][4][1][$AB$][$CD$]
			\manualterms{0,0,0,1,1,1,1,1,0,0,0,0,1,1,0,0}
			\implicant{4}{6}
			\implicant{4}{13}
			\implicant{3}{7}
			%\implicantedge{0}{2}{8}{10}
			%\implicantcorner
			%\implicantedge{4}{12}{6}{14}
		\end{karnaugh-map}
	}
	
	\begin{block}{Expressão simplificada}
		\[ S = {\color{red} \notted{C}\cdot D} + {\color{green} \notted{A}\cdot D} + {\color{YellowOrange} A\cdot B\cdot \notted{C}} \]
	\end{block}
}


\section*{Exercícios}

\frame{
	\frametitle{Exercícios}
	\begin{block}{}
		01. Simplifique as expressões booleanas a seguir pelo Mapa de Karnaugh:
		\begin{itemize}
			\item $S = \notted{C}\cdot(\notted{A}\cdot \notted{B}\cdot \notted{D} + D) + A\cdot \notted{B}\cdot C + \notted{D}$
			\item $S = \notted{A}\cdot \notted{B}\cdot \notted{C } + \notted{A}\cdot B\cdot \notted{C} + \notted{A}\cdot B\cdot C + A\cdot \notted{B}\cdot \notted{C} + A\cdot B\cdot \notted{C}$
			\item $S = \notted{A}\cdot \notted{B}\cdot \notted{C} + \notted{A}\cdot B\cdot \notted{C} + \notted{A}\cdot B\cdot C + A\cdot B\cdot C$
		\end{itemize}
	\end{block}
}


\section*{Referências}

\frame{
	\frametitle{Referências e exercícios complementares}
	\begin{itemize}
		\item IDOETA, Ivan V. e CAPUANO, Francisco G. Elementos de Eletrônica Digital. São Paulo:
		      Editora Érica, ed. 40. 2008.
	\end{itemize}
	\centering{\alert{Página 149 - \textbf{3.10.9 até 3.10.15}}}
}