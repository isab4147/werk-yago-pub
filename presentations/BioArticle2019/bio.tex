\documentclass[article,11pt,oneside,a4paper,
	english,	% idioma adicional para hifenização
	brazil,		% o último idioma é o principal do documento
	sumario=tradicional]{abntex2}

\usepackage{lmodern}			% Usa a fonte Latin Modern
\usepackage[T1]{fontenc}		% Selecao de codigos de fonte.
\usepackage[utf8]{inputenc}		% Codificacao do documento (conversão automática dos acentos)
\usepackage{indentfirst}		% Indenta o primeiro parágrafo de cada seção.
\usepackage{nomencl} 			% Lista de simbolos
\usepackage{color}				% Controle das cores
\usepackage{microtype} 			% para melhorias de justificação
\usepackage[brazilian,hyperpageref]{backref}	 % Paginas com as citações na bibl
\usepackage[alf]{abntex2cite}	% Citações padrão ABNT

\usepackage{graphicx}
\usepackage[version=4]{mhchem}
\usepackage{siunitx}
\DeclareSIUnit\ppm{ppm}


% ---
% Configurações do pacote backref
% Usado sem a opção hyperpageref de backref
\renewcommand{\backrefpagesname}{Citado na(s) página(s):~}
% Texto padrão antes do número das páginas
\renewcommand{\backref}{}
% Define os textos da citação
\renewcommand*{\backrefalt}[4]{
	\ifcase #1 %
		Nenhuma citação no texto.%
	\or
		Citado na página #2.%
	\else
		Citado #1 vezes nas páginas #2.%
	\fi}%

\definecolor{blue}{RGB}{41,5,195} % alterando o aspecto da cor azul

% informações do PDF
\makeatletter
\hypersetup{
     	%pagebackref=true,
		pdftitle={\@title}, 
		pdfauthor={\@author},
    	pdfsubject={Modelo de artigo científico com abnTeX2},
	    pdfcreator={LaTeX with abnTeX2},
		pdfkeywords={abnt}{latex}{abntex}{abntex2}{atigo científico}, 
		colorlinks=true,       		% false: boxed links; true: colored links
    	linkcolor=blue,          	% color of internal links
    	citecolor=blue,        		% color of links to bibliography
    	filecolor=magenta,      		% color of file links
		urlcolor=blue,
		bookmarksdepth=4
}
\makeatother

% Altera as margens padrão
\setlrmarginsandblock{1in}{1in}{*}
\setulmarginsandblock{1in}{1in}{*}
\checkandfixthelayout

\setlength{\parindent}{1.3cm} % tamanho do parágrafo

% Controle do espaçamento entre um parágrafo e outro:
\setlength{\parskip}{0.2cm}  % tente também \onelineskip

\SingleSpacing

\titulo{Efeitos do \texorpdfstring{\ce{CO2}}{CO2} na atmosfera e na saúde}
\tituloestrangeiro{} % titulo em outro idioma (opcional)

\autor{Isabella Basso \and
	João Pedro Lacerda \and
	Juliana Oliveira \and
	Maria Cecília Nascimento \and
	Maria Eduarda Santos
}

\local{Campos dos Goytacazes, Brasil}
\data{23 de junho de 2019}

\makeindex

\begin{document}

%\selectlanguage{english}
\selectlanguage{brazil}

\frenchspacing

% página de titulo principal (obrigatório)
\maketitle

% resumo em português
\begin{resumoumacoluna}
	Há mais de um século o homem vem alterando a composição da atmosfera de forma destrutiva, tanto para si quanto para as outras espécies que habitam o planeta Terra. Nesse artigo explorar-se-ão -- através de uma apresentação concisa, pensada para um público relativamente leigo -- os efeitos do gás carbônico (\ce{CO2}) na saúde e no meio ambiente com o uso de maquetes e cartazes explicativos.

	\vspace{\onelineskip}

	\noindent
	\textbf{Palavras-chave}: Efeitos do dióxido de carbono. Projeto escolar. Saúde. Meio ambiente. Aquecimento global.
\end{resumoumacoluna}


\begin{center}\smaller
	% elemento obrigatório. Indicar dia, mês e ano
	\textbf{Data de submissão e aprovação}:

	% elemento opcional. Pode ser indicado o endereço eletrônico, DOI, suportes e outras informações relativas ao acesso.
	%\textbf{Identificação e disponibilidade}: 
\end{center}

\textual

% Fundamentação teórica + justificativa
\section{Introdução}

Na Primeira Revolução Industrial, ocorrida no séc. XVIII, o homem começou a queimar combustíveis fosseis em quantidades significativas que, dentro de dois séculos -- tempo irrelevante na perspectiva geológica --, já tiveram seus efeitos notados \cite{NASAevi,Hansen2010}. O efeito mais imediato da queima de combustíveis fosseis na atmosfera é o chamado \textit{Efeito Estufa}, o qual recebe seu nome devido à ligação imediata que possui com a temperatura em um dado espaço.


\subsection{Efeito Estufa}

O efeito estufa é, por definição, o processo pelo qual a superfície de um planeta se aquece na presença de uma atmosfera que recebeu radiação, sendo esse aquecimento superior àquele pelo qual a superfície passaria na ausência de uma atmosfera.

Na Terra, nossa atmosfera absorve e re-emite cerca de $ 70\% $ da energia solar que a atinge \cite{NASA2009}, a porcentagem que incide sobre a superfície é, majoritariamente, luz visível e que, então, é absorvida e re-emitida como calor. Esse calor fica preso na atmosfera pelos gases do efeito estufa, que o emitirão para fora da atmosfera eventualmente.

\subsection{Ação Humana}

Devido à supracitada queima de combustíveis fosseis promovida pelo homem, a quantidade de \ce{CO2} aumentou em $ 40\% $ desde a Revolução Industrial \cite{WMO2014,Stuiver1984}, o que levou a um aumento na temperatura do globo em cerca de \SI{1}{\degreeCelsius} desde, aproximadamente, 1900 \cite{NASAevi,Hansen2010}.


\subsection{Efeitos no homem}

Graças ao aumento significativo do uso dos aparelhos ares condicionados (e também a previsão de maior demanda futura) \cite{Agency2018}, segue que haverá uma maior demanda de energia elétrica \cite{Agency2016}, o que pode provocar ainda mais o aumento de gás carbônico. Haverá também efeitos na saúde causados pelas altas concentrações de \ce{CO2} no ar \cite{Allen2016,Stafford2015}. Esse aumento na concentração se deve ao fato de a população global, em média, passar mais tempo em ambientes fechados e, portanto, onde não há a devida "reciclagem" do ar respirado.

Os efeitos se tornam visíveis a níveis tão baixos quanto \SI{1000}{\ppm} (perda de $ 15\% $ das funções cognitivas), facilmente encontrados em ambientes fechados mal-ventilados/climatizados \cite{Allen2016,Stafford2015}.


\section{Objetivos}

Esse projeto espera conscientizar o público geral dos efeitos do \ce{CO2} na saúde e como de fato ocorre o processo conhecido como \textit{efeito estufa}.

% Materiais usados
\section{Metodologia}

\subsection{Materiais}

Uma maquete representará dois ambientes (cidade e campo) e deve consistir de:
\begin{itemize}
	\item $ 2\times $ aquários pequenos;
	\item Isolante térmico de E.V.A. laminado em alumínio refletivo;
	\item $ 2\times $ lâmpadas incandescentes;
	\item $ 2\times $ abajures;
	\item Arduino Mega 2560 Rev3;
	\item $ 3\times $ módulo LCD $ 16\times2 $ 1602 I2C;
	\item $ 3\times $ sensores de temperatura e umidade DHT22;
	\item $ 2\times $ sensores de dióxido de carbono (\ce{CO2}) MH-Z14A;
	\item Planta;
	\item \SI{150}{\gram} de bicarbonato de sódio (\ce{NaHCO3});
	\item \SI{150}{\gram} de fermento biológico;
	\item \SI{500}{\gram} de açúcar branco, e;
	\item \SI{1}{\liter} de água;
\end{itemize}
além de materiais artísticos diversos para representação adequada dos ambientes, e cabos e conexões eletrônicas.

A outra maquete será meramente a representação de um ambiente fechado.

Haverá também cartazes decorados para auxílio na explicação.

\subsection{Métodos}

O projeto será montado sobre uma mesa com os aquários -- dispostos lado a lado e isolados entre si com o isolante térmico de E.V.A. -- representando os ambientes:
\begin{enumerate}
	\item \textbf{Cidade}

	      Os níveis de \ce{CO2} serão elevados a níveis aprox. \SI{150}{\ppm} acima dos da atmosfera (\SI{411}{\ppm} na média global atual\footnote{Dado de maio de 2019, retirado de \url{https://climate.nasa.gov/vital-signs/carbon-dioxide/}.}) através da injeção do gás.

	\item \textbf{Campo}

	      Idealmente, os níveis de \ce{CO2} deverão ser reduzidos a níveis pré industriais (aprox. \SI{265}{\ppm} na média global \cite{Wigley1983}), utilizando o processo da fotossíntese.
	      %	Por questões práticas, usar-se-á a concentração local de gás carbônico.
\end{enumerate}

Dentro dos aquários os sensores serão dispostos em pares, sendo o terceiro sensor DHT22 posicionado em outro local, para referência de zero. Ao lado dos aquários estarão os abajures com as lâmpadas incandescentes, uma para cada aquário, a fim de iluminá-los igualmente. A frente dos aquários estarão o Arduino Mega e os módulos LCD.

A outra parte do projeto, que ilustra os malefícios do gás carbônico à saúde, também será montada ao lado dos aquários, com a maquete do ambiente fechado a frente dos cartazes, que deverão conter dados estatísticos e gráficos.


\subsubsection{Preparo dos aquários}

Estes procedimentos descrevem como os aquários serão preparados para apresentação e devem ser realizados, idealmente, uma vez, antes do começo das apresentações.

\begin{itemize}
	\item Geração de \ce{CO2}

	      O \ce{CO2} para injeção será gerado através da reação resultante da mistura de:
	      \begin{itemize}
		      \item açúcar branco (1:2 \si{\kilo\gram} em relação ao volume de água em litro);
		      \item bicarbonato de sódio -- \ce{NaHCO3} (1:150 gramas em relação ao volume de água em litro), e;
		      \item fermento biológico (1:150 gramas em relação ao volume de água em litro);
	      \end{itemize}
	      em água num recipiente fechado (garrafa PET).

	      A injeção será fracionada medindo a concentração dentro do recipiente a cada porção adicionada, até que se chegue a quantidade especificada acima.

	\item Retirada de \ce{CO2}

	      Através do processo de fotossíntese pretende-se remover o \ce{CO2} de um dos aquários. Esse processo será realizado introduzindo-se uma planta dentro do aquário, por baixo (aquário com abertura para baixo), a fim de que não entre o gás dentro do compartimento.

	      O conjunto planta $ + $ aquário deverá ficar ao ar livre durante os dias não chuvosos antecedendo alguns dias à apresentação, a fim de permitir que haja o tempo devido para o procedimento ocorra.

	      Os níveis de \ce{CO2} serão medidos durante todo o tempo até chegarem a quantidade especificada acima.
\end{itemize}

\subsubsection{Controle de variáveis}

As aferições de \ce{CO2} dentro dos aquários serão feitas com o uso de sensores MH-Z14A, cada qual posicionado em altura média na parede de seu respectivo aquário. Ao lado dos sensores de gás carbônico serão posicionados sensores DHT22, que devem medir a temperatura (variável dependente) no projeto.

A temperatura será controlada através de um sensor externo aos aquários, posicionado em local com interferência humana ou eletrônica mínima, para referência de zero da temperatura do ar.

Não havendo alterações significativas em temperatura ou composição química do ar nos aquários, são dispensadas outras medições.

Além disso, como o aquário será lacrado, não há necessidade do controle de \ce{CO2} externo.


\subsubsection{Coleta de dados}

Os dados disponibilizados pelos sensores no projeto serão coletados pelo Arduino Mega e serão mostrados ao público através dos sensores LCD, os quais ficarão a mostra e rotulados (de acordo com o que medem) para a visualização do público.



\section{Resultados esperados}

Espera-se que o público note a diferença entre as duas simulações de aquecimento, assim como sua conscientização do papel maléfico do excesso da concentração do gás carbônico em ambientes fechados em relação à saúde pública.



% Finaliza a parte no bookmark do PDF, para que se inicie o bookmark na raiz
\bookmarksetup{startatroot}% 

% Conclusão
%\section{Considerações finais}



\postextual

\bibliography{biobib}


% Glossário
% Há diversas soluções prontas para glossário em LaTeX. 
% Consulte o manual do abnTeX2 para obter sugestões.
%
%\glossary

% Apêndices
%\begin{apendicesenv}
%
%
%\chapter{Nullam elementum urna vel imperdiet sodales elit ipsum pharetra ligula
%ac pretium ante justo a nulla curabitur tristique arcu eu metus}
%
%
%\end{apendicesenv}

% Anexos
%\cftinserthook{toc}{AAA}

%\anexos
%\begin{anexosenv}
%
%\chapter{Cras non urna sed feugiat cum sociis natoque penatibus et magnis dis
%parturient montes nascetur ridiculus mus}
%
%
%\end{anexosenv}


% Agradecimentos
%\section*{Agradecimentos}
%Texto sucinto aprovado pelo periódico em que será publicado. Último 
%elemento pós-textual.

\end{document}