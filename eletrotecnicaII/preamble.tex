\mode<presentation>
{
	\usetheme{JuanLesPins}
	\usefonttheme{serif}
	\usecolortheme{beaver}
	\setbeamercovered{invisible} \setbeamertemplate{blocks}[rounded][shadow=true] 
	\setbeamertemplate{navigation symbols}{} 
	\setbeamertemplate{footline}[frame number]
	\usecolortheme[RGB={122,4,24}]{structure}
}

\setcounter{tocdepth}{1}

%\usepackage{fancybox}
%\usepackage{graphicx}
%\usepackage{colortbl}
%\usepackage{textcomp}
%\usepackage{multirow}
%\usepackage{calligra}
%\usepackage{srcltx}
%\usepackage{enumerate}
%\usepackage{refcount}

%%%%%%%%%%%%%%%%%%%%%%%%%%%%%%%%%%%%%%%%%%%
%%%%%%%%%%%%%%%%%%%%%%%%%%%%%%%%%%%%%%%%%%%
%%%%%%%%%%%%%%%%% PACKAGES %%%%%%%%%%%%%%%%%

% BASICAO

\usepackage{lmodern}
\usepackage[T1]{fontenc}
\usepackage[useregional]{datetime2}

% Para usar com PDFTEX
%\usepackage[brazilian]{babel}
%\usepackage[utf8]{inputenc}

% Para usar com XELATEX
\usepackage{polyglossia}
\setdefaultlanguage{brazil}


% FIGURAS
\usepackage{graphicx,float}
\usepackage{booktabs, longtable}
\usepackage{pdfpages}


% TABELAS E DIAGRAMAÇÃO
\usepackage{enumerate}
\usepackage{multirow,multicol}
\usepackage{makecell}
\usepackage{array}
%\usepackage{fancybox}
%\usepackage{colortbl}


% ?????????????

% Fonte Courier?
%\usepackage{courier}

% Simbolos
%\usepackage{textcomp}

% Texto caligrafico
%\usepackage{calligra}

% Conexão entre pdfs e dvis
%\usepackage{srcltx}

% Para delimitar environments de tipo float
%\usepackage{cprotect}

% Para texto literal (programação)
%\usepackage{verbatim}

% Pegar números relacionados a referências para uso
%\usepackage{refcount}

% Para colocar relógios de ponteiro
%\usepackage{tdclock}

%%%%%%%%%%%%%%%%%%%%%%%%%%%%%%%%%%%%%%%%%%%
%%%%%%%%%%%%%%%%%%%%%%%%%%%%%%%%%%%%%%%%%%%
%%%%%%%%%%%%%%%%% COISAS MATEMATICAS %%%%%%%%%%%%%%%%%

\usepackage{amsmath,amssymb,amsfonts,xfrac,cancel,bm,mathtools}

% Notação de ângulos para complexos
\usepackage{steinmetz}

% Notação de derivadas
%\usepackage[thinc]{esdiff}
%\usepackage{commath}

% Setinhas para notação de vetor
%\usepackage{esvect}

% Negrito para notação de vetor
\renewcommand{\vec}[1]{\bm{\mathrm{#1}}}

% Fonte para notação em rsfs
%\usepackage{mathrsfs}

% Para mapas de karnaugh
%\usepackage{karnaugh-map}


% setup do (circui)tikz
%\DeclareMathSymbol{\Omega}{\mathalpha}{letters}{"0A}% italics
%\DeclareMathSymbol{\varOmega}{\mathalpha}{operators}{"0A}% upright
%\providecommand*{\upOmega}{\varOmega}% for siunitx
\usepackage[american, nooldvoltagedirection, siunitx, cuteinductors]{circuitikz}

% Comando para mudar escala das tikzpictures \setmyunit{length}
\newcommand{\setmyunit}[1]{\tikzset{every picture/.style={x=#1, y=#1}}}

\usepackage{pgfplots}
\setmyunit{2cm}

% Unidades (inclusa no circuitikz)
%\usepackage{siunitx}

%\usepackage{tikz}
\usetikzlibrary{positioning, calc,
	patterns, arrows.meta, decorations.pathmorphing, decorations.markings,angles,quotes,intersections,decorations.pathreplacing,arrows}

\tikzset{ang/.style={draw, angle radius=15pt,angle eccentricity=1.3}}

\definecolor{copper}{RGB}{184, 115, 51}

% Planos xz e xy
\tikzset{xzplane/.style={canvas is xz plane at y=#1,very thin}}
\tikzset{xyplane/.style={canvas is xy plane at z=#1,very thin}}


% Macros para encontrar distancias no tikz
\pgfmathsetmacro{\xdeg}{15}
\pgfmathsetmacro{\xx}{cos(\xdeg)}
\pgfmathsetmacro{\xy}{sin(\xdeg)}

\pgfmathsetmacro{\ydeg}{170}
\pgfmathsetmacro{\yx}{cos(\ydeg)}
\pgfmathsetmacro{\yy}{sin(\ydeg)}

\pgfmathsetmacro{\zdeg}{90}
\pgfmathsetmacro{\zx}{cos(\zdeg)}
\pgfmathsetmacro{\zy}{sin(\zdeg)}

%\newlength{\ladderskip}
%\setlength{\ladderskip}{5\tikzcircuitssizeunit} % 5\tikzcircuitssizeunit = 35pt
%\newlength{\ladderrungsep}
%\setlength{\ladderrungsep}{.2\ladderskip}
%\def\ladderrungend#1{\pgftransformyshift{-#1\ladderskip-\ladderrungsep}}
%\newcommand{\powerrails}[1][0.7]{\draw let \p1=(laddertopright) in
%	(0,\y1+#1\ladderskip) -- (0,\ladderskip)
%	(\x1,\y1+#1\ladderskip) -- (\x1,\ladderskip);}

% Cores (inclusa no tikz)
%\usepackage{xcolor}


%%%%%%%%%%%%%%%%%%%%%%%%%%%%%%%%%%%%%%%%%%%
%%%%%%%%%%%%%%%%%%%%%%%%%%%%%%%%%%%%%%%%%%%
%%%%%%%%%%%%%%%%% COISAS DOIDAS %%%%%%%%%%%%%%%%%

% setup do siunitx
\sisetup{per-mode=symbol,output-decimal-marker={,},math-micro=\text{µ},text-micro=µ,exponent-product = \cdot,math-ohm=\Omega,
	text-ohm=\ensuremath{\Omega},detect-weight=true}

% MATLAB
%\usepackage{xspace}
%\newcommand{\MATLAB}{\textsc{Matlab}\xspace}

% Notação de negação lógica
%\newcommand{\notted}[1]{%
%	\overline{#1}%
%}

% Padrões de linhas que não funcionam no overleaf

%\pgfdeclarepatternformonly{south east lines}{\pgfqpoint{-0pt}{-0pt}}{\pgfqpoint{3pt}{3pt}}{\pgfqpoint{3pt}{3pt}}{
%	\pgfsetlinewidth{0.4pt}
%	\pgfpathmoveto{\pgfqpoint{0pt}{3pt}}
%	\pgfpathlineto{\pgfqpoint{3pt}{0pt}}
%	\pgfpathmoveto{\pgfqpoint{.2pt}{-.2pt}}
%	\pgfpathlineto{\pgfqpoint{-.2pt}{.2pt}}
%	\pgfpathmoveto{\pgfqpoint{3.2pt}{2.8pt}}
%	\pgfpathlineto{\pgfqpoint{2.8pt}{3.2pt}}
%	\pgfusepath{stroke}}
%
%\pgfdeclarepatternformonly{south west lines}{\pgfqpoint{-0pt}{-0pt}}{\pgfqpoint{3pt}{3pt}}{\pgfqpoint{3pt}{3pt}}{
%	\pgfsetlinewidth{0.4pt}
%	\pgfpathmoveto{\pgfqpoint{0pt}{0pt}}
%	\pgfpathlineto{\pgfqpoint{3pt}{3pt}}
%	\pgfpathmoveto{\pgfqpoint{2.8pt}{-.2pt}}
%	\pgfpathlineto{\pgfqpoint{3.2pt}{.2pt}}
%	\pgfpathmoveto{\pgfqpoint{-.2pt}{2.8pt}}
%	\pgfpathlineto{\pgfqpoint{.2pt}{3.2pt}}
%	\pgfusepath{stroke}}


% cor do fundo do block
%\definecolor{mWhite}{RGB}{239, 230, 231}

% Definições de funções trinométricas em pt-br
\DeclareMathOperator{\sen}{sen}
\DeclareMathOperator{\tg}{tg}
\DeclareMathOperator{\cotg}{cotg}
\DeclareMathOperator{\cossec}{cossec}
\DeclareMathOperator{\arctg}{arctg}

% QED branco
%\newcommand*{\QEDB}{\hfill\ensuremath{\square}}

% estilos para diagrama de blocos
%\newcommand{\deftkzbds}{
%	\tikzstyle{block} = [draw, fill=blue!20, rectangle, minimum height=3em, minimum width=6em]
%	\tikzstyle{sum} = [draw, fill=blue!20, circle, node distance=1cm]
%	\tikzstyle{input} = [coordinate]
%	\tikzstyle{output} = [coordinate]
%	\tikzstyle{pinstyle} = [pin edge={to-,thin,black}]
%}

% tipo de coluna matemática centralizada
%\newcolumntype{C}{>{$}c<{$}}

% para marcar ponto na tela e desenhar sobre
\newcommand{\tikzmark}[1]{\tikz[baseline,remember picture] \coordinate (#1) {};}

% unidades uteis para siuntix

\DeclareSIUnit{\voltef}{V_{ef}}
\DeclareSIUnit{\kvar}{kVAR}
\DeclareSIUnit{\kva}{kVA}
%\DeclareSIUnit{\lbf}{lbf}
%\DeclareSIUnit{\kgf}{kgf}
%\DeclareSIUnit{\kgfp}{\kgf \per \centi\meter\squared}
%\DeclareSIUnit{\mca}{mca}
%\DeclareSIUnit{\barp}{bar}
%\DeclareSIUnit{\pol}{pol}
%\DeclareSIUnit{\psip}{psi}
%\DeclareSIUnit{\atm}{atm}
%\DeclareSIUnit{\HP}{HP}
%\DeclareSIUnit{\psid}{\lbf\per\pol\squared}
%\DeclareSIUnit{}{}

% Usar para diminuir espaçamento antes/depois da equação (\useshortskip)
%\usepackage{nccmath}
%\usepackage{xpatch}
%\xpatchcmd{\NCC@ignorepar}{%
%	\abovedisplayskip\abovedisplayshortskip}
%{%
%	\abovedisplayskip\abovedisplayshortskip%
%	\belowdisplayskip\belowdisplayshortskip}
%{}{}

\usepackage{chngcntr}
\counterwithin*{equation}{section}
\newcounter{saveenumi}
\newcommand{\saveenumerate}{%
	\stepcounter{saveenumi}%
	\label{saveenumi-\thesaveenumi}}
\newcommand{\restoreenumerate}{%
	\setcounterref{enumi}{saveenumi-\thesaveenumi}}

% Símbolo para graus (usar \ang{degrees} do package siunitx)
%\newcommand{\degree}{\ensuremath{^\circ}}

% X de errado e certinho correspondente
\usepackage{pifont}
\newcommand{\cmark}{\ding{51}}%
\newcommand{\xmark}{\ding{55}}%

% \itemequation[label]{text before}{equation}

\makeatletter
\newcommand*{\itemequation}[3][]{%
  \item
  \begingroup
    \refstepcounter{equation}%
    \ifx\\#1\\%
    \else
      \label{#1}%
    \fi
    \sbox0{#2}%
    \sbox2{$\displaystyle#3\m@th$}%
    \sbox4{ \@eqnnum}%
    \dimen@=.5\dimexpr\linewidth-\wd2\relax
    % Warning for overlapping
    \let\CenterInSpace=N%
    \ifcase
    \ifdim\wd0>\dimen@
          \z@
        \else
          \ifdim\wd4>\dimen@
            \z@
          \else
            \@ne
          \fi
        \fi
      \let\CenterInSpace=Y%
    \fi
    \ifdim\dimexpr\wd0+\wd2+\wd4\relax>\linewidth
      \@latex@warning{Equation is too large}%
    \fi
    \noindent
    \rlap{\copy0}%
    \ifx\CenterInSpace Y%
      \rlap{\hbox to \linewidth{\kern\wd0\hss\copy2\hss\kern\wd4}}%
    \else
      \rlap{\hbox to \linewidth{\hfill\copy2\hfill}}%
    \fi
    \hbox to \linewidth{\hfill\copy4}%
    \hspace{0pt}% allow linebreak
  \endgroup
  \ignorespaces
}
\makeatother


%%%%%%%%%%%%%%%%%%%%%%%%%%%%%%%%%%%%%%%%%%%
%%%%%%%%%%%%%%%%%%%%%%%%%%%%%%%%%%%%%%%%%%%
%%%%%%%%%%%%%%%%%% RODAPÉ %%%%%%%%%%%%%%%%%

\setbeamercolor{footline}{fg=white}
\setbeamertemplate{footline}
{\begin{tikzpicture}
    \node [inner sep=0pt, anchor=east] (0,0) {\includegraphics[width=\paperwidth,height=1cm]{Figuras/Capa/macaefooter.png}};
    \node [inner sep=0pt, anchor=east] at (-2ex,-3ex) {\insertframenumber{} / \inserttotalframenumber};
\end{tikzpicture}}


%%%%%%%%%%%%%%%%%%%%%%%%%%%%%%%%%%%%%%%%%%%
%%%%%%%%%%%%%%%%%%%%%%%%%%%%%%%%%%%%%%%%%%%
%%%%%%%%%%% INFORMAÇÕES DO CURSO %%%%%%%%%%

\title[Eletrotécnica II] {Eletrotécnica II}

\subtitle {Notas de aula}

\author{Prof. Yago Pessanha Corrêa}

\institute[MSP/IFF] 
{
Laboratório de Mecatrônica e Processamento de Sinais (MSP) \\
Instituto Federal de Educação, Ciência e Tecnologia Fluminense (IFFluminense) \\
Cursos Técnicos em Automação, Eletrônica e Eletromecânica \\
\vspace*{.1cm} {\tt \textbf{yago.correa@iff.edu.br}}\\
}

%\tddate


%%%%%%%%%%%%%%%%%%%%%%%%%%%%%%%%%%%%%%%%%%%
%%%%%%%%%%%%%%%%%%%%%%%%%%%%%%%%%%%%%%%%%%%
%%%%%%%%%%%%%%%%% SUMÁRIO %%%%%%%%%%%%%%%%%

\AtBeginSection[]
{
  \begin{frame}<beamer>{Sumário}
    \tableofcontents[currentsection]
  \end{frame}
}