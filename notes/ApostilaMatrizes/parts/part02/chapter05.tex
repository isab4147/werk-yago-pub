\chapter{Regra de Chió \texorpdfstring{\commentary{(abaixamento de grau)}}{(abaixamento de grau)}}

\section{Introdução}
A \textit{regra de Chió} tem como base os teoremas de \textit{Laplace} e \textit{Jacobi}. Basicamente, quando uma matriz tem seu elemento $a_{11}=1$, usando a \textbf{primeira coluna} como referência e aplicando o teorema de \textit{Jacobi} \textbf{sucessivamente} poderemos usar \textit{Laplace} facilmente.

\section{Aplicação (com variáveis)}

$$
A=\begin{bmatrix*}[r]
\color{cyan}1 & \color{green}-3 & \color{green}12 & \color{green}5\\
\color{red}5 & 9 & -1 & 2\\
\color{red}-2 & 4 & 4 & -1\\
\color{red}7 & 1 & 2 & 0
\end{bmatrix*}
$$

Repare que o elemento $a_{11}$ (em \textcolor{cyan}{azul claro}) tem valor $1$, como desejamos. Usaremos a \textbf{primeira coluna} (em \textcolor{red}{vermelho}) como \textbf{referência} para aplicar o teorema de \textit{Jacobi}, com o objetivo de \textbf{zerar} a \textbf{primeira linha} (em \textcolor{green}{verde}). Fazendo o passo a passo para o teorema de \textit{Jacobi} podemos notar que por conta do elemento $a_{11}=1$ as constantes vão ser o \textbf{oposto} dos números da primeira linha. Por exemplo:
$$
-3+1\cdot k_1=0\Rightarrow k_1=3
$$

Que é o oposto de $-3$.
Aplicando \textit{Jacobi}, temos:

\vskip7em

\begin{align*}
    A&=\begin{bmatrix*}[r]
        \tikzmark{m3c1}1 & -3+3\cdot\tikzmark{m3c2}1 & 12+(-12)\cdot\tikzmark{m3c3}1 & 5+(-5)\cdot\tikzmark{m3c4}1\\
        5 & 9+3\cdot5 & -1+(-12)\cdot5 & 2+(-5)\cdot5\\
        -2 & 4+3\cdot(-2) & 4+(-12)\cdot(-2) & -1+(-5)\cdot(-2)\\
        7 & 1+3\cdot7 & 2+(-12)\cdot7 & 0+(-5)\cdot7
    \end{bmatrix*}=\\
     &=\begin{bmatrix*}[r]
        1 & -3+3 & 12-12 & 5-5\\
        5 & 9+15 & -1-60 & 2-25\\
        -2 & 4-6 & 4+24 & -1+10\\
        7 & 1+21 & 2-84 & 0-35
    \end{bmatrix*}= \\
     &=\begin{bmatrix*}[r]
        \color{cyan}1 & \color{cyan}0 & \color{cyan}0 & \color{cyan}0\\
        5 & 24 & -61 & -23\\
        -2 & -2 & 28 & 9\\
        7 & 22 & -82 & -35
    \end{bmatrix*}
\end{align*}

\begin{tikzoverlay}
    \curvedarrow[orange!80, ultra thick] (m3c1:0.4 m3c4:0.4);
    
    \curvedarrow[greenAr!80,very thick] (m3c1:0.4 m3c3:0.4);
    
    \curvedarrow[redAr!80, thick] (m3c1:0.4 m3c2:0.4);
\end{tikzoverlay}

Agora podemos aplicar o teorema de \textit{Laplace} usando a \textbf{primeira linha} da matriz $A$ (em \textcolor{cyan}{azul}):

$$
\det A=1\cdot A_{11} - 0\cdot_{12} - ~0\cdot A_{13} - 0\cdot A_{14} \Rightarrow \det A=A_{11}
$$

Podemos ver que utilizando o método descrito podemos reduzir o determinante de uma matriz à apenas \textbf{um} de seus cofatores.

\section{Uma visão alternativa}

Outra forma de ver esse método é memorizando a seguinte receita:

$$
\text{Dada a matriz }A=\begin{bmatrix*}[r]
1 & -3 & 12 & 5 \\
5 & 9 & -1 & 2 \\
-2 & 4 & 4 & -1 \\
7 & 1 & 2 & 0 
\end{bmatrix*}
$$

\begin{enumerate}
    \item Verificamos se o primeiro elemento da matriz ($a_{11}$) é igual à $1$. \checkmark
    \item Prosseguimos com o seguinte método:
\end{enumerate}

Pegamos os elementos da \textbf{primeira linha} e da \textbf{primeira coluna}, multiplicamos eles \textbf{ordenadamente} um pelo outro e \textbf{subtraímos} do elemento que se encontra na junção deles.

No exemplo abaixo nós vamos começar multiplicando $-3$ (2\textsuperscript{o} elemento da 1\textsuperscript{a} linha) por $5$ (2\textsuperscript{o} elemento da 1\textsuperscript{a} coluna) e, então, vamos subtrair de $9$ (que é o elemento que encontramos seguindo as linhas \textcolor{orange}{laranjas}).

\begin{center}
\hspace{-43pt}
    \begin{tikzpicture}[node distance=0pt, every left delimiter/.style={xshift=1em},
        every right delimiter/.style={xshift=-1em}, every node/.style={minimum width=20pt, minimum height=14pt}, ampersand replacement=\&]
        
        \matrix(M)[matstyle] {
            1 \& -3 \& 12 \& 5 \\
            5 \& 9 \& -1 \& 2 \\
            -2 \& 4 \& 4 \& -1 \\
            7 \& 1 \& 2 \& 0 \\
        };
        
        \node[left=of M] {$A=$};
        \node[right=of M]{$=$};
        
        \begin{scope}[on background layer, every node/.style={draw, draw=white, circle, fill=white,minimum width=8pt, minimum height=10pt}]
            \draw[thick,orange] (M-1-2.north) -- (M-2-2.south);
            \draw[thick,orange] (M-2-1.west) -- (M-2-2.east);
            
            \node[] at (M-1-2){};
            \node[] at (M-2-1){};
            \node[draw=orange] at (M-2-2){};
        \end{scope}
    \end{tikzpicture}
\end{center}

\vspace{-30pt}

\begin{align*}
    &=\begin{bmatrix}
        1 & -3 & 12 & 5 \\
        5 & 9-(-3)\cdot5 & -1 & 2 \\
        -2 & 4 & 4 & -1 \\
        7 & 1 & 2 & 0
    \end{bmatrix}=\\
    &=\begin{bmatrix*}[r]
        1 & -3 & 12 & 5 \\
        5 & 24 & -1 & 2 \\
        -2 & 4 & 4 & -1 \\
        7 & 1 & 2 & 0
    \end{bmatrix*}
\end{align*}

Agora realizamos o mesmo procedimento para os outros elementos da matriz:

\begin{center}
\hspace{-155pt}
    \begin{tikzpicture}[node distance=0pt, every left delimiter/.style={xshift=1em},
        every right delimiter/.style={xshift=-1em}, every node/.style={minimum width=20pt, minimum height=14pt}, ampersand replacement=\&]
        
        \matrix(M)[matstyle]{
            1 \& -3 \& 12 \& 5 \\
            5 \& 24 \& -1 \& 2 \\
            -2 \& 4 \& 4 \& -1 \\
            7 \& 1 \& 2 \& 0 \\
        };
        
        \node[left=of M] {$A=$};
        
        \begin{scope}[on background layer, every node/.style={draw, draw=white, fill=white,minimum width=14pt, minimum height=10pt}]
            \foreach \i in {2,3,4}{
                \draw[thick, orange] (M-1-\i.north) -- (M-4-\i.south) {};
                \draw[thick, orange] (M-\i -1.west) -- (M-\i -4.east) {};
            }
            
            \foreach \i in {1,2,3,4}{
                \foreach \j in {1,2,3,4}{
                \ifthenelse{\(\i = \j \AND \i = 2\) \OR \(\i = 1 \OR \j = 1\)}{\node[] at (M-\i -\j){};}{\node[draw=orange, circle] at (M-\i -\j) {};}
                }
            }
        \end{scope}
    \end{tikzpicture}
\end{center}

\vspace{-30pt}

\begin{align*}
    &=\begin{bmatrix}
        1 & -3 & 12 & 5 \\
        5 & 24 & -1-12\cdot5 & 2-5\cdot5 \\
        -2 & 4-(-3)\cdot(-2) & 4-12\cdot(-2) & -1-5\cdot(-2) \\
        7 & 1-(-3)\cdot7 & 2-12\cdot7 & 0-5\cdot7
    \end{bmatrix}=\\
    &=\begin{bmatrix*}[r]
        1 & -3 & 12 & 5\\
        5 & 24 & -61 & -23\\
        -2 & -2 & 28 & 9\\
        7 & 22 & -82 & -35
    \end{bmatrix*}
\end{align*}

A regra de Chió nos diz que, após aplicar a receita descrita, o cofator $A_{11}$ é \textbf{igual} ao determinante da matriz $A$.

$$
A_{11}= \begin{vmatrix*}[r]
        24 & -61 & -23\\
        -2 & 28 & 9\\
        22 & -82 & -35
\end{vmatrix*} = \det A
$$
