\chapter{Teorema de Jacobi}

\section{Introdução}
O teorema diz que, se tomarmos uma matriz quadrada qualquer $A$ podemos \textbf{somar} uma coluna $c$ dessa matriz à uma outra coluna $c'$ qualquer (também da matriz) multiplicada por uma \textbf{constante} $k$ de escolha, e esse processo \textbf{não altera} o valor do determinante dessa matriz.

\Example

\begin{gather*}
    \text{Dada a matriz }A=\begin{bmatrix*}[r]
        a_{11} & a_{12} \\
        a_{21} & a_{22}
    \end{bmatrix*} \text{, podemos fazer} \\
    \begin{bmatrix*}[r]
        a_{11} & a_{12} + k \cdot a_{11} \\
        a_{21} & a_{22} + k \cdot a_{22}
    \end{bmatrix*} \text{, onde } \det B = \det A
\end{gather*}

\section{Aplicação}
A utilidade desse teorema não é clara à primeira vista, mas vamos discorrer uma técnica que esclarece seu uso:

\begin{enumerate}
    \item Primeiro devemos tomar uma \textbf{coluna de referência}, vamos usar a primeira coluna (em \textcolor{red}{vermelho}).
    $$
    A=\begin{bmatrix*}[r]
    \color{red}4 & -2 & 8 \\
    \color{red}3 & 7 & -2 \\
    \color{red}5 & 1 & -1
    \end{bmatrix*}
    $$
    \item Agora temos que escolher uma linha. Para o exemplo vamos usar a primeira (em \textcolor{green}{verde}).
    $$
    \begin{bmatrix*}[r]
    \color{green}-2 & \color{green}8 \\
    7 & -2 \\
    1 & -1
    \end{bmatrix*}
    $$
    \item Vamos aplicar o teorema de \textit{Jacobi} de modo a \textbf{zerar} essa linha. Para facilitar podemos usar uma equação. Para zerar o $-2$ podemos fazer:
    $$
    -2 + 4 \cdot k_1 = 0 \Rightarrow k_1 = \sfrac{1}{2}
    $$
    \item Repare o que ocorre:
    \begin{enumerate}
        \item Pegamos a coluna de referência e vamos usá-la com \textit{Jacobi} nas duas outras.
        \item Escolhemos alguma linha e vamos usar o teorema com o objetivo de zerá-la.
    \end{enumerate}
    \item Continuando. Para zerar o $8$ segue:
    $$
    8+4 \cdot k_2 = 0 \Rightarrow k_2 = -2
    $$
\end{enumerate}

\paragraph{Nota:}
Repare que para cada coluna podemos usar uma constante \textbf{diferente} ($k_1$ e $k_2$ nesse caso).

\smallskip

Aplicando o teorema na segunda coluna (em \textcolor{teal}{azul}) ficamos com:

\vskip2.5em

\begin{minipage}{\textwidth}
$$
A=\begin{bmatrix*}[r]
\tikzmark{m1c1}4 & \textcolor{teal}{-2}+\sfrac{1}{2}\cdot\tikzmark{m1c2}4 & 8 \\
3 & \textcolor{teal}{7}+\sfrac{1}{2}\cdot3 & -2 \\
5 & \textcolor{teal}{1}+\sfrac{1}{2}\cdot5 & -1
\end{bmatrix*}
$$

\begin{tikzoverlay}
    \curvedarrow[redAr!80, thick] (m1c1:0.5 m1c2:0.5);
\end{tikzoverlay}
\end{minipage}
    
A coluna de referência é \textbf{multiplicada} pela \textbf{constante} e \textbf{somada} à segunda coluna, realizando a operação ficamos com:

$$
A=\begin{bmatrix*}[r]
4 & -2+2 & 8 \\
3 & 7+\sfrac{3}{2} & -2\\
5 & 1+\sfrac{5}{2} & -1
\end{bmatrix*} \Rightarrow A=\begin{bmatrix*}[r]
4 & 0 & 8\\
3 & \sfrac{17}{2} & -2\\
5 & \sfrac{7}{2} & -1
\end{bmatrix*}
$$

Agora vamos aplicar o teorema à terceira coluna usando a outra constante ($-2$):

\vskip2.5em

\begin{gather*}
    A=\begin{bmatrix*}[r]
    \tikzmark{m2c1}4 & 0 & 8+(-\tikzmark{m2c3}2) \cdot 4\\
    3 & \sfrac{17}{2} & -2+(-2)\cdot3\\
    5 & \sfrac{7}{2} & -1+(-2)\cdot5
    \end{bmatrix*}\Rightarrow\\
    A=\begin{bmatrix*}[r]
    4 & 0 & 8-8\\
    3 & \sfrac{17}{2} & -2-6\\
    5 & \sfrac{7}{2} & -1-10
    \end{bmatrix*}\Rightarrow A=\begin{bmatrix*}[r]
    4 & 0 & 0\\
    3 & \sfrac{17}{2} & -8\\
    5 & \sfrac{7}{2} & -11
    \end{bmatrix*}
\end{gather*}

\begin{tikzoverlay}
    \curvedarrow[redAr!80,thick] (m2c1:0.5 m2c3:0.5);
\end{tikzoverlay}

Agora podemos ver como esse teorema pode ser útil, se quisermos usar o \textit{teorema de Laplace} para encontrar o determinante da matriz $A$ faremos pouquíssimas operações pois temos uma fileira quase toda zerada.