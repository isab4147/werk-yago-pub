\chapter{Cofator}
\label{chap:cof}

\section{Definição}
Dada uma matriz quadrada $A$, o cofator de $A_{ij}$ é definido como o determinante (com sinal característico) da matriz $A$ obtido \textbf{suprimindo-se} a linha $i$ e a coluna $j$ da matriz original $A$.

Basicamente, um cofator é definido como o \textbf{determinante} de um \textit{recorte} da matriz original, com um sinal específico.

\section{Na prática}
Esse recorte se dá da seguinte forma:

$$
\text{Se }A=\begin{bmatrix*}[r]
    a_{11} & a_{12} & a_{13} \\
    a_{21} & a_{22} & a_{23} \\
    a_{31} & a_{32} & a_{33}
\end{bmatrix*} \text{, fazendo o recorte do cofator $A_{11}$ ficamos com}
$$
\vspace{-10pt}
\begin{tikzmatrix}
    \tikzset{faded/.style={color=red!50}}
    
    \matrix(M)[matstyle]{
        |[faded]|a_{11} \& |[faded]|a_{12} \& |[faded]|a_{13} \\
        |[faded]|a_{21} \& a_{22} \& a_{23} \\
        |[faded]|a_{31} \& a_{32} \& a_{33} \\
    };
    \node[left=of M] {$A_{11}=\det$};
    \node[right=of M] {$
\Rightarrow A_{11}=\begin{vmatrix*}[r]
a_{22} & a_{23} \\
a_{32} & a_{33}
\end{vmatrix*}
$};
    
    \draw[thick] (M-1-1.west) -- (M-1-3.east)
    (M-1-1.north) -- (M-3-1.south);
        
\end{tikzmatrix}

Agora que o recorte está feito, basta tirar o determinante da matriz que sobrou e colocar um sinal nele para conseguir o cofator.

O sinal será explicado em detalhe no próximo capítulo, mas consiste em multiplicar o determinante por $(-1)^{i+j}$ onde $i$ e $j$ são o número da linha e da coluna que suprimimos.