\chapter{Introdução}

\section{Um pouco de contexto}

Historicamente, os determinantes eram usados muito anteriormente em relação às matrizes, e eram considerados uma propriedade dos sistemas lineares (abordados na \autoref{p:3}). Os determinantes "determinam"~se um sistema linear qualquer tem ou não solução única.

O japonês \textit{Seki Takakazu} (\begin{CJK}{UTF8}{min}関 孝和\end{CJK}) leva o crédito da descoberta da resultante e do determinante (1683-1710) e, na Europa, depois que \textit{Leibniz} introduziu o estudo dos determinantes, \textit{Gabriel Cramer} expandiu a teoria relacionando-a à conjuntos de equações.

\textit{Vandermonde} foi o primeiro a tratar os determinantes como funções independentes de sistemas lineares e \textit{Laplace} contribuiu com o método geral para escrever um determinante através de seus cofatores. \textit{Lagrange} e \textit{Gauss}, utilizando os determinantes na teoria dos números, fizeram avanços importantes na teoria, e \textit{Gauss} foi o primeiro a usar o nome \textit{determinante}, embora não no sentido atual.

Foi \textit{Cauchy} quem primeiro introduziu o termo \textit{determinante} no sentido aceito atualmente, num trabalho publicado em 1812, anteriormente o termo \textit{resultante} havia sido utilizado por \textit{Laplace}.

Quem mais contribuiu para a teoria de determinantes foi \textit{Carl G. J. Jacobi} (1804-1851). A simplicidade atual da apresentação dessa teoria se deve a ele.


\section{Definição e resolução para \texorpdfstring{$n \leqslant 3$}{n <= 3}}

Podemos dizer que o \textit{determinante} de uma matriz quadrada é o seu \textbf{valor numérico}.

Denotamos o \textit{determinante} de uma matriz quadrada qualquer $A$ por

\begin{center}
    \begin{romanlistinline}
        \item $\det A$ , ou \item $|A|$
    \end{romanlistinline}
\end{center}

\Example

$$
\text{Seja a matriz }A=\begin{bmatrix*}[r]
1 & 2\\ 3 & 4
\end{bmatrix*}\text{, denotamos seu determinante por }\det A=\det\begin{bmatrix*}[r]
1 & 2\\ 3 & 4
\end{bmatrix*}=\begin{vmatrix}
1 & 2\\ 3 & 4
\end{vmatrix}
$$

\subsection{Caso \texorpdfstring{$n=1$}{n=1}}

Quando temos uma matriz com um único elemento seu determinante será esse elemento. 

\Example

$$
\text{Dada a matriz }A=[-2]\text{, }\det A=-2
$$

\subsection{Caso \texorpdfstring{$n=2$}{n=2}}

Quando temos uma matriz quadrada de ordem 2 basta tomar o produto dos elementos da \textbf{diagonal principal} e \textbf{subtraí-lo} pelo produto dos elementos da \textbf{diagonal secundária}.

\Example

$$
\text{Dada a matriz }A=\begin{bmatrix*}[r]
    2 & 3\\ -3 & 1
\end{bmatrix*}\text{, }\det A=2\cdot 1 -3\cdot (-3) \Rightarrow \det A=2+9=11 
$$

\subsection{Caso \texorpdfstring{$n=3$}{n=3}}

Com uma matriz quadrada de grau 3 o processo é bastante similar ao do grau $2$ (descrito logo acima). 

Primeiro devemos \textbf{repetir} a primeira e a segunda colunas da matriz à sua direita (como mostrado em \textcolor{red}{vermelho}):

\begin{gather*}
    A=\begin{bmatrix*}[r]
        a_{11} & a_{12} & a_{13} \\
        a_{21} & a_{22} & a_{23} \\
        a_{31} & a_{32} & a_{33}
    \end{bmatrix*} \Rightarrow \\
    \det A= \begin{vmatrix*}[r]
        a_{11} & a_{12} & a_{13} \\
        a_{21} & a_{22} & a_{23} \\
        a_{31} & a_{32} & a_{33}
    \end{vmatrix*} \color{red} \begin{matrix}
        a_{11} & a_{12} \\
        a_{21} & a_{22} \\
        a_{31} & a_{32}
    \end{matrix}
\end{gather*}

Em seguida devemos calcular o \textbf{produto} das diagonais formadas:

\begin{center}
        \begin{tikzpicture}[node distance=-6pt, every left delimiter/.style={xshift=.50em},
        every right delimiter/.style={xshift=-.50em}, ampersand replacement=\&]
        
        \matrix(M1)[matstyle]{
            a_{11} \& a_{12} \& a_{13} \\
            a_{21} \& a_{22} \& a_{23} \\
            a_{31} \& a_{32} \& a_{33} \\
        };
        \matrix(M2)[matrix of math nodes, right=of M1, nodes={color=red!80}]{
            a_{11} \& a_{12} \\
            a_{21} \& a_{22} \\
            a_{31} \& a_{32} \\
        };
    
        \begin{scope}[on background layer, every node/.style={fill=white, circle, draw, draw=white, minimum width=14pt}]
            \draw[->, redAr!80] (M1-1-1.north west) -- (M1-3-3.south east);
            \draw[->, redAr!80] (M1-1-2.north west) -- (M2-3-1.south east);
            \draw[->, redAr!80] (M1-1-3.north west) -- (M2-3-2.south east);
            \draw[->, greenAr!80] (M2-1-2.north east) -- (M1-3-3.south west);
            \draw[->, greenAr!80] (M2-1-1.north east) -- (M1-3-2.south west);
            \draw[->, greenAr!80] (M1-1-3.north east) -- (M1-3-1.south west);
            
            \foreach \i in {1,2,3}{
                \foreach \j in {1,2,3}
                    \node[] at (M1-\i -\j) {  };
            }
            \foreach \i in {1,2,3}{
                \foreach \j in {1,2}
                    \node [] at (M2-\i -\j) {  };
            }
        \end{scope}
    \end{tikzpicture}
\end{center}

Ficamos com:

\begin{gather*}
    \det A=\\
    a_{11} \cdot a_{22} \cdot a_{33} + 
    a_{12} \cdot a_{23} \cdot a_{31} +
    a_{13} \cdot a_{21} \cdot a_{32} \\
    - a_{13} \cdot a_{22} \cdot a_{31}
    - a_{22} \cdot a_{23} \cdot a_{32}
    - a_{31} \cdot a_{21} \cdot a_{33}
\end{gather*}

Note que as diagonais na direção da principal devem ser \textbf{somadas} e as contrárias devem ser \textbf{subtraídas}.

\paragraph{Nota:}
Esse método é chamado de \textit{Regra de Sarrus}.