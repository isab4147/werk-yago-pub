\chapter{Classificação dos sistemas lineares}

\section{Introdução}

A partir do teorema de \textit{Cramer} e da técnica do \textit{escalonamento} podemos \textbf{classificar} os sistemas lineares em \textbf{três} tipos:

\begin{enumerate}[label=(\Roman*), align=Center]
    \item S.P.D. $\Rightarrow$ Sistema Possível Determinado
    \item S.P.I. $\Rightarrow$ Sistema Possível Indeterminado
    \item S.I. $\Rightarrow$ Sistema Impossível

\end{enumerate}

O primeiro já foi abordado anteriormente (veja o \autoref{chap:cram}). Todo sistema onde o determinante dos coeficientes (chamado $D$) for diferente de $0$ será S.P.D.

O segundo e terceiro tipos ocorrerão quando $D=0$. No segundo teremos \textbf{infinitas soluções possíveis} e no terceiro, \textbf{nenhuma} (daí o nome \textit{impossível}).

\section{Classificação}
No exemplo do \autoref{chap:esc} usamos um sistema onde $D\neq 0$ e, então, caímos no caso S.P.D., porém se fosse o caso $D=0$ a classificação em S.P.I. ou S.I. se faria necessária.

\begin{enumerate}[label=(\Roman*), align=Center]
    \item O caso \textbf{S.P.I.} ocorre quando:
    \begin{enumerate}
        \item após o escalonamento, ocorrer uma expressão do tipo:
        $$
        0a+0b+0c+\dots+0z=0 \text{ onde ($a$, $b$, $c$, $\dots$, $z$) representam incógnitas.}
        $$
        
        Repare que nesse caso temos \textbf{incógnitas} do sistema com coeficiente \textbf{igual à $\mathbf{0}$ igualadas à $\mathbf{0}$}.
        
        \item os determinantes $D_i$ para as incógnitas do sistema forem iguais à zero.
    \end{enumerate}
    
    \item O caso \textbf{S.I.} ocorre quando:
    \begin{enumerate}
        \item após o escalonamento ocorrer uma expressão do tipo:
        $$
        0a+0b+0c+\dots+0z \neq 0 \text{ onde ($a$, $b$, $c$, $\dots$, $z$) representam incógnitas.}
        $$
        
        Repare que nesse caso temos incógnitas do sistema com coeficientes \textbf{iguais à $\mathbf{0}$}, porém igualadas à um valor \textbf{diferente de $\mathbf{0}$}, o que é \textbf{impossível}, afinal $\mathbf{0x=0}$ para \textbf{qualquer valor de $\mathbf{x}$}, e o \textbf{contrário} disso é \textbf{absurdo!}
        
        \item os determinantes $D_i$ para as incógnitas do sistema forem diferentes de zero.
        
    \end{enumerate}
    
    \paragraph{Nota:}
        Os dois casos para S.I. e S.P.I. são equivalentes, portanto podemos testá-los tanto por determinantes quanto por escalonamento.
\end{enumerate}

\section{Resolvendo sistemas S.P.I. e S.I. com variáveis \texorpdfstring{$a$}{a} e \texorpdfstring{$b$}{b}}

Podemos nos deparar com sistemas lineares onde algum ou alguns de seus coeficientes são representados por uma variável $a$, e talvez algum de seus termos independentes seja representado por outra variável, $b$.

\paragraph{Nota:}
as variáveis não necessariamente tem que ser representadas por $a$ e $b$, podendo ser quaisquer letras de escolha (frequentemente são $m$ e $k$).

\medskip

Para resolver esse tipo de sistema devemos dividi-lo em \textbf{casos}. O enunciado geralmente pede para que \textit{discutamos} o sistema.
$$\text{Seja o sistema }S_1\begin{cases}
2x+y=1\\
2x-ay=-3
\end{cases}$$

Podemos começar resolvendo-o pelo teorema de \textit{Cramer}:
$$
D=\begin{vmatrix}
2 & 1\\
2 & -a
\end{vmatrix}\Rightarrow D=2\cdot (-a) -1\cdot 2\Rightarrow D=-2\cdot a-2
$$

Para que o sistema seja S.P.D. devemos ter $D\neq 0$, então, para que \textbf{não} seja S.P.D. devemos ter $D=0$, ou, substituindo:
$$
-2\cdot a-2=0\Rightarrow a=-1
$$

Agora que achamos o valor de $a$ para que o sistema não seja S.P.D., basta descobrirmos se ele será S.P.I. ou S.I.

Vamos substituir $a$ por $-1$ no sistema.
$$
S_1\begin{cases}
2x+y=1\\
2x-(-1)y=-3
\end{cases}\Rightarrow\begin{cases}
2x+y=1\\
2x+y=-3
\end{cases}
$$

Usando a primeira expressão como referência, por escalonamento temos:
$$S_a\times k+S_b=0$$

Para zerar o coeficiente de $x$ na segunda expressão temos:
$$2\cdot k+2=0\Rightarrow k=-1$$

Multiplicando a primeira expressão por $-1$ e somando à segunda, temos:
\begin{gather*}
    (2x+y=1) \times (-1)+S_b \Rightarrow\\ -2x-y=-1 + S_b\Rightarrow\\
    \begin{aligned}
    -2x-y&=-1\\
    \mathbf{+}~~2x+y&=-3\\\midrule
    0x+0y&=-4
    \end{aligned}
\end{gather*}

Claramente, para $D=0$ o sistema é \textbf{impossível}.

\medskip

A discussão desse sistema, então, será:

\begin{enumerate}[label={Caso (\Roman*)}, align=Center]
    \item quando $a\neq-1$ teremos $D\neq0$ e o sistema será S.P.D.
    \item quando $a=-1$ o sistema será S.I.
\end{enumerate}

Outra possibilidade de sistema é:
$$
S_2\begin{cases}
x-y=2\\
2x+ay=b
\end{cases}
$$

Onde temos as variáveis $a$ e $b$.

Novamente, dividiremos em casos, começando pela aplicação do teorema de \textit{Cramer}.
$$
D=\begin{vmatrix}
1 & -1\\
2 & a
\end{vmatrix}\Rightarrow D=1\cdot a-(-1)\cdot 2\Rightarrow D=a+2
$$

Para que o sistema seja S.P.D. segue que $D\neq 0$, substituindo, temos:
$$
a+2\neq 0\Rightarrow a\neq -2
$$

Para que o sistema não seja S.P.D. temos que $D=0$, ou, substituindo, $a=-2$.
Substituindo $a$ no sistema ficamos com
$$
S_2 \begin{cases}
x-y=2\\
2x+(-2)y=b
\end{cases} \Rightarrow S_2 \begin{cases}
x-y=2\\
2x-2y=b
\end{cases}
$$

Usando a primeira expressão como referência, por escalonamento, temos:
$$
S_a \times k + S_b=0
$$

Para zerar o coeficiente de $x$ na segunda expressão temos
$$1\cdot k + 2= 0\Rightarrow k=-2$$

Multiplicando a primeira expressão por $-2$ e somando à segunda, temos
\begin{gather*}
    (x-y=2) \times (-2) + S_b \Rightarrow\\
    -2x+2y=-4 + S_b\Rightarrow\\
    \begin{aligned}
    -2x+2y&=-4\\
    \mathbf{+}~~2x-2y&=b\\\midrule
    0x+0y&=b-4
    \end{aligned}
\end{gather*}

Agora temos dois casos, um onde o termo independente é igual a zero e, portanto, o sistema será S.P.I. e outro onde o termo independente é diferente de zero e o sistema será S.I.

Basta, por fim, discuti-los junto ao caso $D\neq0$.

\begin{enumerate}[label={Caso (\Roman*)}, align=Center]
    \item quando $a\neq -2$ teremos $D\neq0$ e o sistema será S.P.D.
    \item quando $a=-2$ e $b-4=0$ (portanto, $b=4$) segue que o sistema será S.P.I
    \item quando $a=-2$ e $b-4\neq 0$ (logo $b\neq4$) segue que o sistema será S.I
\end{enumerate}