\chapter{Soluções de um sistema linear}

\section{O que são?}

As \textbf{soluções} para um dado sistema linear devem ser escritas organizadamente em uma \textbf{ênupla} (coordenada com $n$ posições, a palavra vem de \textit{n-upla}) \textbf{ordenada} (pois tem uma ordem específica) \textbf{de reais}. Uma solução de um sistema linear consiste em uma série de valores que deverá resolver o sistema de igualdades proposto.

\Example

No sistema

$$
S_1\begin{cases}
2x+y=1\\
2x-y=3
\end{cases}
$$

temos como solução a ênupla $(1, -1)$, pois se substituirmos $x$ e $y$ por esses números respectivamente, teremos as igualdades

$$
S_1\begin{cases}
2\cdot 1 + (-1)=1\\
2\cdot 1-(-1)= 3
\end{cases}
$$

Repare que na ênupla o primeiro número substitui a incógnita $x$ e o segundo, $y$. A posição de cada valor na ênupla nos dirá qual incógnita cada valor deverá substituir (daí vem o termo \textit{ordenada}).