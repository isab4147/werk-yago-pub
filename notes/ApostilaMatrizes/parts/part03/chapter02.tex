\chapter{Forma matricial}

Todo sistema linear pode ser escrito na \textit{forma matricial}, realizando alguns passos...

\section{Passo a passo}

\begin{enumerate}
    \item É necessário avaliar se nosso sistema é ou não linear.
    
    \item É necessário organizar o sistema de forma adequada.
    
    \Example
    
    \begin{gather*}
        \text{Dado o sistema }S_1\begin{cases}
        \textcolor{red}{2x}+\textcolor{green}{3y}-4=0\\
        \textcolor{green}{-~y}+\textcolor{red}{x}=2
        \end{cases} \text{, podemos reescrevê-lo como:}\\
        S_1\begin{cases}
        \textcolor{red}{2x}+\textcolor{green}{3y}=4\\
        \textcolor{red}{x}+(\textcolor{green}{-~y})=2
        \end{cases}
    \end{gather*}
    
    \paragraph{Note que:}
    
    \begin{enumerate}
        \item \textbf{Separamos} as incógnitas (junto com seus coeficientes) do \textbf{termo independente}.
        \item Colocamos incógnitas \textbf{correspondentes} na mesma \textbf{coluna} (\textcolor{green}{verde} em baixo de \textcolor{green}{verde} e \textcolor{red}{vermelho} em baixo de \textcolor{red}{vermelho}).
    \end{enumerate}
    
    \item Vamos \textbf{identificar} os coeficientes e escrevê-los:
    
    \Example
    
    \begin{gather*}
        \text{Dado o sistema }S_2\begin{cases}
        x-3y+2z=2\\
        2y-z=1
        \end{cases} \text{, podemos reescrevê-lo como:}\\
        S_2\begin{cases}
        1x-3y+2z=2\\
        0x+2y-z=1
        \end{cases}
    \end{gather*}
    
    O passo a passo pode ser descrito como:
    
    \begin{enumerate}
        \item \textbf{Identificamos} quais são as incógnitas (no caso, $x$, $y$ e $z$).
        
        \item \textbf{Reescrevemos a equação} (similar ao passo anterior), mas agora colocando os coeficientes \textbf{explicitamente} (sem omitir $0$’s ou $1$’s).
        
    \end{enumerate}
    
    \item Vamos escrever os \textbf{coeficientes} em \textbf{colunas} de uma matriz (cada sistema será \textbf{uma linha}).
    
    \Example
    
    \begin{gather*}
        S_3\begin{cases}
        2x-1y+1z=0\\
        0x+1y-1z=3\\
        5x+2y-3z=-2
        \end{cases}\\
        \begin{bmatrix*}[r]
        2 & -1 & 1\\
        0 & 1 & -1\\
        5 & 2 & -3
        \end{bmatrix*}
    \end{gather*}
    
    \item Vamos reescrever a matriz do passo anterior sendo \textbf{multiplicada} por uma matriz com as incógnitas.
    
    A partir do último exemplo:
    
    $$
    \begin{bmatrix*}[r]
        2 & -1 & 1\\
        0 & 1 & -1\\
        5 & 2 & -3
    \end{bmatrix*} \cdot \begin{bmatrix*}[r] x\\y\\z \end{bmatrix*}
    $$
    
    Repare que elas estão escritas na \textbf{mesma ordem}, porém de cima pra baixo.
    
    \item Por fim, basta \textbf{igualar} esse produto aos termos independentes, também em forma de matriz.
    
    Continuando o exemplo dos passos anteriores:
    
    $$
    \begin{bmatrix*}[r]
        2 & -1 & 1\\
        0 & 1 & -1\\
        5 & 2 & -3
    \end{bmatrix*} \cdot \begin{bmatrix*}[r] 
    x\\y\\z 
    \end{bmatrix*}=\begin{bmatrix*}[r]
    0 & 3 & -2
    \end{bmatrix*}
    $$
    
\end{enumerate}