\chapter{Operações com matrizes}
\label{chap:opermat}

\section{Transposta}

\paragraph{Definição}

Dada a matriz $A$, a \textit{matriz transposta} de $A$ (dentoada por $A^t$) é a matriz que encontramos trocando as \textbf{linhas} da matriz $A$ por suas \textbf{colunas} e vice-versa (ordenadamente).

\Example

$$
\text{Se } A=\begin{bmatrix*}[r]
1 & 0 & -3 \\
-1 & 7 & 2
\end{bmatrix*}
\Rightarrow A^t=\begin{bmatrix*}[r]
1 & -1 \\
0 & 7 \\
-3 & 2
\end{bmatrix*}
$$

\section{Igualdade}

\paragraph{Definição:}

Dadas duas matrizes $A$ e $B$, dizemos que $A=B$ se, e somente se, $A$ e $B$ possuírem as \textbf{mesmas dimensões} e se, para todo elemento $a_{mn}$ da matriz $A$ tivermos um elemento $b_{mn}$ da matriz $B$ tal que $a_{mn}=b_{mn}$.

\Example

\begin{gather*}
\text{Dados } A=\begin{bmatrix*}[r]
a_{11} & a_{12} \\
a_{21} & a_{22}
\end{bmatrix*}
\text{ e }
B=\begin{bmatrix*}[r]
b_{11} & b_{12} \\
b_{21} & b_{22}
\end{bmatrix*}
\text{, se } A=B \text{ segue que:} \\
a_{11} = b_{11} \\
a_{12} = b_{12} \\
a_{21} = b_{21} \\
a_{22} = b_{22}
\end{gather*}

\Example

$$
\text{Se } A=B \text{, então, se } A=\begin{bmatrix*}[r]
-1 & 3 \\
5 & 4
\end{bmatrix*} \text{ sabemos que } B =\begin{bmatrix*}[r]
-1 & 3 \\
5 & 4
\end{bmatrix*} \text{ também.}
$$
\centerline{\footnotesize{Repare que todos os elementos são iguais, em suas respectivas posições}}

\section{Adição}

\paragraph{Definição:}

A adição de matrizes é a operação onde, dadas duas matrizes $A$ e $B$ com as mesmas dimensões, conseguimos uma matriz soma $(A+B)$ que será a matriz obtida adicionando-se os elementos correspondentes das matrizes $A$ e $B$.

\Example

\begin{gather*}
    \text{Dadas } A=\begin{bmatrix*}[r]
    a_{11} & a_{12} \\
    a_{21} & a_{22}
    \end{bmatrix*}
    \text{ e } B=\begin{bmatrix*}[r]
    b_{11} & b_{12} \\
    b_{21} & b_{22}
    \end{bmatrix*}
    \text{, a matriz } A+B \text{ será dada por:} \\
    A+B=\begin{bmatrix*}[r]
    a_{11} + b_{11} & a_{12} + b_{12} \\
    a_{21} + b_{21} & a_{22} + b_{22}
    \end{bmatrix*}    
\end{gather*}

\Example

\begin{gather*}
    \text{Se } A=\begin{bmatrix*}[r]
    -1 & 3 \\
    -4 & 7
    \end{bmatrix*} \text{ e } B =\begin{bmatrix*}[r]
    9 & 2\\
    -4 & 5
    \end{bmatrix*} \text{, segue que} \\
    A+B=\begin{bmatrix*}[r]
    -1+9 & 3+2 \\
    -4-4 & 7+5
    \end{bmatrix*}=\begin{bmatrix*}[r]
    8 & 5 \\
    -8 & 12
    \end{bmatrix*}
\end{gather*}

A soma de matrizes pode ser realizada em qualquer ordem ($A+B=B+A$ ou \textit{“A ordem dos tratores não altera o viaduto”}), pode também ser realizada sem prioridade específica ($A+(B+C)=(A+B)+C$ ou \textit{“tanto faz somar A com B e depois com C ou fazer B com C primeiro”}).

\section{Oposta}

\paragraph{Definição:}

Dada uma matriz $A$, sua \textit{matriz oposta} $X$ é a que satisfaz a condição:
$$X+A=0\Rightarrow X=-A$$
Encontramos a \textit{oposta} de $A$ (denotada por $-A$) \textbf{trocando os sinais de todos os elementos} da matriz $A$ original.

\Example

$$
\text{Se }A=\begin{bmatrix*}[r]
-1 & 3\\
5 & 8\\
-5 & -2
\end{bmatrix*} \Rightarrow -A=\begin{bmatrix*}[r]
1 & -3 \\
-5 & -8\\
5 & 2
\end{bmatrix*}
$$

\section{Subtração}

A \textit{subtração} de matrizes funciona de forma similar a adição, bastando apenas \textbf{trocar os sinais} da matriz que estamos subtraindo.

\begin{gather*}
    \text{Dadas }A=\begin{bmatrix*}[r]
    -1 & 7 \\ -3 & 2
    \end{bmatrix*} \text{ e }B=\begin{bmatrix*}[r]
    5 & -5 \\ -2 & 3
    \end{bmatrix*} \text{, a matriz }A-B \text{ será dada por} \\
    A-B=\begin{bmatrix*}[r]
    -1-5 & 7-(-5) \\ -3-(-2) & 2-3
    \end{bmatrix*} = \begin{bmatrix*}[r]
    -6 & 12 \\ -1 & -1
    \end{bmatrix*}
\end{gather*}

\centerline{\footnotesize{Repare que primeiro trocamos os sinais dos elementos da matriz $B$ para depois somá-los.}}

\section{Multiplicação por constante}

Na multiplicação de uma matriz por um número real basta multiplicarmos cada um de seus elementos pelo número.

$$
\text{Dada a matriz }A=\begin{bmatrix*}[r]
3 & 2 \\ -1 & 6
\end{bmatrix*}\Rightarrow2A=\begin{bmatrix*}[r]
2 \cdot 3 & 2 \cdot 2 \\ 2 \cdot (-1) & 2 \cdot 6
\end{bmatrix*}=\begin{bmatrix*}[r]
6 & 4 \\ -2 & 12
\end{bmatrix*}
$$
