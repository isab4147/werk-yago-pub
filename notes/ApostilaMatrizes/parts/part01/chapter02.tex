\chapter{Matrizes Notáveis}
Temos alguns tipos de matrizes com propriedades especiais, as quais vamos destacar nesta seção.

\section{Matriz Quadrada}

Denominamos \textit{quadrada} uma matriz onde o \textbf{número de linhas} é \textbf{igual} ao \textbf{número de colunas}. Nesse caso, ao invés de denotar seu tamanho pelos comprimentos $m \times n$ usamos somente uma de suas dimensões (dizemos então que a matriz é \textit{quadrada de ordem} $m$).

\begin{minipage}{\textwidth}
    $$
    B=
    \begin{bmatrix*}[r]
        a & b\\
        c & d
    \end{bmatrix*}$$
    \centerline{\footnotesize{Uma matriz $2 \times 2$ é quadrada, dizemos então que ela é \textit{quadrada de ordem} $2$.}}
\end{minipage}

Em toda matriz quadrada teremos duas \textbf{diagonais especiais}, chamadas \textit{principal} e \textit{secundária}.

\subsection{Diagonal principal}
A \textit{diagonal principal} será formada pelos elementos $a_{mn}$ onde ${m=n}$.

\begin{tikzmatrix}
        
    \matrix (M) [matstyle]{
    c_{11} \& c_{12} \& c_{13} \\
    c_{21} \& c_{22} \& c_{23} \\
    c_{31} \& c_{32} \& c_{33} \\
    };
    
    \node [left=of M] {$C=$};
    \scoped [on background layer]
    \fill[blue!30,rounded corners] (M-1-1.east) to (M-3-3.north) to (M-3-3.east) to (M-3-3.south) to (M-1-1.west) to (M-1-1.north) -- cycle;
        
\end{tikzmatrix}

\subsection{Diagonal secundária}
A \textit{secundária} será formada pelos elementos $a_{mn}$ tais que ${m+n=\mathcal{O}+1}$ onde $\mathcal{O}$ é a ordem da matriz quadrada.

\begin{tikzmatrix}
    
    \matrix (M) [matstyle]{
    c_{11} \& c_{12} \& c_{13} \\
    c_{21} \& c_{22} \& c_{23} \\
    c_{31} \& c_{32} \& c_{33} \\
    };
    
    \node [left=of M] {$C=$};
    \scoped [on background layer]
    \fill[blue!30,rounded corners] (M-1-3.west) to (M-3-1.north) to (M-3-1.west) to (M-3-1.south) to (M-1-3.east) to (M-1-3.north) -- cycle;
        
\end{tikzmatrix}

\centerline{\footnotesize{Podemos ver a diagonal secundária grifada, onde os elementos obedecem a relação $m+n=3+1 \left(\mathcal{O}=3\right)$}}

\section{Matriz identidade}
As matrizes \textit{identidade} são outro tipo especial que consiste em uma matriz quadrada que possui $1$’s em sua diagonal principal e $0$’s em todas as outras posições. Denominamos \textit{matriz identidade de ordem} $n$ (denotada por $I_n$) uma matriz quadrada dessa ordem que satisfaz essas condições.

Veremos mais a frente que essa matriz será equivalente ao número $1$ na operação de multiplicação de matrizes.

$$
I_2=
\begin{bmatrix*}[r]
    1 & 0 \\
    0 & 1
\end{bmatrix*}
$$
\centerline{\footnotesize{A matriz acima é uma \textit{matriz identidade de ordem} $2$}}

\section{Matriz nula}
Temos também as matrizes denominadas \textit{nulas}, as quais possuem todos os seus elementos igualando $0$ (nulos).

Uma matriz $m\times n$ que satisfaça essa condição é denotada por $0_{m \times n}$.
Caso essa matriz seja quadrada podemos denotá-la por $0_n$, onde $n$ é a ordem da matriz (a qual vamos chamar de \textit{matriz nula de ordem} $n$).

$$
0_{3 \times 2}=
\begin{bmatrix*}[r]
0 & 0 \\
0 & 0 \\
0 & 0
\end{bmatrix*}
$$
\centerline{\footnotesize{Acima temos uma \textit{matriz nula} $3 \times 2$}}

\section{Matriz linha/coluna}

Podemos ter também matrizes \textit{linha} ou \textit{coluna}, as quais são matrizes que se resumem a uma \textbf{linha} ou \textbf{coluna}, respectivamente.

\begin{alignat*}{3}
    A=\begin{bmatrix*}[r] a & b \end{bmatrix*} & \hspace{50pt} & B=\begin{bmatrix}a \\ b\end{bmatrix}
\end{alignat*}

\begin{center}
    \footnotesize{A matriz $A$ é uma \textit{matriz linha}, pois todos os seus elementos se\\ encontram em uma única linha, já a matriz $B$ é uma \textit{matriz coluna}}
\end{center}

\section{Matriz triangular}

Dizemos que uma matriz é \textit{triangular} se esta, além de se quadrada, tiver todos os elementos acima ou abaixo da diagonal principal \textbf{nulos}.

\subsection{Superior}
Uma matriz triangular é dita \textit{superior} se a parte \textbf{abaixo} da diagonal principal for nula.

$$
A_3=\begin{bmatrix*}[c]
a_{11} & a_{12} & a_{13}\\
0 & a_{22} & a_{23}\\
0 & 0 & a_{33}
\end{bmatrix*}
$$

\begin{center}
    \footnotesize{A matriz acima é triangular superior de ordem 3, pois todos os elementos abaixo da diagonal principal são nulos}
\end{center}

\subsection{Inferior}
Uma matriz triangular é dita \textit{inferior} se a parte \textbf{acima} da diagonal principal for nula.


$$
A_3=\begin{bmatrix*}[c]
a_{11} & 0 & 0\\
a_{21} & a_{22} & 0\\
a_{31} & a_{32} & a_{33}
\end{bmatrix*}
$$

\begin{center}
    \footnotesize{A matriz acima é triangular superior de ordem 3, pois todos os elementos abaixo da diagonal principal são nulos}
\end{center}

\section{Matriz diagonal}
Chamamos de \textit{diagonal} uma matriz quadrada onde todos os elementos que não pertencem à diagonal principal são nulos.

$$
A_4=\begin{bmatrix*}[c]
a_{11} & 0 & 0 & 0\\
0 & a_{22} & 0 & 0\\
0 & 0 & a_{33} & 0\\
0 & 0 & 0 & a_{44}
\end{bmatrix*}
$$

\begin{center}
    \footnotesize{A matriz acima é diagonal de ordem 4}
\end{center}

\paragraph{Nota:}
Repare que se uma matriz triangular for \textbf{simultaneamente} superior e inferior, esta será uma matriz chamada \textit{diagonal}.

\section{Matriz de Vandermonde}

Vamos analisar um pequeno exemplo:

$$
\begin{bmatrix*}[r]
1 & 1 & 1\\
3 & 5 & 2\\
9 & 25 & 4
\end{bmatrix*}
$$

Repare como a matriz acima pode ser escrita como:

$$
\begin{bmatrix*}[r]
3^0 & 5^0 & 2^0\\
3^1 & 5^1 & 2^1\\
3^2 & 5^2 & 2^2
\end{bmatrix*}
$$

As matrizes que têm essa propriedade são chamadas \textit{matrizes de Vandermonde}.

\paragraph{Definição:}
A matriz de \textit{Vandermonde} é aquela onde cada \textbf{linha} ou \textbf{coluna} representa um \textbf{termo} de uma \textbf{progressão geométrica} de base $a_n$.

\vspace{-2em}

\begin{multicols}{2}
\null
\noindent A matriz de \textit{Vandermonde} ao lado possui PG's em suas \textbf{linhas}. Podemos ver que os expoentes começam em $0$ e vão até $n-1$, aumentando para a direita.

\vskip0.2em\null

\columnbreak

$$
V_{m \times n} =
\begin{bmatrix}
    a_1^0 & a_1^1 & \cdots & a_1^{n-1}\tikzmark{mv1r1} \\[0.5em]
    a_2^0 & a_2^1 & \cdots & a_2^{n-1}\tikzmark{mv1r2} \\
    \vdots  & \vdots  & \ddots & \vdots  \\
    a_m^0 & a_m^1 & \cdots & a_m^{n-1}\tikzmark{mv1r3}
\end{bmatrix}
$$
\end{multicols}

\begin{multicols}{2}

\null

\noindent A matriz de \textit{Vandermonde} ao lado possui PG's em suas \textbf{colunas}. Podemos ver que os expoentes começam em $0$ e vão até $n-1$, aumentando para baixo.

\vskip1em\null

\columnbreak

\null\vskip1em

$$
V_{n \times m} =
\begin{bmatrix}
    \tikzmark{mv2c1}a_1^0 & \tikzmark{mv2c2}a_2^0 & \cdots & \tikzmark{mv2c3}a_m^0 \\[0.5em]
    a_1^1 & a_2^1 & \cdots & a_m^1 \\
    \vdots  & \vdots  & \ddots & \vdots  \\
    a_1^{n-1} & a_2^{n-1} & \cdots & a_m^{n-1}
\end{bmatrix}
$$

\null

\end{multicols}
\begin{tikzoverlay}[]
    \draw[latex-] (mv1r1) ++(5pt,2pt) -- +(0.5cm, 0) node[anchor=west] {PG de base $a_1$};
    \draw[latex-] (mv1r2) ++(5pt,3pt) -- +(0.5cm, 0) node[anchor=west] {PG de base $a_2$};
    \draw[latex-] (mv1r3) ++(5pt,4pt) -- +(0.5cm, 0) node[anchor=west] {PG de base $a_m$};
    
    \draw[latex-] (mv2c1) ++(5pt, 10pt) -- +(0, 0.5cm) node(a)[]{};
    
    \node at ($(a) +(20pt, -3pt)$) [above left] {PG de base $a_1$};
    
    \draw[latex-] (mv2c2) ++(5pt, 10pt) -- +(0, 1cm) node(b) [] {};
    
    \node at ($(b) +(10pt, -3pt)$) [above] {PG de base $a_2$};
    
    \draw[latex-] (mv2c3) ++(5pt, 10pt) -- +(0, 0.5cm) node(c)[] {};
    
    \node at ($(c) +(-20pt, -3pt)$)[above right]{PG de base $a_m$};
\end{tikzoverlay}