\chapter{Multiplicação de matrizes}
\label{chap:multmat}

\section{Definição:}
Dadas duas matrizes $A_{(m \times n)}$  e $B_{(n \times j)}$, a \textit{multiplicação} de $A$ por $B$, escrita como $A\cdot B$ vai se dar em algumas etapas:

\subsection{Condição de existência}
A \textit{multiplicação de matrizes} só vai ser possível se o \textbf{número de colunas} da primeira for \textbf{igual} ao \textbf{número de linhas} da segunda ($A_{(m \times n)}$  e $B_{(n \times j)}$).

\subsection{Formato}
Após checada a condição de existência podemos prosseguir para o cálculo da multiplicação. O resultado dessa multiplicação terá dimensões $m \times j$, ou seja, terá \textbf{tantas linhas} quanto a matriz $A$ e \textbf{tantas colunas} quanto a matriz $B$.

\subsection{Cálculo}
A matriz $C=A\cdot B$ terá seus elementos $c_{ik}$ obtidos tomando-se a \textbf{linha} $i$ da matriz $A$ e a \textbf{coluna} $k$ da matriz $B$ \textbf{multiplicando-se} seus elementos \textbf{respectivos} (1\textsuperscript{o} com 1\textsuperscript{o}, 2\textsuperscript{o} com 2\textsuperscript{o}, em diante...) e \textbf{somando os produtos}.

\paragraph{Nota:}
A matriz $A\cdot B$ será \textbf{diferente} da matriz $B\cdot A$, não só por conta da \textbf{condição de existência} e do \textbf{formato}, mas também porque o cálculo de $A\cdot B$ tomará as \textbf{linhas} de $A$ e as \textbf{colunas} de $B$, e o cálculo de $B\cdot A$ tomará as \textbf{linhas} de $B$ e as \textbf{colunas} de $A$, o que é um processo \textbf{completamente} diferente.

\Example

\begin{gather*}
    \text{Dados }A=\begin{bmatrix*}[r]
    a_{11} & a_{12} \\ a_{21} & a_{22} 
    \end{bmatrix*} \text{ e }B=\begin{bmatrix*}[r]
    b_{11} & b_{12} \\ b_{21} & b_{22}
    \end{bmatrix*} \text{, seja } C=A \cdot B
\end{gather*}
Calculamos $C$ fazendo:
\begin{gather*}
    C=\begin{bmatrix*}[r]
    c_{11} & c_{12} \\ c_{21} & c_{22}
    \end{bmatrix*}\text{, onde }\\
    \begin{aligned}
        c_{11}=a_{11} \cdot b_{11} + a_{12} \cdot b_{21} \\
        c_{12}=a_{11} \cdot b_{12} + a_{12} \cdot b_{22} \\
        c_{21}=a_{21} \cdot b_{11} + a_{22} \cdot b_{21} \\
        c_{22}=a_{21} \cdot b_{12} + a_{22} \cdot b_{22} \\
    \end{aligned} \\
    \text{resultando em }C=\begin{bmatrix*}[r]
    a_{11} \cdot b_{11} + a_{12} \cdot b_{21} &
    a_{11} \cdot b_{12} + a_{12} \cdot b_{22} \\
    a_{21} \cdot b_{11} + a_{22} \cdot b_{21} &
    a_{21} \cdot b_{12} + a_{22} \cdot b_{22}
    \end{bmatrix*}
\end{gather*}

\subsection{Ilustrado}

\begin{center}
    \begin{tikzpicture}[>=latex, ampersand replacement=\&]

\matrix (A) [matstyle, nodes = {node style ge}] at (0,0)
{
  a_{11} \& a_{12} \& \ldots \& a_{1n}\\
  |[node style sp]| a_{21} \& |[node style sp]| a_{22} \& \ldots \& |[node style sp]| a_{2n} \\
  \vdots \& \vdots \& \ddots \& \vdots  \\
  a_{m1} \& a_{m2} \& \ldots \& a_{mn}  \\
};

\node [draw,below=10pt] at (A.south) {$A$ : \textcolor{red}{$m$ linhas} $n$ colunas};

\matrix (B) [matstyle, nodes = {node style ge}] at (6*\myunit,6*\myunit)
{
  b_{11} \& |[node style sp]| b_{12} \& \ldots \& b_{1j}  \\
  b_{21} \& |[node style sp]| b_{22} \& \ldots \& b_{2j}  \\
  \vdots \& \vdots \& \ddots \& \vdots  \\
  b_{i1} \& |[node style sp]| b_{i2} \& \ldots \& b_{ij}  \\
};

\node [draw,above=10pt] at (B.north) { $B$ : $i$ linhas \textcolor{red}{$j$ colunas}};

\matrix (C) [matstyle, nodes = {node style ge}] at (6*\myunit,0)
{ 
c_{11} \& c_{12} \& \ldots \& c_{1j} \\
  c_{21} \& |[node style sp,red]| c_{22} \& \ldots \& c_{2j} \\
  \vdots \& \vdots \& \ddots \& \vdots \\
  c_{m1} \& c_{m2} \& \ldots \& c_{mj} \\
};

\draw[blue] (A-2-1.north) -- (C-2-2.north)
            (A-2-1.south) -- (C-2-2.south)
            (B-1-2.west)  -- (C-2-2.west)
            (B-1-2.east)  -- (C-2-2.east);
            
\draw[<->,red](A-2-1) to[in=180,out=90]	node[arrow style mul] (x) {$a_{21}\times b_{12}$} (B-1-2);

\draw[<->,red] (A-2-2) to[in=180,out=90]
	node[arrow style mul] (y) {$a_{22}\times b_{22}$} (B-2-2);
	
\draw[<->,red](A-2-4) to[in=180,out=90]
	node[arrow style mul] (z) {$a_{2n}\times b_{i2}$} (B-4-2);
	
\draw[red,->] (x) to node[arrow style plus] {$+$} (y)%
                  to node[arrow style plus] {$+\raisebox{.5ex}{\ldots}+$} (z)%
                  to (C-2-2.north west);


\node [draw,below=10pt] at (C.south) 
    {$ C=A\times B$ : \textcolor{red}{$m$ linhas}  \textcolor{red}{$j$ colunas}};

\end{tikzpicture}
\end{center}

\section{Na prática}

\begin{gather*}
    \text{Dadas as matrizes }A=\begin{bmatrix*}[r]
    1 & -5 & 7 \\ 12 & 3 & -9
    \end{bmatrix*} \text{ e }B=\begin{bmatrix*}[r]
    4 & -2 \\ -3 & 1 \\ 2 & 7
    \end{bmatrix*} \\ \text{calculamos }C=A \cdot B \text{ pelos seguinte passos:}
\end{gather*}

\begin{enumerate}
    \item Primeiro devemos checar a condição de existência do produto: 
    
    Podemos ver que $A_{2\times3}\cdot B_{3\times 2}$ obedece à condição de existência. \checkmark
    
    \item Agora vamos determinar as dimensões do produto $A\cdot B$: 
    
    Tomando a quantidade de linhas de $A$ (2) e a quantidade de colunas de $B$ (2) temos $C_{2\times2}$.
    
    \item Agora preencheremos a matriz resultante:

\end{enumerate}

\subsection{Usando a definição}

\begin{gather*}
    A=\begin{bmatrix*}[r]
    1 & -5 & 7 \\ 12 & 3 & -9
    \end{bmatrix*} \text{ e }B=\begin{bmatrix*}[r]
    4 & -2 \\ -3 & 1 \\ 2 & 7
    \end{bmatrix*} \text{, logo }C=\begin{bmatrix*}[r]
    c_{11} & c_{12} \\ c_{21} & c_{22}
    \end{bmatrix*}\text{, onde} \\
    c_{11} = 1 \cdot 4 + (-5) \cdot (-3) + 7 \cdot 2 \\
    c_{12} = 1 \cdot (-2) + (-5) \cdot 1 + 7 \cdot 7 \\
    c_{21} = 12 \cdot 4 + 3 \cdot (-3) + (-9) \cdot 2 \\
    c_{22} = 12 \cdot (-2) + 3 \cdot 1 + (-9) \cdot 7 \\
    \text{resultando em }C=\begin{bmatrix*}[r]
    4 + 15 + 14 &
    (-2) + (-5) + 49 \\
    48 + (-9) + (-18) &
    (-24) + 3 + (-63)
    \end{bmatrix*}=\begin{bmatrix*}[r]
    34 & 42 \\
    21 & -80
    \end{bmatrix*}
\end{gather*}

\subsection{Uma visão alternativa}

Vamos preencher a matriz $C$ num ziguezague, conforme a figura abaixo.

\begin{center}
    \begin{tikzpicture}[node distance = 8pt,every left delimiter/.style={xshift=.75em},
        every right delimiter/.style={xshift=-.75em}, ampersand replacement=\&]
        
        \matrix(M)[matstyle]{
            c_{11} \& c_{12} \\
            c_{21} \& c_{22} \\
        };
        \node [left=of M] {$C=$};
        \begin{scope}[on background layer, every node/.style={circle,fill=white, above=-5pt,inner sep=1pt}]
            \draw[->, color=redAr!50] (M-1-1.north) to node[] {\small1} (M-1-2.north);
            \draw[->, color=redAr!50] (M-1-2.center) to node[] {\small2} (M-2-1.center);
            \draw[->, color=redAr!50] (M-2-1.south) to node[] {\small3} (M-2-2.south);
        \end{scope}
    \end{tikzpicture}
\end{center}

Começamos, então, pelo elemento $c_{\textcolor{red}{1}\textcolor{teal}{1}}$. O digito \textcolor{red}{vermelho} se refere ao número da \textbf{linha} da primeira matriz e o digito \textcolor{teal}{azul} ao número da \textbf{coluna} da segunda matriz.

Nesse caso pegaremos a primeira \textbf{linha} da matriz $B$ e a primeira \textbf{coluna} da matriz $A$ e vamos multiplicá-las.

$$
c_{11}=\begin{bmatrix*}[r]
1 & -5 & 7
\end{bmatrix*}\cdot\begin{bmatrix*}[r]
4 \\ -3 \\ 2
\end{bmatrix*}
$$

Agora podemos ter uma visão mais clara do processo. Se fizermos a \textit{transposta} da linha de $B$ ficamos com:

$$
c_{11}=\begin{bmatrix*}[r]
1 \\ -5 \\ 7
\end{bmatrix*}\cdot\begin{bmatrix*}[r]
4 \\ -3 \\ 2
\end{bmatrix*}
$$

A operação pode, então, ser feita facilmente agora. Vamos multiplicar os elementos correspondentes e somar todos no final, conforme é mostrado abaixo:

\begin{multicols}{3}
    $$
    \begin{array}{l c r}
        c_{11}= & ~ & 1\cdot4 \\
        ~ & \multicolumn{2}{l}{(-5)\cdot(-3)}\\
        ~ & + & 7\cdot2\\
        \cline{2-3} ~ & ~ & 33
    \end{array}
    $$
    \columnbreak
    \centering
    
    \null
    \noindent OU
    \vskip2em\null
    
    \columnbreak
    \null
    
    \begin{align*}
        c_{11}&=\tikzmark{a1}1\cdot\tikzmark{b1}4 + (-\tikzmark{a2}5)\cdot(-\tikzmark{b2}3) + \tikzmark{a3}7\cdot\tikzmark{b3}2 \\
              &=4 + 15 + 14 = 33
    \end{align*}
    
    \vskip1em\null
\end{multicols}
\begin{tikzoverlay}[node distance=2em]
    \foreach \num in {1, 3} {
        \node (aa\num) [above = of a\num, xshift=2pt] {\footnotesize $a_{1\num}$};
        \node (ba\num) [above = of b\num, xshift= 6pt] {\footnotesize $b_{1\num}$};
    }
    \node (aa2) [above = of a2] {\footnotesize $a_{12}$};
    \node (ba2) [above = of b2] {\footnotesize $b_{12}$};
    
    \foreach \num in {1, 3}{
        \draw[->] (aa\num) -- +(0,-1.8em);
        \draw[->] (ba\num) -- +(-3pt, -1.9em);
    }
    
    \draw[->] (aa2) -- +(0, -1.8em);
    \draw[->] (ba2) -- +(0, -1.8em);

\end{tikzoverlay}
\begin{minipage}{\textwidth}
    \begin{multicols}{2}
        \noindent Essa operação seria o equivalente a preencher o primeiro elemento do diagrama:
        \smallskip
    
        \begin{center}
    \begin{tikzpicture}[>=latex, scale=\myscale, ampersand replacement=\&]

\matrix (B) [matstyle, nodes = {node style ge}] at (0,0)
{
  |[node style sp]| 1 \& |[node style sp]|-5 \& |[node style sp]|7\\
  12 \& 3 \& -9 \\
};

\node [draw,below=10pt] at (B.south) {linha $1$};

\matrix (A) [matstyle, nodes = {node style ge}] at (5*\myunit,5*\myunit)
{
  |[node style sp]| 4 \& -2\\
  |[node style sp]| -3 \& 1\\
  |[node style sp]| 2 \& 7\\
};

\node [draw,above=10pt] at (A.north) {coluna $1$};

\matrix (C) [matstyle, nodes = {node style ge}] at (5*\myunit,0)
{ 
  |[node style sp,red]| 33 \& c_{12} \\
  c_{21} \& c_{22} \\
};

\draw[blue] (B-1-1.north) -- (C-1-1.north)
            (B-1-1.south) -- (C-1-1.south)
            (A-1-1.west)  -- (C-1-1.west)
            (A-1-1.east)  -- (C-1-1.east);
            
\draw[<->,red](B-1-1) to[in=180,out=90]	node[arrow style mul] (x) {$1\times 4$} (A-1-1);

\draw[<->,red] (B-1-2) to[in=180,out=90]
	node[arrow style mul] (y) {$-5\times -3$} (A-2-1);
	
\draw[<->,red](B-1-3) to[in=180,out=90]
	node[arrow style mul] (z) {$7\times 2$} (A-3-1);
	
\draw[red,->] (x) to node[arrow style plus] {$+$} (y) to node[arrow style plus] {$+$} (z) to (C-1-1.north west);


\node [draw,below=10pt] at (C.south) 
    {$c_{11}$};

\end{tikzpicture}
\end{center}
        
        \columnbreak
        
        \noindent Podemos fazer o mesmo com os outros elementos da matriz $C$:
        \begin{align*}
            c_{12}&=1\cdot(-2) + (-5) \cdot 1 + 7 \cdot 7=42\\
            c_{21}&= 12\cdot4 +3\cdot(-3)+ (-9)\cdot2=21\\
            c_{22}&= 12\cdot(-2)+3 \cdot1 +(-9)\cdot7=-84
        \end{align*}
        
        \noindent Preenchendo a matriz $C$, temos:
        $$
        C=
        \begin{bmatrix*}[r]
            33 & 42 \\21 & -84
        \end{bmatrix*}
        $$
        \null
    \end{multicols}
\end{minipage}
