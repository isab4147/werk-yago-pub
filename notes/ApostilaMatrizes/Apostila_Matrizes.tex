\documentclass{IMTbook}

%\usepackage{fontspec} % specify font

\usepackage{CJKutf8} %japanese char

\usepackage{caption, multicol, empheq, booktabs}
\usepackage[enums,externalize,weirdsymbols]{IMTtikz}
\usetikzlibrary{matrix}

\externalizefigs

% define general tikz matrix style
\tikzset{matstyle/.style={matrix of math nodes,inner sep=3pt, left delimiter={[}, right delimiter={]}}}


%define new env for tikz matrix
\NewEnvironx{tikzmatrix}[1][1=]{\begin{center}
		\begin{tikzpicture}[node distance=0pt, every left delimiter/.style={xshift=.75em},
		every right delimiter/.style={xshift=-.75em}, ampersand replacement=\&, #1]}{
		\end{tikzpicture}
	\end{center}}

\newcommand{\myscale}{0.8}
\newcommand{\myunit}{\myscale*1 cm}

\tikzset{node style sp/.style={draw,circle,minimum size=\myunit}, 
	node style ge/.style={circle,minimum size=\myunit},
	arrow style mul/.style={draw,sloped,midway,fill=white},
	arrow style plus/.style={midway,sloped,fill=white}}

\title{Matrizes, Determinantes e Sistemas lineares}
\subtitle{uma apostila para o ensino médio}
\author{Isabella B.}

\begin{document}

\maketitle

\tableofcontents

%begin content inputting here

\part{Trigonometria no triângulo retângulo}

\chapter{Introdução}

\section{Um pouco de contexto}

Historicamente, os determinantes eram usados muito anteriormente em relação às matrizes, e eram considerados uma propriedade dos sistemas lineares (abordados na \autoref{p:3}). Os determinantes "determinam"~se um sistema linear qualquer tem ou não solução única.

O japonês \textit{Seki Takakazu} (\begin{CJK}{UTF8}{min}関 孝和\end{CJK}) leva o crédito da descoberta da resultante e do determinante (1683-1710) e, na Europa, depois que \textit{Leibniz} introduziu o estudo dos determinantes, \textit{Gabriel Cramer} expandiu a teoria relacionando-a à conjuntos de equações.

\textit{Vandermonde} foi o primeiro a tratar os determinantes como funções independentes de sistemas lineares e \textit{Laplace} contribuiu com o método geral para escrever um determinante através de seus cofatores. \textit{Lagrange} e \textit{Gauss}, utilizando os determinantes na teoria dos números, fizeram avanços importantes na teoria, e \textit{Gauss} foi o primeiro a usar o nome \textit{determinante}, embora não no sentido atual.

Foi \textit{Cauchy} quem primeiro introduziu o termo \textit{determinante} no sentido aceito atualmente, num trabalho publicado em 1812, anteriormente o termo \textit{resultante} havia sido utilizado por \textit{Laplace}.

Quem mais contribuiu para a teoria de determinantes foi \textit{Carl G. J. Jacobi} (1804-1851). A simplicidade atual da apresentação dessa teoria se deve a ele.


\section{Definição e resolução para \texorpdfstring{$n \leqslant 3$}{n <= 3}}

Podemos dizer que o \textit{determinante} de uma matriz quadrada é o seu \textbf{valor numérico}.

Denotamos o \textit{determinante} de uma matriz quadrada qualquer $A$ por

\begin{center}
    \begin{romanlistinline}
        \item $\det A$ , ou \item $|A|$
    \end{romanlistinline}
\end{center}

\Example

$$
\text{Seja a matriz }A=\begin{bmatrix*}[r]
1 & 2\\ 3 & 4
\end{bmatrix*}\text{, denotamos seu determinante por }\det A=\det\begin{bmatrix*}[r]
1 & 2\\ 3 & 4
\end{bmatrix*}=\begin{vmatrix}
1 & 2\\ 3 & 4
\end{vmatrix}
$$

\subsection{Caso \texorpdfstring{$n=1$}{n=1}}

Quando temos uma matriz com um único elemento seu determinante será esse elemento. 

\Example

$$
\text{Dada a matriz }A=[-2]\text{, }\det A=-2
$$

\subsection{Caso \texorpdfstring{$n=2$}{n=2}}

Quando temos uma matriz quadrada de ordem 2 basta tomar o produto dos elementos da \textbf{diagonal principal} e \textbf{subtraí-lo} pelo produto dos elementos da \textbf{diagonal secundária}.

\Example

$$
\text{Dada a matriz }A=\begin{bmatrix*}[r]
    2 & 3\\ -3 & 1
\end{bmatrix*}\text{, }\det A=2\cdot 1 -3\cdot (-3) \Rightarrow \det A=2+9=11 
$$

\subsection{Caso \texorpdfstring{$n=3$}{n=3}}

Com uma matriz quadrada de grau 3 o processo é bastante similar ao do grau $2$ (descrito logo acima). 

Primeiro devemos \textbf{repetir} a primeira e a segunda colunas da matriz à sua direita (como mostrado em \textcolor{red}{vermelho}):

\begin{gather*}
    A=\begin{bmatrix*}[r]
        a_{11} & a_{12} & a_{13} \\
        a_{21} & a_{22} & a_{23} \\
        a_{31} & a_{32} & a_{33}
    \end{bmatrix*} \Rightarrow \\
    \det A= \begin{vmatrix*}[r]
        a_{11} & a_{12} & a_{13} \\
        a_{21} & a_{22} & a_{23} \\
        a_{31} & a_{32} & a_{33}
    \end{vmatrix*} \color{red} \begin{matrix}
        a_{11} & a_{12} \\
        a_{21} & a_{22} \\
        a_{31} & a_{32}
    \end{matrix}
\end{gather*}

Em seguida devemos calcular o \textbf{produto} das diagonais formadas:

\begin{center}
        \begin{tikzpicture}[node distance=-6pt, every left delimiter/.style={xshift=.50em},
        every right delimiter/.style={xshift=-.50em}, ampersand replacement=\&]
        
        \matrix(M1)[matstyle]{
            a_{11} \& a_{12} \& a_{13} \\
            a_{21} \& a_{22} \& a_{23} \\
            a_{31} \& a_{32} \& a_{33} \\
        };
        \matrix(M2)[matrix of math nodes, right=of M1, nodes={color=red!80}]{
            a_{11} \& a_{12} \\
            a_{21} \& a_{22} \\
            a_{31} \& a_{32} \\
        };
    
        \begin{scope}[on background layer, every node/.style={fill=white, circle, draw, draw=white, minimum width=14pt}]
            \draw[->, redAr!80] (M1-1-1.north west) -- (M1-3-3.south east);
            \draw[->, redAr!80] (M1-1-2.north west) -- (M2-3-1.south east);
            \draw[->, redAr!80] (M1-1-3.north west) -- (M2-3-2.south east);
            \draw[->, greenAr!80] (M2-1-2.north east) -- (M1-3-3.south west);
            \draw[->, greenAr!80] (M2-1-1.north east) -- (M1-3-2.south west);
            \draw[->, greenAr!80] (M1-1-3.north east) -- (M1-3-1.south west);
            
            \foreach \i in {1,2,3}{
                \foreach \j in {1,2,3}
                    \node[] at (M1-\i -\j) {  };
            }
            \foreach \i in {1,2,3}{
                \foreach \j in {1,2}
                    \node [] at (M2-\i -\j) {  };
            }
        \end{scope}
    \end{tikzpicture}
\end{center}

Ficamos com:

\begin{gather*}
    \det A=\\
    a_{11} \cdot a_{22} \cdot a_{33} + 
    a_{12} \cdot a_{23} \cdot a_{31} +
    a_{13} \cdot a_{21} \cdot a_{32} \\
    - a_{13} \cdot a_{22} \cdot a_{31}
    - a_{22} \cdot a_{23} \cdot a_{32}
    - a_{31} \cdot a_{21} \cdot a_{33}
\end{gather*}

Note que as diagonais na direção da principal devem ser \textbf{somadas} e as contrárias devem ser \textbf{subtraídas}.

\paragraph{Nota:}
Esse método é chamado de \textit{Regra de Sarrus}.
\chapter{Matrizes Notáveis}
Temos alguns tipos de matrizes com propriedades especiais, as quais vamos destacar nesta seção.

\section{Matriz Quadrada}

Denominamos \textit{quadrada} uma matriz onde o \textbf{número de linhas} é \textbf{igual} ao \textbf{número de colunas}. Nesse caso, ao invés de denotar seu tamanho pelos comprimentos $m \times n$ usamos somente uma de suas dimensões (dizemos então que a matriz é \textit{quadrada de ordem} $m$).

\begin{minipage}{\textwidth}
    $$
    B=
    \begin{bmatrix*}[r]
        a & b\\
        c & d
    \end{bmatrix*}$$
    \centerline{\footnotesize{Uma matriz $2 \times 2$ é quadrada, dizemos então que ela é \textit{quadrada de ordem} $2$.}}
\end{minipage}

Em toda matriz quadrada teremos duas \textbf{diagonais especiais}, chamadas \textit{principal} e \textit{secundária}.

\subsection{Diagonal principal}
A \textit{diagonal principal} será formada pelos elementos $a_{mn}$ onde ${m=n}$.

\begin{tikzmatrix}
        
    \matrix (M) [matstyle]{
    c_{11} \& c_{12} \& c_{13} \\
    c_{21} \& c_{22} \& c_{23} \\
    c_{31} \& c_{32} \& c_{33} \\
    };
    
    \node [left=of M] {$C=$};
    \scoped [on background layer]
    \fill[blue!30,rounded corners] (M-1-1.east) to (M-3-3.north) to (M-3-3.east) to (M-3-3.south) to (M-1-1.west) to (M-1-1.north) -- cycle;
        
\end{tikzmatrix}

\subsection{Diagonal secundária}
A \textit{secundária} será formada pelos elementos $a_{mn}$ tais que ${m+n=\mathcal{O}+1}$ onde $\mathcal{O}$ é a ordem da matriz quadrada.

\begin{tikzmatrix}
    
    \matrix (M) [matstyle]{
    c_{11} \& c_{12} \& c_{13} \\
    c_{21} \& c_{22} \& c_{23} \\
    c_{31} \& c_{32} \& c_{33} \\
    };
    
    \node [left=of M] {$C=$};
    \scoped [on background layer]
    \fill[blue!30,rounded corners] (M-1-3.west) to (M-3-1.north) to (M-3-1.west) to (M-3-1.south) to (M-1-3.east) to (M-1-3.north) -- cycle;
        
\end{tikzmatrix}

\centerline{\footnotesize{Podemos ver a diagonal secundária grifada, onde os elementos obedecem a relação $m+n=3+1 \left(\mathcal{O}=3\right)$}}

\section{Matriz identidade}
As matrizes \textit{identidade} são outro tipo especial que consiste em uma matriz quadrada que possui $1$’s em sua diagonal principal e $0$’s em todas as outras posições. Denominamos \textit{matriz identidade de ordem} $n$ (denotada por $I_n$) uma matriz quadrada dessa ordem que satisfaz essas condições.

Veremos mais a frente que essa matriz será equivalente ao número $1$ na operação de multiplicação de matrizes.

$$
I_2=
\begin{bmatrix*}[r]
    1 & 0 \\
    0 & 1
\end{bmatrix*}
$$
\centerline{\footnotesize{A matriz acima é uma \textit{matriz identidade de ordem} $2$}}

\section{Matriz nula}
Temos também as matrizes denominadas \textit{nulas}, as quais possuem todos os seus elementos igualando $0$ (nulos).

Uma matriz $m\times n$ que satisfaça essa condição é denotada por $0_{m \times n}$.
Caso essa matriz seja quadrada podemos denotá-la por $0_n$, onde $n$ é a ordem da matriz (a qual vamos chamar de \textit{matriz nula de ordem} $n$).

$$
0_{3 \times 2}=
\begin{bmatrix*}[r]
0 & 0 \\
0 & 0 \\
0 & 0
\end{bmatrix*}
$$
\centerline{\footnotesize{Acima temos uma \textit{matriz nula} $3 \times 2$}}

\section{Matriz linha/coluna}

Podemos ter também matrizes \textit{linha} ou \textit{coluna}, as quais são matrizes que se resumem a uma \textbf{linha} ou \textbf{coluna}, respectivamente.

\begin{alignat*}{3}
    A=\begin{bmatrix*}[r] a & b \end{bmatrix*} & \hspace{50pt} & B=\begin{bmatrix}a \\ b\end{bmatrix}
\end{alignat*}

\begin{center}
    \footnotesize{A matriz $A$ é uma \textit{matriz linha}, pois todos os seus elementos se\\ encontram em uma única linha, já a matriz $B$ é uma \textit{matriz coluna}}
\end{center}

\section{Matriz triangular}

Dizemos que uma matriz é \textit{triangular} se esta, além de se quadrada, tiver todos os elementos acima ou abaixo da diagonal principal \textbf{nulos}.

\subsection{Superior}
Uma matriz triangular é dita \textit{superior} se a parte \textbf{abaixo} da diagonal principal for nula.

$$
A_3=\begin{bmatrix*}[c]
a_{11} & a_{12} & a_{13}\\
0 & a_{22} & a_{23}\\
0 & 0 & a_{33}
\end{bmatrix*}
$$

\begin{center}
    \footnotesize{A matriz acima é triangular superior de ordem 3, pois todos os elementos abaixo da diagonal principal são nulos}
\end{center}

\subsection{Inferior}
Uma matriz triangular é dita \textit{inferior} se a parte \textbf{acima} da diagonal principal for nula.


$$
A_3=\begin{bmatrix*}[c]
a_{11} & 0 & 0\\
a_{21} & a_{22} & 0\\
a_{31} & a_{32} & a_{33}
\end{bmatrix*}
$$

\begin{center}
    \footnotesize{A matriz acima é triangular superior de ordem 3, pois todos os elementos abaixo da diagonal principal são nulos}
\end{center}

\section{Matriz diagonal}
Chamamos de \textit{diagonal} uma matriz quadrada onde todos os elementos que não pertencem à diagonal principal são nulos.

$$
A_4=\begin{bmatrix*}[c]
a_{11} & 0 & 0 & 0\\
0 & a_{22} & 0 & 0\\
0 & 0 & a_{33} & 0\\
0 & 0 & 0 & a_{44}
\end{bmatrix*}
$$

\begin{center}
    \footnotesize{A matriz acima é diagonal de ordem 4}
\end{center}

\paragraph{Nota:}
Repare que se uma matriz triangular for \textbf{simultaneamente} superior e inferior, esta será uma matriz chamada \textit{diagonal}.

\section{Matriz de Vandermonde}

Vamos analisar um pequeno exemplo:

$$
\begin{bmatrix*}[r]
1 & 1 & 1\\
3 & 5 & 2\\
9 & 25 & 4
\end{bmatrix*}
$$

Repare como a matriz acima pode ser escrita como:

$$
\begin{bmatrix*}[r]
3^0 & 5^0 & 2^0\\
3^1 & 5^1 & 2^1\\
3^2 & 5^2 & 2^2
\end{bmatrix*}
$$

As matrizes que têm essa propriedade são chamadas \textit{matrizes de Vandermonde}.

\paragraph{Definição:}
A matriz de \textit{Vandermonde} é aquela onde cada \textbf{linha} ou \textbf{coluna} representa um \textbf{termo} de uma \textbf{progressão geométrica} de base $a_n$.

\vspace{-2em}

\begin{multicols}{2}
\null
\noindent A matriz de \textit{Vandermonde} ao lado possui PG's em suas \textbf{linhas}. Podemos ver que os expoentes começam em $0$ e vão até $n-1$, aumentando para a direita.

\vskip0.2em\null

\columnbreak

$$
V_{m \times n} =
\begin{bmatrix}
    a_1^0 & a_1^1 & \cdots & a_1^{n-1}\tikzmark{mv1r1} \\[0.5em]
    a_2^0 & a_2^1 & \cdots & a_2^{n-1}\tikzmark{mv1r2} \\
    \vdots  & \vdots  & \ddots & \vdots  \\
    a_m^0 & a_m^1 & \cdots & a_m^{n-1}\tikzmark{mv1r3}
\end{bmatrix}
$$
\end{multicols}

\begin{multicols}{2}

\null

\noindent A matriz de \textit{Vandermonde} ao lado possui PG's em suas \textbf{colunas}. Podemos ver que os expoentes começam em $0$ e vão até $n-1$, aumentando para baixo.

\vskip1em\null

\columnbreak

\null\vskip1em

$$
V_{n \times m} =
\begin{bmatrix}
    \tikzmark{mv2c1}a_1^0 & \tikzmark{mv2c2}a_2^0 & \cdots & \tikzmark{mv2c3}a_m^0 \\[0.5em]
    a_1^1 & a_2^1 & \cdots & a_m^1 \\
    \vdots  & \vdots  & \ddots & \vdots  \\
    a_1^{n-1} & a_2^{n-1} & \cdots & a_m^{n-1}
\end{bmatrix}
$$

\null

\end{multicols}
\begin{tikzoverlay}[]
    \draw[latex-] (mv1r1) ++(5pt,2pt) -- +(0.5cm, 0) node[anchor=west] {PG de base $a_1$};
    \draw[latex-] (mv1r2) ++(5pt,3pt) -- +(0.5cm, 0) node[anchor=west] {PG de base $a_2$};
    \draw[latex-] (mv1r3) ++(5pt,4pt) -- +(0.5cm, 0) node[anchor=west] {PG de base $a_m$};
    
    \draw[latex-] (mv2c1) ++(5pt, 10pt) -- +(0, 0.5cm) node(a)[]{};
    
    \node at ($(a) +(20pt, -3pt)$) [above left] {PG de base $a_1$};
    
    \draw[latex-] (mv2c2) ++(5pt, 10pt) -- +(0, 1cm) node(b) [] {};
    
    \node at ($(b) +(10pt, -3pt)$) [above] {PG de base $a_2$};
    
    \draw[latex-] (mv2c3) ++(5pt, 10pt) -- +(0, 0.5cm) node(c)[] {};
    
    \node at ($(c) +(-20pt, -3pt)$)[above right]{PG de base $a_m$};
\end{tikzoverlay}
%\chapter{Teorema de Laplace}

\section{Introdução}

Podemos encontrar o determinante de uma matriz quadrada de \textbf{qualquer} grau através desse teorema. Primeiro vamos escolher uma \textbf{coluna} ou \textbf{fileira} dessa matriz, uma boa prática é escolher baseando-se na que possuir \textbf{mais zeros} (veremos a razão para isso mais a frente).

\Example

$$
\text{Dada a matriz }
A=\begin{bmatrix*}[c]
    a_{11} & a_{12} & \cdots & a_{1n} \\
    a_{21} & a_{22} & \cdots & a_{2n} \\
    \vdots  & \vdots  & \ddots & \vdots  \\
    a_{m1} & a_{m2} & \cdots & a_{mn}
\end{bmatrix*} \\
$$
seu determinante pode ser obtido pela fórmula:
$$\det A=a_{k1}\cdot A_{k1}+a_{k2}\cdot A_{k2}+\dots +a_{kn}\cdot A_{kn}$$

\centerline{OU}

$$\det A=a_{1k}\cdot A_{1k}+a_{2k}\cdot A_{2k}+\dots +a_{nk}\cdot A_{nk}$$

onde $k$ é uma linha ou coluna qualquer da matriz $A$.

\section{Passo a passo}

\begin{enumerate}
    \item Usando a coluna/fileira escolhida como referência (no exemplo vamos usar a 1\textsuperscript{a} coluna), multiplicar os elementos dessa fileira/coluna pelos cofatores da sua posição.
    \item Em seguida devemos somá-los ou subtraí-los dependendo da soma $i+j$ , caso seja par, somamos, caso seja ímpar devemos subtrair. (Isso vem da definição do cofator).
\end{enumerate}

\Example

$$
\text{Dada a matriz }A=\begin{bmatrix*}[r]
    4 & 3 & -5 & 4\\ 2 & 1 & 7 & 14 \\ 1 & -9 & 6 & 9 \\ 7 & 2 & -12 & -1
\end{bmatrix*} \text{ tomamos sua 1\textsuperscript{a} coluna: }\begin{bmatrix*}[r]
    4 \\ 2 \\ 1 \\ 7
\end{bmatrix*}
$$

Pegamos elemento a elemento e \textbf{multiplicamos} pelo \textit{cofator} de sua posição.

O primeiro elemento da coluna é o $4$, sua posição é $11$, então fazemos:
$$4\cdot A_{11}$$

Repetimos isso para todos os elementos dessa fileira:
\vspace{-2em}
\begin{multicols}{3}
    $$2\cdot A_{11}$$ \\ $$1\cdot A_{31}$$ \\ $$7 \cdot A_{41}$$
\end{multicols}

Agora devemos determinar o \textbf{sinal} de cada um desses produtos. A regra diz, se a soma dos dígitos da posição for \textbf{par}, o sinal vai ser \textbf{positivo} e se for \textbf{ímpar} o sinal deverá ser \textbf{negativo}.

Para o elemento $a_{11}$ ($4$) ficamos com $+~4\cdot A_{11}$, pois $1+1=2$ que é \textbf{par}.

Somando todos os elementos conseguimos o determinante:

$$
\det A= +"4\cdot A_{11} - 2\cdot A_{11} + 1\cdot A_{31} - 7 \cdot A_{41}
$$

Para encontrar os cofatores podemos aplicar o método \textbf{novamente} (isso se chama \textbf{recursão}).

Como cada cofator é multiplicado por um elemento da matriz, no caso de algum desses elementos ser $0$ não precisaremos calcular esse cofator. Por isso é boa prática selecionar a coluna/fileira com a \textbf{maior quantidade de zeros}.
\part{Geometria}
\part{Funções trigonométricas}

\end{document}
 
