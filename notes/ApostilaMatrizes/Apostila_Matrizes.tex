\documentclass{IMTbook}

%\usepackage{fontspec} % specify font

\usepackage{CJKutf8} %japanese char

\usepackage{caption, multicol, empheq, booktabs}
\usepackage[enums,externalize,weirdsymbols]{IMTtikz}
\usetikzlibrary{matrix}

\externalizefigs

% define general tikz matrix style
\tikzset{matstyle/.style={matrix of math nodes,inner sep=3pt, left delimiter={[}, right delimiter={]}}}


%define new env for tikz matrix
\NewEnvironx{tikzmatrix}[1][1=]{\begin{center}
		\begin{tikzpicture}[node distance=0pt, every left delimiter/.style={xshift=.75em},
		every right delimiter/.style={xshift=-.75em}, ampersand replacement=\&, #1]}{
		\end{tikzpicture}
	\end{center}}

\newcommand{\myscale}{0.8}
\newcommand{\myunit}{\myscale*1 cm}

\tikzset{node style sp/.style={draw,circle,minimum size=\myunit}, 
	node style ge/.style={circle,minimum size=\myunit},
	arrow style mul/.style={draw,sloped,midway,fill=white},
	arrow style plus/.style={midway,sloped,fill=white}}

\title{Matrizes, Determinantes e Sistemas lineares}
\subtitle{uma apostila para o ensino médio}
\author{Isabella B.}

\begin{document}

\maketitle

\tableofcontents

%begin content inputting here

\part{Matrizes}

\chapter{Introdução}

\section{Um pouco de contexto}

Historicamente, as matrizes foram utilizadas para a resolução de sistemas lineares (a \autoref{p:3} é inteiramente dedicada a este tópico) que são, basicamente, conjuntos de equações com uma ou mais incógnitas.

Eram conhecidas como \textit{tabelas} (do francês \textit{tableau}), nome (aparentemente) dado por \textit{Cauchy}, em 1826. O nome \textit{matriz} (derivado do latim \textit{mater} - \textit{mãe}, que também tem a conotação de \textit{útero}) surgiu depois, em 1850, quando o matemático inglês \textit{James Joseph Sylvester} veio a nomeá-las com a ideia de que matrizes seriam \textit{úteros} de determinantes (ele estava se referindo aos \textit{cofatores}, que serão discutidos no \autoref{chap:cof} da \autoref{p:2}), pois "dariam luz"~à vários desses.

Quem primeiro deu vida às matrizes como entidades matemáticas independentes foi \textit{Arthur Cayley}, que definiu operações básicas com matrizes (que serão discutidas nos Capítulos \ref{chap:opermat} e \ref{chap:multmat}). Antes de \textit{Cayley} as matrizes eram meros ingredientes dos determinantes (que serão discutidos na \autoref{p:2}), que eram o tópico de estudo até então.

No início do século passado, as matrizes se estabeleceram como ferramentas fundamentais para o estudo da álgebra linear (que é um ramo da matemática que estuda os sistemas de equações lineares).

\section{Definição}

Uma \textit{matriz} consiste em uma estrutura organizada em \textbf{linhas} e \textbf{colunas}, composta de elementos que podem ser \textbf{números}, \textbf{símbolos} ou \textbf{expressões}. Representamos o \textbf{tamanho} da matriz por $m \times n$, onde $m$ se refere ao número de \textbf{linhas} e $n$ ao número de \textbf{colunas}.

\begin{multicols}{2}
    \noindent Podemos ver na matriz ao lado o uso de \textbf{índices} para indicar a \textit{posição} de seus elementos. 
    
    \noindent O índice $mn$ de um elemento $a$ da matriz $A$ indica que o elemento $a$ se encontra na linha $m$ e na coluna $n$. Esse índice abstrato pode ser dado por quaisquer letras de escolha, um caso comum é usarmos as letras $i$ e $j$ para indicar linha e coluna respectivamente.
    
    \columnbreak
        
    \null
    
	\begin{tikzmatrix}[]
		
		\matrix(M)[matstyle]{
			a_{11} \& a_{12} \\
			a_{21} \& a_{22} \\
		};
		
		\node [left=of M] {$A_{2 \times 2} =$};
		
		\draw[thick, draw=greenAr!80, <-] (M-1-1.north) ++(0, 5pt) -- node[right] {\textit{1\textsuperscript{a} coluna}} +(0, 20pt);
		
		\draw[thick, draw=redAr!80, <-] (M-1-2.east) ++(5pt, 0) -- node[below, near end] {\textit{1\textsuperscript{a} linha}} +(20pt, 0);
		
	\end{tikzmatrix}
    \centerline{\footnotesize{representação de uma matriz $2 \times 2$}}
    
    \vskip1em\null

\end{multicols}

\paragraph{Nota:} Geralmente usamos alguma \textbf{letra maiúscula} do nosso alfabeto para representar uma matriz $\left( A,B,C,\dots Z \right)$.

\subsection{Representação}
Uma matriz qualquer pode ser representada por:

$$
A_{m \times n} =
\begin{bmatrix}
    a_{11} & a_{12} & \cdots & a_{1n} \\
    a_{21} & a_{22} & \cdots & a_{2n} \\
    \vdots  & \vdots  & \ddots & \vdots  \\
    a_{m1} & a_{m2} & \cdots & a_{mn}
\end{bmatrix}
$$

Repare que para a indicação de matriz podemos usar tanto:
$$
\text{(I) parênteses: } A_{m \times n}=
\begin{pmatrix}
    a_{11} & \cdots & a_{1n} \\
    \vdots & \ddots & \vdots \\
    a_{n1} & \cdots & a_{mn}
\end{pmatrix}
$$

quanto:

$$
\text{(II) colchetes: }A_{m \times n}= \begin{bmatrix}
    a_{11} & \cdots & a_{1n} \\
    \vdots & \ddots & \vdots \\
    a_{n1} & \cdots & a_{mn}
\end{bmatrix}
$$
\chapter{Matrizes Notáveis}
Temos alguns tipos de matrizes com propriedades especiais, as quais vamos destacar nesta seção.

\section{Matriz Quadrada}

Denominamos \textit{quadrada} uma matriz onde o \textbf{número de linhas} é \textbf{igual} ao \textbf{número de colunas}. Nesse caso, ao invés de denotar seu tamanho pelos comprimentos $m \times n$ usamos somente uma de suas dimensões (dizemos então que a matriz é \textit{quadrada de ordem} $m$).

\begin{minipage}{\textwidth}
    $$
    B=
    \begin{bmatrix*}[r]
        a & b\\
        c & d
    \end{bmatrix*}$$
    \centerline{\footnotesize{Uma matriz $2 \times 2$ é quadrada, dizemos então que ela é \textit{quadrada de ordem} $2$.}}
\end{minipage}

Em toda matriz quadrada teremos duas \textbf{diagonais especiais}, chamadas \textit{principal} e \textit{secundária}.

\subsection{Diagonal principal}
A \textit{diagonal principal} será formada pelos elementos $a_{mn}$ onde ${m=n}$.

\begin{tikzmatrix}
        
    \matrix (M) [matstyle]{
    c_{11} \& c_{12} \& c_{13} \\
    c_{21} \& c_{22} \& c_{23} \\
    c_{31} \& c_{32} \& c_{33} \\
    };
    
    \node [left=of M] {$C=$};
    \scoped [on background layer]
    \fill[blue!30,rounded corners] (M-1-1.east) to (M-3-3.north) to (M-3-3.east) to (M-3-3.south) to (M-1-1.west) to (M-1-1.north) -- cycle;
        
\end{tikzmatrix}

\subsection{Diagonal secundária}
A \textit{secundária} será formada pelos elementos $a_{mn}$ tais que ${m+n=\mathcal{O}+1}$ onde $\mathcal{O}$ é a ordem da matriz quadrada.

\begin{tikzmatrix}
    
    \matrix (M) [matstyle]{
    c_{11} \& c_{12} \& c_{13} \\
    c_{21} \& c_{22} \& c_{23} \\
    c_{31} \& c_{32} \& c_{33} \\
    };
    
    \node [left=of M] {$C=$};
    \scoped [on background layer]
    \fill[blue!30,rounded corners] (M-1-3.west) to (M-3-1.north) to (M-3-1.west) to (M-3-1.south) to (M-1-3.east) to (M-1-3.north) -- cycle;
        
\end{tikzmatrix}

\centerline{\footnotesize{Podemos ver a diagonal secundária grifada, onde os elementos obedecem a relação $m+n=3+1 \left(\mathcal{O}=3\right)$}}

\section{Matriz identidade}
As matrizes \textit{identidade} são outro tipo especial que consiste em uma matriz quadrada que possui $1$’s em sua diagonal principal e $0$’s em todas as outras posições. Denominamos \textit{matriz identidade de ordem} $n$ (denotada por $I_n$) uma matriz quadrada dessa ordem que satisfaz essas condições.

Veremos mais a frente que essa matriz será equivalente ao número $1$ na operação de multiplicação de matrizes.

$$
I_2=
\begin{bmatrix*}[r]
    1 & 0 \\
    0 & 1
\end{bmatrix*}
$$
\centerline{\footnotesize{A matriz acima é uma \textit{matriz identidade de ordem} $2$}}

\section{Matriz nula}
Temos também as matrizes denominadas \textit{nulas}, as quais possuem todos os seus elementos igualando $0$ (nulos).

Uma matriz $m\times n$ que satisfaça essa condição é denotada por $0_{m \times n}$.
Caso essa matriz seja quadrada podemos denotá-la por $0_n$, onde $n$ é a ordem da matriz (a qual vamos chamar de \textit{matriz nula de ordem} $n$).

$$
0_{3 \times 2}=
\begin{bmatrix*}[r]
0 & 0 \\
0 & 0 \\
0 & 0
\end{bmatrix*}
$$
\centerline{\footnotesize{Acima temos uma \textit{matriz nula} $3 \times 2$}}

\section{Matriz linha/coluna}

Podemos ter também matrizes \textit{linha} ou \textit{coluna}, as quais são matrizes que se resumem a uma \textbf{linha} ou \textbf{coluna}, respectivamente.

\begin{alignat*}{3}
    A=\begin{bmatrix*}[r] a & b \end{bmatrix*} & \hspace{50pt} & B=\begin{bmatrix}a \\ b\end{bmatrix}
\end{alignat*}

\begin{center}
    \footnotesize{A matriz $A$ é uma \textit{matriz linha}, pois todos os seus elementos se\\ encontram em uma única linha, já a matriz $B$ é uma \textit{matriz coluna}}
\end{center}

\section{Matriz triangular}

Dizemos que uma matriz é \textit{triangular} se esta, além de se quadrada, tiver todos os elementos acima ou abaixo da diagonal principal \textbf{nulos}.

\subsection{Superior}
Uma matriz triangular é dita \textit{superior} se a parte \textbf{abaixo} da diagonal principal for nula.

$$
A_3=\begin{bmatrix*}[c]
a_{11} & a_{12} & a_{13}\\
0 & a_{22} & a_{23}\\
0 & 0 & a_{33}
\end{bmatrix*}
$$

\begin{center}
    \footnotesize{A matriz acima é triangular superior de ordem 3, pois todos os elementos abaixo da diagonal principal são nulos}
\end{center}

\subsection{Inferior}
Uma matriz triangular é dita \textit{inferior} se a parte \textbf{acima} da diagonal principal for nula.


$$
A_3=\begin{bmatrix*}[c]
a_{11} & 0 & 0\\
a_{21} & a_{22} & 0\\
a_{31} & a_{32} & a_{33}
\end{bmatrix*}
$$

\begin{center}
    \footnotesize{A matriz acima é triangular superior de ordem 3, pois todos os elementos abaixo da diagonal principal são nulos}
\end{center}

\section{Matriz diagonal}
Chamamos de \textit{diagonal} uma matriz quadrada onde todos os elementos que não pertencem à diagonal principal são nulos.

$$
A_4=\begin{bmatrix*}[c]
a_{11} & 0 & 0 & 0\\
0 & a_{22} & 0 & 0\\
0 & 0 & a_{33} & 0\\
0 & 0 & 0 & a_{44}
\end{bmatrix*}
$$

\begin{center}
    \footnotesize{A matriz acima é diagonal de ordem 4}
\end{center}

\paragraph{Nota:}
Repare que se uma matriz triangular for \textbf{simultaneamente} superior e inferior, esta será uma matriz chamada \textit{diagonal}.

\section{Matriz de Vandermonde}

Vamos analisar um pequeno exemplo:

$$
\begin{bmatrix*}[r]
1 & 1 & 1\\
3 & 5 & 2\\
9 & 25 & 4
\end{bmatrix*}
$$

Repare como a matriz acima pode ser escrita como:

$$
\begin{bmatrix*}[r]
3^0 & 5^0 & 2^0\\
3^1 & 5^1 & 2^1\\
3^2 & 5^2 & 2^2
\end{bmatrix*}
$$

As matrizes que têm essa propriedade são chamadas \textit{matrizes de Vandermonde}.

\paragraph{Definição:}
A matriz de \textit{Vandermonde} é aquela onde cada \textbf{linha} ou \textbf{coluna} representa um \textbf{termo} de uma \textbf{progressão geométrica} de base $a_n$.

\vspace{-2em}

\begin{multicols}{2}
\null
\noindent A matriz de \textit{Vandermonde} ao lado possui PG's em suas \textbf{linhas}. Podemos ver que os expoentes começam em $0$ e vão até $n-1$, aumentando para a direita.

\vskip0.2em\null

\columnbreak

$$
V_{m \times n} =
\begin{bmatrix}
    a_1^0 & a_1^1 & \cdots & a_1^{n-1}\tikzmark{mv1r1} \\[0.5em]
    a_2^0 & a_2^1 & \cdots & a_2^{n-1}\tikzmark{mv1r2} \\
    \vdots  & \vdots  & \ddots & \vdots  \\
    a_m^0 & a_m^1 & \cdots & a_m^{n-1}\tikzmark{mv1r3}
\end{bmatrix}
$$
\end{multicols}

\begin{multicols}{2}

\null

\noindent A matriz de \textit{Vandermonde} ao lado possui PG's em suas \textbf{colunas}. Podemos ver que os expoentes começam em $0$ e vão até $n-1$, aumentando para baixo.

\vskip1em\null

\columnbreak

\null\vskip1em

$$
V_{n \times m} =
\begin{bmatrix}
    \tikzmark{mv2c1}a_1^0 & \tikzmark{mv2c2}a_2^0 & \cdots & \tikzmark{mv2c3}a_m^0 \\[0.5em]
    a_1^1 & a_2^1 & \cdots & a_m^1 \\
    \vdots  & \vdots  & \ddots & \vdots  \\
    a_1^{n-1} & a_2^{n-1} & \cdots & a_m^{n-1}
\end{bmatrix}
$$

\null

\end{multicols}
\begin{tikzoverlay}[]
    \draw[latex-] (mv1r1) ++(5pt,2pt) -- +(0.5cm, 0) node[anchor=west] {PG de base $a_1$};
    \draw[latex-] (mv1r2) ++(5pt,3pt) -- +(0.5cm, 0) node[anchor=west] {PG de base $a_2$};
    \draw[latex-] (mv1r3) ++(5pt,4pt) -- +(0.5cm, 0) node[anchor=west] {PG de base $a_m$};
    
    \draw[latex-] (mv2c1) ++(5pt, 10pt) -- +(0, 0.5cm) node(a)[]{};
    
    \node at ($(a) +(20pt, -3pt)$) [above left] {PG de base $a_1$};
    
    \draw[latex-] (mv2c2) ++(5pt, 10pt) -- +(0, 1cm) node(b) [] {};
    
    \node at ($(b) +(10pt, -3pt)$) [above] {PG de base $a_2$};
    
    \draw[latex-] (mv2c3) ++(5pt, 10pt) -- +(0, 0.5cm) node(c)[] {};
    
    \node at ($(c) +(-20pt, -3pt)$)[above right]{PG de base $a_m$};
\end{tikzoverlay}
\chapter{Soluções de um sistema linear}

\section{O que são?}

As \textbf{soluções} para um dado sistema linear devem ser escritas organizadamente em uma \textbf{ênupla} (coordenada com $n$ posições, a palavra vem de \textit{n-upla}) \textbf{ordenada} (pois tem uma ordem específica) \textbf{de reais}. Uma solução de um sistema linear consiste em uma série de valores que deverá resolver o sistema de igualdades proposto.

\Example

No sistema

$$
S_1\begin{cases}
2x+y=1\\
2x-y=3
\end{cases}
$$

temos como solução a ênupla $(1, -1)$, pois se substituirmos $x$ e $y$ por esses números respectivamente, teremos as igualdades

$$
S_1\begin{cases}
2\cdot 1 + (-1)=1\\
2\cdot 1-(-1)= 3
\end{cases}
$$

Repare que na ênupla o primeiro número substitui a incógnita $x$ e o segundo, $y$. A posição de cada valor na ênupla nos dirá qual incógnita cada valor deverá substituir (daí vem o termo \textit{ordenada}).
\chapter{Teorema de Cramer}
\label{chap:cram}

\section{O que é?}

Esse teorema nos diz que, dado um sistema linear $S$ que tenha o \textbf{mesmo número} de incógnitas e expressões, se o colocarmos na forma matricial poderemos tomar seu determinante $D$, e, caso esse seja diferente de $0$, poderemos \textbf{determinar} os valores da \textbf{única solução possível} desse sistema.

\section{Passo a passo}

$$
\text{Dado o sistema }S_1\begin{cases}
2x-3y=0\\
x+2y=2
\end{cases}
$$

\begin{enumerate}
    \item Primeiro devemos checar se o número de \textbf{incógnitas} é \textbf{igual} ao número de \textbf{expressões}.
    
    No sistema acima isso é verdadeiro, pois temos $x$ e $y$ (duas incógnitas) e temos duas expressões (duas linhas). \checkmark
    
    \paragraph{Nota:}
    Caso o sistema tenha mais expressões do que incógnitas podemos ignorar algumas das expressões a fim de realizar os cálculos necessários com esse sistema.
    
    \item Agora podemos escrever o sistema em \textit{forma matricial}, para a realização desse teorema basta escrever a \textbf{primeira parte}, formada pelos \textbf{coeficientes}. Chamaremos essa matriz de $A$.
    $$
    A=\begin{bmatrix*}[r]
    2 & -3\\
    1 & 2
    \end{bmatrix*}
    $$

    \item Devemos, então, tomar o \textbf{determinante} desse sistema. 
    
    Chamaremos esse determinante de $D$.
    
    $$
    D=\det A=2\cdot 2-(-3)\cdot 1\Rightarrow D=4+3=7
    $$
    
    \item Se o determinante $D$ for \textbf{diferente} de zero, podemos prosseguir com o teorema.
    
    No sistema que estamos utilizando isso se verifica. \checkmark
\end{enumerate}

\section{Encontrando a única solução possível}

\begin{enumerate}
    \item Para encontrar o \textbf{valor} de uma incógnita $i$ qualquer (valor esse que \textbf{resolverá} o sistema) devemos \textbf{repetir} a matriz dos coeficientes ($A$) e trocar a coluna representada por essa incógnita de escolha por uma coluna feita \textbf{a partir} dos termos independentes. O determinante da nova matriz será chamado $D_i$.
    
    Continuando com o mesmo exemplo, vamos resolver para $x$:
    
    $$
    D_x=\begin{vmatrix*}[r]
    \color{green}0 & -3\\ \color{green}2 & 2
    \end{vmatrix*} \Rightarrow \\
    D_x=0 \cdot 2 -(-3)\cdot 2 = 6
    $$
    
    Repare que substituímos os termos da coluna que representa os coeficientes de $x$ por uma coluna formada pelos termos independentes (em \textcolor{green}{verde}), que devem estar na \textbf{mesma ordem} do sistema.
    
    \item O valor solução $\alpha_i$ da incógnita escolhida ($i$) será encontrado pela equação:
    
    $$
    \alpha_i=\frac{D_i}{D}
    $$
    
    Seguindo com o exemplo, fazemos:
    
    $$
    \text{Substituíndo } \alpha_x = \frac{D_x}{D} \text{ pelos valores que possuímos, encontramos } \alpha_x=\frac{6}{7}
    $$
    
    \item Agora devemos repetir o processo para as outras incógnitas.
    
    Fazendo para $y$ temos:
    
    \begin{gather*}
        D_y=\begin{vmatrix}
        2 & 0\\ 1 & 2
        \end{vmatrix} \Rightarrow D_y= 2\cdot 2-0\cdot 2=4\\
        \alpha_y=\frac{D_y}{D}\Rightarrow\alpha_y=\frac{4}{7}
    \end{gather*}
    
    \item Por fim, escrevemos a solução encontrada em forma de ênupla.
    
    $$
    \left(\sfrac{6}{7},~\sfrac{4}{7}\right)
    $$
    
    \item \textbf{(Extra)} Caso ache necessário, basta substituir os valores da ênupla nas expressões do sistema para \textbf{testar} a solução.
    
    $$
    S_1\begin{cases}
    2\cdot \dfrac{6}{7}-3\cdot\dfrac{4}{7}=0 \\[1em]
    \dfrac{6}{7}+2\cdot\dfrac{4}{7}=2
    \end{cases}\Rightarrow 
    S_1\begin{cases}
    \dfrac{12-12}{7}=0\\[1em]
    \dfrac{6+8}{7}=2
    \end{cases}\Rightarrow 
    S_1\begin{cases}
    \dfrac{0}{7}=0\\[1em]
    \dfrac{14}{7}=2
    \end{cases}
    $$

\end{enumerate}
\chapter{Escalonamento}
\label{chap:esc}

\section{Introdução}
Para resolver e diferenciar os sistemas lineares dispomos da técnica chamada \mbox{\textit{escalonamento}}. Essa técnica consiste em um \textbf{algoritmo} (outra palavra para \textit{receita}) onde vamos, através de soma e multiplicação, \textbf{zerar} coeficientes das expressões de um sistema linear, um por um.

\section{Algoritmo (ou receita)}
\begin{enumerate}
    \item Devemos escolher alguma das expressões de um sistema para usar como \textbf{referência}. É boa prática escolher uma expressão onde os coeficientes são \textbf{menores}.
    \Example
    $$
    S_1\begin{cases}
    \color{red}x+y-z=2\\
    3x-4y+2z=1\\
    5x+2y-z=7
    \end{cases}
    $$
    
    Vamos usar a primeira expressão (em \textcolor{red}{vermelho}). Vamos chamá-la de \ref{eq:a}.
    
    \item Agora vamos \textbf{escolher} alguma incógnita com o objetivo de \textbf{zerar} seu \textbf{coeficiente}. Nessa etapa podemos organizar as expressões colocando a escolhida \textbf{em cima} das outras.
    
    \paragraph{Continuação:}
    
    \begin{empheq}[left=S_1\empheqlbrace]{align*}
    &x+y-z=2 \label{eq:a}\tag{$S_a$}\\
    &3x-4y+2z=1 \label{eq:b}\tag{$S_b$}\\
    &5x+2y-z=7 \label{eq:c}\tag{$S_c$}
    \end{empheq}
    
    A expressão referência se encontra \textbf{acima} das que pretendemos mexer (\ref{eq:a} acima de \ref{eq:b} e \ref{eq:c}).
    
    \item Através de equação podemos encontrar valores que, \textbf{multiplicados} à primeira equação, vão \textbf{zerar} o \textbf{coeficiente} da incógnita de escolha nas outras.
    
    A expressão \ref{eq:a} multiplicada por uma constante $k_1$ e somada à expressão \ref{eq:b} deverá zerar o coeficiente que multiplica $x$:
    $$
    S_a\times k_1 + S_b=0
    $$
    
    Porém, como é \textbf{absurdo} fazer contas para $n$ valores ao mesmo tempo, vamos focar no que nos interessa, o valor de $x$ nas expressões \ref{eq:a} e \ref{eq:b}. Para isso basta \textbf{substituir as expressões} \ref{eq:a} e \ref{eq:b} por seus \textbf{respectivos coeficientes} de $x$:
    $$
    1\cdot k_1 + 3=0
    $$
    Resolvendo, temos:
    $$
    1\cdot k_1 + 3=0 \Rightarrow k=-3
    $$
    
    Fazendo o mesmo para \ref{eq:a} e \ref{eq:c}, temos:
    $$
    S_a\times k_2+S_c \Rightarrow 1\cdot k_2+5=0 \Rightarrow k=-5
    $$
    
    \paragraph{Nota:}
    O valor $k_1$ (que zera o coeficiente de $x$ em \ref{eq:b}) \textbf{não necessariamente} é o mesmo que $k_2$ (que zera o coeficiente de $x$ em \ref{eq:c}).
    
    \item Agora basta \textbf{multiplicar} a expressão de \textbf{referência} pelo valor encontrado e \textbf{somar} às outras expressões. 
    
    Como os valores que zeram cada uma são \textbf{diferentes}, executamos esse passo \textbf{independentemente}.
    
    Sabemos que $k_1=-3$ zera a expressão \ref{eq:b}, então faremos:
    \begin{gather*}
        (x+y-z=2)\times (-3) +S_b\Rightarrow\\
        (-3x-3y+3z=-6) + S_b\Rightarrow\\
        \begin{aligned}
        -3x-3y+3z&=-6\\
        \mathbf{+}~~3x-4y+2z&=2\\\midrule
         0x-7y+5z&=-4
        \end{aligned}
    \end{gather*}
    
    Colocando a nova expressão no lugar de \ref{eq:b}, ficamos com:
    $$
    S_1\begin{cases}
    x+y-z=2\\
    0x-7y+5z=-4\\
    5x+2y-z=7
    \end{cases}
    $$
    
    \paragraph{Nota:} 
    a expressão de referência (\ref{eq:a}) \textbf{não se altera}, e nem a expressão \ref{eq:c}. No escalonamento só alteramos \textbf{uma expressão de cada vez}.
    
    Vamos fazer a mesma coisa para \ref{eq:a} e \ref{eq:c}. Sabemos que $k=-5$ zera a expressão \ref{eq:c}, então fazemos:
    \begin{gather*}
        (x+y-z=2) \times (-5) + S_c\Rightarrow\\
        (-5x-5y+5z=-10) + S_c\Rightarrow\\
        \begin{aligned}
        -5x-5y+5z&=-10\\
        \mathbf{+}~~5x+2y-z&=7\\\midrule
        0x-3y+4z&=-3
        \end{aligned}
    \end{gather*}
    
    Colocando a nova expressão no lugar de \ref{eq:c}, temos:
    $$
    S_1\begin{cases}
    x+y-z=2\\
    0x-7y+5z=-4\\
    0x-3y+4z=-3
    \end{cases}
    $$
    
    \item Por fim, devemos repetir os passos $1$ a $4$ escolhendo \textbf{outra incógnita} para "zerar" e outra expressão como referência, \textbf{sem mexer} na referência anterior.
    
    Usando \ref{eq:b} como referência para zerar o coeficiente $y$ em \ref{eq:c} temos:
    $$
    S_b\times k_3 +S_c=0\Rightarrow -7\cdot k + (-3) =0\Rightarrow k_3= \sfrac{3}{7}
    $$
    
    Como não devemos alterar as referências anteriores (expressão \ref{eq:a}), basta encontrar o valor $k_3$ para zerar o coeficiente $y$ em \ref{eq:c}.
    
    Fazendo a conta ficamos com:
    \begin{gather*}
        (-7+5z=-4) \times \left(\sfrac{-3}{7}\right)+S_c\Rightarrow\\
        \left(3y-\frac{15}{7}=-\frac{12}{7}\right) + S_c\Rightarrow\\
        \begin{aligned}
        3y-\frac{15}{7}&=-\frac{12}{7}\\
        \mathbf{+}~~-3y+4z&=-3\\\midrule
        0y+\frac{13}{7}z&=-\frac{33}{7}
        \end{aligned}
    \end{gather*}

    Colocando a nova expressão no lugar de \ref{eq:c} temos:
    $$
    S_1\begin{cases}
    x+y-z=2\\
    0x-7y+5z=-4\\
    0x+0y+\dfrac{13}{7}=-\dfrac{33}{7}
    \end{cases}
    $$
    
    Podemos ver que a cada vez que \textbf{repetimos} o algoritmo zeramos os coeficientes de \textbf{outra incógnita}.
    
    Agora que não há mais expressões para mexer, pois devemos ter pelo menos uma nova referência e uma outra expressão para mexer (e no momento só teríamos uma referência e mais nada), podemos, então, interromper o algoritmo.


\end{enumerate}
\part{Trigonometria na circunferência}

\chapter{Introdução}

\section{Um pouco de contexto}

Historicamente, as matrizes foram utilizadas para a resolução de sistemas lineares (a \autoref{p:3} é inteiramente dedicada a este tópico) que são, basicamente, conjuntos de equações com uma ou mais incógnitas.

Eram conhecidas como \textit{tabelas} (do francês \textit{tableau}), nome (aparentemente) dado por \textit{Cauchy}, em 1826. O nome \textit{matriz} (derivado do latim \textit{mater} - \textit{mãe}, que também tem a conotação de \textit{útero}) surgiu depois, em 1850, quando o matemático inglês \textit{James Joseph Sylvester} veio a nomeá-las com a ideia de que matrizes seriam \textit{úteros} de determinantes (ele estava se referindo aos \textit{cofatores}, que serão discutidos no \autoref{chap:cof} da \autoref{p:2}), pois "dariam luz"~à vários desses.

Quem primeiro deu vida às matrizes como entidades matemáticas independentes foi \textit{Arthur Cayley}, que definiu operações básicas com matrizes (que serão discutidas nos Capítulos \ref{chap:opermat} e \ref{chap:multmat}). Antes de \textit{Cayley} as matrizes eram meros ingredientes dos determinantes (que serão discutidos na \autoref{p:2}), que eram o tópico de estudo até então.

No início do século passado, as matrizes se estabeleceram como ferramentas fundamentais para o estudo da álgebra linear (que é um ramo da matemática que estuda os sistemas de equações lineares).

\section{Definição}

Uma \textit{matriz} consiste em uma estrutura organizada em \textbf{linhas} e \textbf{colunas}, composta de elementos que podem ser \textbf{números}, \textbf{símbolos} ou \textbf{expressões}. Representamos o \textbf{tamanho} da matriz por $m \times n$, onde $m$ se refere ao número de \textbf{linhas} e $n$ ao número de \textbf{colunas}.

\begin{multicols}{2}
    \noindent Podemos ver na matriz ao lado o uso de \textbf{índices} para indicar a \textit{posição} de seus elementos. 
    
    \noindent O índice $mn$ de um elemento $a$ da matriz $A$ indica que o elemento $a$ se encontra na linha $m$ e na coluna $n$. Esse índice abstrato pode ser dado por quaisquer letras de escolha, um caso comum é usarmos as letras $i$ e $j$ para indicar linha e coluna respectivamente.
    
    \columnbreak
        
    \null
    
	\begin{tikzmatrix}[]
		
		\matrix(M)[matstyle]{
			a_{11} \& a_{12} \\
			a_{21} \& a_{22} \\
		};
		
		\node [left=of M] {$A_{2 \times 2} =$};
		
		\draw[thick, draw=greenAr!80, <-] (M-1-1.north) ++(0, 5pt) -- node[right] {\textit{1\textsuperscript{a} coluna}} +(0, 20pt);
		
		\draw[thick, draw=redAr!80, <-] (M-1-2.east) ++(5pt, 0) -- node[below, near end] {\textit{1\textsuperscript{a} linha}} +(20pt, 0);
		
	\end{tikzmatrix}
    \centerline{\footnotesize{representação de uma matriz $2 \times 2$}}
    
    \vskip1em\null

\end{multicols}

\paragraph{Nota:} Geralmente usamos alguma \textbf{letra maiúscula} do nosso alfabeto para representar uma matriz $\left( A,B,C,\dots Z \right)$.

\subsection{Representação}
Uma matriz qualquer pode ser representada por:

$$
A_{m \times n} =
\begin{bmatrix}
    a_{11} & a_{12} & \cdots & a_{1n} \\
    a_{21} & a_{22} & \cdots & a_{2n} \\
    \vdots  & \vdots  & \ddots & \vdots  \\
    a_{m1} & a_{m2} & \cdots & a_{mn}
\end{bmatrix}
$$

Repare que para a indicação de matriz podemos usar tanto:
$$
\text{(I) parênteses: } A_{m \times n}=
\begin{pmatrix}
    a_{11} & \cdots & a_{1n} \\
    \vdots & \ddots & \vdots \\
    a_{n1} & \cdots & a_{mn}
\end{pmatrix}
$$

quanto:

$$
\text{(II) colchetes: }A_{m \times n}= \begin{bmatrix}
    a_{11} & \cdots & a_{1n} \\
    \vdots & \ddots & \vdots \\
    a_{n1} & \cdots & a_{mn}
\end{bmatrix}
$$
\chapter{Matrizes Notáveis}
Temos alguns tipos de matrizes com propriedades especiais, as quais vamos destacar nesta seção.

\section{Matriz Quadrada}

Denominamos \textit{quadrada} uma matriz onde o \textbf{número de linhas} é \textbf{igual} ao \textbf{número de colunas}. Nesse caso, ao invés de denotar seu tamanho pelos comprimentos $m \times n$ usamos somente uma de suas dimensões (dizemos então que a matriz é \textit{quadrada de ordem} $m$).

\begin{minipage}{\textwidth}
    $$
    B=
    \begin{bmatrix*}[r]
        a & b\\
        c & d
    \end{bmatrix*}$$
    \centerline{\footnotesize{Uma matriz $2 \times 2$ é quadrada, dizemos então que ela é \textit{quadrada de ordem} $2$.}}
\end{minipage}

Em toda matriz quadrada teremos duas \textbf{diagonais especiais}, chamadas \textit{principal} e \textit{secundária}.

\subsection{Diagonal principal}
A \textit{diagonal principal} será formada pelos elementos $a_{mn}$ onde ${m=n}$.

\begin{tikzmatrix}
        
    \matrix (M) [matstyle]{
    c_{11} \& c_{12} \& c_{13} \\
    c_{21} \& c_{22} \& c_{23} \\
    c_{31} \& c_{32} \& c_{33} \\
    };
    
    \node [left=of M] {$C=$};
    \scoped [on background layer]
    \fill[blue!30,rounded corners] (M-1-1.east) to (M-3-3.north) to (M-3-3.east) to (M-3-3.south) to (M-1-1.west) to (M-1-1.north) -- cycle;
        
\end{tikzmatrix}

\subsection{Diagonal secundária}
A \textit{secundária} será formada pelos elementos $a_{mn}$ tais que ${m+n=\mathcal{O}+1}$ onde $\mathcal{O}$ é a ordem da matriz quadrada.

\begin{tikzmatrix}
    
    \matrix (M) [matstyle]{
    c_{11} \& c_{12} \& c_{13} \\
    c_{21} \& c_{22} \& c_{23} \\
    c_{31} \& c_{32} \& c_{33} \\
    };
    
    \node [left=of M] {$C=$};
    \scoped [on background layer]
    \fill[blue!30,rounded corners] (M-1-3.west) to (M-3-1.north) to (M-3-1.west) to (M-3-1.south) to (M-1-3.east) to (M-1-3.north) -- cycle;
        
\end{tikzmatrix}

\centerline{\footnotesize{Podemos ver a diagonal secundária grifada, onde os elementos obedecem a relação $m+n=3+1 \left(\mathcal{O}=3\right)$}}

\section{Matriz identidade}
As matrizes \textit{identidade} são outro tipo especial que consiste em uma matriz quadrada que possui $1$’s em sua diagonal principal e $0$’s em todas as outras posições. Denominamos \textit{matriz identidade de ordem} $n$ (denotada por $I_n$) uma matriz quadrada dessa ordem que satisfaz essas condições.

Veremos mais a frente que essa matriz será equivalente ao número $1$ na operação de multiplicação de matrizes.

$$
I_2=
\begin{bmatrix*}[r]
    1 & 0 \\
    0 & 1
\end{bmatrix*}
$$
\centerline{\footnotesize{A matriz acima é uma \textit{matriz identidade de ordem} $2$}}

\section{Matriz nula}
Temos também as matrizes denominadas \textit{nulas}, as quais possuem todos os seus elementos igualando $0$ (nulos).

Uma matriz $m\times n$ que satisfaça essa condição é denotada por $0_{m \times n}$.
Caso essa matriz seja quadrada podemos denotá-la por $0_n$, onde $n$ é a ordem da matriz (a qual vamos chamar de \textit{matriz nula de ordem} $n$).

$$
0_{3 \times 2}=
\begin{bmatrix*}[r]
0 & 0 \\
0 & 0 \\
0 & 0
\end{bmatrix*}
$$
\centerline{\footnotesize{Acima temos uma \textit{matriz nula} $3 \times 2$}}

\section{Matriz linha/coluna}

Podemos ter também matrizes \textit{linha} ou \textit{coluna}, as quais são matrizes que se resumem a uma \textbf{linha} ou \textbf{coluna}, respectivamente.

\begin{alignat*}{3}
    A=\begin{bmatrix*}[r] a & b \end{bmatrix*} & \hspace{50pt} & B=\begin{bmatrix}a \\ b\end{bmatrix}
\end{alignat*}

\begin{center}
    \footnotesize{A matriz $A$ é uma \textit{matriz linha}, pois todos os seus elementos se\\ encontram em uma única linha, já a matriz $B$ é uma \textit{matriz coluna}}
\end{center}

\section{Matriz triangular}

Dizemos que uma matriz é \textit{triangular} se esta, além de se quadrada, tiver todos os elementos acima ou abaixo da diagonal principal \textbf{nulos}.

\subsection{Superior}
Uma matriz triangular é dita \textit{superior} se a parte \textbf{abaixo} da diagonal principal for nula.

$$
A_3=\begin{bmatrix*}[c]
a_{11} & a_{12} & a_{13}\\
0 & a_{22} & a_{23}\\
0 & 0 & a_{33}
\end{bmatrix*}
$$

\begin{center}
    \footnotesize{A matriz acima é triangular superior de ordem 3, pois todos os elementos abaixo da diagonal principal são nulos}
\end{center}

\subsection{Inferior}
Uma matriz triangular é dita \textit{inferior} se a parte \textbf{acima} da diagonal principal for nula.


$$
A_3=\begin{bmatrix*}[c]
a_{11} & 0 & 0\\
a_{21} & a_{22} & 0\\
a_{31} & a_{32} & a_{33}
\end{bmatrix*}
$$

\begin{center}
    \footnotesize{A matriz acima é triangular superior de ordem 3, pois todos os elementos abaixo da diagonal principal são nulos}
\end{center}

\section{Matriz diagonal}
Chamamos de \textit{diagonal} uma matriz quadrada onde todos os elementos que não pertencem à diagonal principal são nulos.

$$
A_4=\begin{bmatrix*}[c]
a_{11} & 0 & 0 & 0\\
0 & a_{22} & 0 & 0\\
0 & 0 & a_{33} & 0\\
0 & 0 & 0 & a_{44}
\end{bmatrix*}
$$

\begin{center}
    \footnotesize{A matriz acima é diagonal de ordem 4}
\end{center}

\paragraph{Nota:}
Repare que se uma matriz triangular for \textbf{simultaneamente} superior e inferior, esta será uma matriz chamada \textit{diagonal}.

\section{Matriz de Vandermonde}

Vamos analisar um pequeno exemplo:

$$
\begin{bmatrix*}[r]
1 & 1 & 1\\
3 & 5 & 2\\
9 & 25 & 4
\end{bmatrix*}
$$

Repare como a matriz acima pode ser escrita como:

$$
\begin{bmatrix*}[r]
3^0 & 5^0 & 2^0\\
3^1 & 5^1 & 2^1\\
3^2 & 5^2 & 2^2
\end{bmatrix*}
$$

As matrizes que têm essa propriedade são chamadas \textit{matrizes de Vandermonde}.

\paragraph{Definição:}
A matriz de \textit{Vandermonde} é aquela onde cada \textbf{linha} ou \textbf{coluna} representa um \textbf{termo} de uma \textbf{progressão geométrica} de base $a_n$.

\vspace{-2em}

\begin{multicols}{2}
\null
\noindent A matriz de \textit{Vandermonde} ao lado possui PG's em suas \textbf{linhas}. Podemos ver que os expoentes começam em $0$ e vão até $n-1$, aumentando para a direita.

\vskip0.2em\null

\columnbreak

$$
V_{m \times n} =
\begin{bmatrix}
    a_1^0 & a_1^1 & \cdots & a_1^{n-1}\tikzmark{mv1r1} \\[0.5em]
    a_2^0 & a_2^1 & \cdots & a_2^{n-1}\tikzmark{mv1r2} \\
    \vdots  & \vdots  & \ddots & \vdots  \\
    a_m^0 & a_m^1 & \cdots & a_m^{n-1}\tikzmark{mv1r3}
\end{bmatrix}
$$
\end{multicols}

\begin{multicols}{2}

\null

\noindent A matriz de \textit{Vandermonde} ao lado possui PG's em suas \textbf{colunas}. Podemos ver que os expoentes começam em $0$ e vão até $n-1$, aumentando para baixo.

\vskip1em\null

\columnbreak

\null\vskip1em

$$
V_{n \times m} =
\begin{bmatrix}
    \tikzmark{mv2c1}a_1^0 & \tikzmark{mv2c2}a_2^0 & \cdots & \tikzmark{mv2c3}a_m^0 \\[0.5em]
    a_1^1 & a_2^1 & \cdots & a_m^1 \\
    \vdots  & \vdots  & \ddots & \vdots  \\
    a_1^{n-1} & a_2^{n-1} & \cdots & a_m^{n-1}
\end{bmatrix}
$$

\null

\end{multicols}
\begin{tikzoverlay}[]
    \draw[latex-] (mv1r1) ++(5pt,2pt) -- +(0.5cm, 0) node[anchor=west] {PG de base $a_1$};
    \draw[latex-] (mv1r2) ++(5pt,3pt) -- +(0.5cm, 0) node[anchor=west] {PG de base $a_2$};
    \draw[latex-] (mv1r3) ++(5pt,4pt) -- +(0.5cm, 0) node[anchor=west] {PG de base $a_m$};
    
    \draw[latex-] (mv2c1) ++(5pt, 10pt) -- +(0, 0.5cm) node(a)[]{};
    
    \node at ($(a) +(20pt, -3pt)$) [above left] {PG de base $a_1$};
    
    \draw[latex-] (mv2c2) ++(5pt, 10pt) -- +(0, 1cm) node(b) [] {};
    
    \node at ($(b) +(10pt, -3pt)$) [above] {PG de base $a_2$};
    
    \draw[latex-] (mv2c3) ++(5pt, 10pt) -- +(0, 0.5cm) node(c)[] {};
    
    \node at ($(c) +(-20pt, -3pt)$)[above right]{PG de base $a_m$};
\end{tikzoverlay}
\part{Sistemas lineares}
\label{p:3}

\chapter{Introdução}

\section{Um pouco de contexto}

Historicamente, as matrizes foram utilizadas para a resolução de sistemas lineares (a \autoref{p:3} é inteiramente dedicada a este tópico) que são, basicamente, conjuntos de equações com uma ou mais incógnitas.

Eram conhecidas como \textit{tabelas} (do francês \textit{tableau}), nome (aparentemente) dado por \textit{Cauchy}, em 1826. O nome \textit{matriz} (derivado do latim \textit{mater} - \textit{mãe}, que também tem a conotação de \textit{útero}) surgiu depois, em 1850, quando o matemático inglês \textit{James Joseph Sylvester} veio a nomeá-las com a ideia de que matrizes seriam \textit{úteros} de determinantes (ele estava se referindo aos \textit{cofatores}, que serão discutidos no \autoref{chap:cof} da \autoref{p:2}), pois "dariam luz"~à vários desses.

Quem primeiro deu vida às matrizes como entidades matemáticas independentes foi \textit{Arthur Cayley}, que definiu operações básicas com matrizes (que serão discutidas nos Capítulos \ref{chap:opermat} e \ref{chap:multmat}). Antes de \textit{Cayley} as matrizes eram meros ingredientes dos determinantes (que serão discutidos na \autoref{p:2}), que eram o tópico de estudo até então.

No início do século passado, as matrizes se estabeleceram como ferramentas fundamentais para o estudo da álgebra linear (que é um ramo da matemática que estuda os sistemas de equações lineares).

\section{Definição}

Uma \textit{matriz} consiste em uma estrutura organizada em \textbf{linhas} e \textbf{colunas}, composta de elementos que podem ser \textbf{números}, \textbf{símbolos} ou \textbf{expressões}. Representamos o \textbf{tamanho} da matriz por $m \times n$, onde $m$ se refere ao número de \textbf{linhas} e $n$ ao número de \textbf{colunas}.

\begin{multicols}{2}
    \noindent Podemos ver na matriz ao lado o uso de \textbf{índices} para indicar a \textit{posição} de seus elementos. 
    
    \noindent O índice $mn$ de um elemento $a$ da matriz $A$ indica que o elemento $a$ se encontra na linha $m$ e na coluna $n$. Esse índice abstrato pode ser dado por quaisquer letras de escolha, um caso comum é usarmos as letras $i$ e $j$ para indicar linha e coluna respectivamente.
    
    \columnbreak
        
    \null
    
	\begin{tikzmatrix}[]
		
		\matrix(M)[matstyle]{
			a_{11} \& a_{12} \\
			a_{21} \& a_{22} \\
		};
		
		\node [left=of M] {$A_{2 \times 2} =$};
		
		\draw[thick, draw=greenAr!80, <-] (M-1-1.north) ++(0, 5pt) -- node[right] {\textit{1\textsuperscript{a} coluna}} +(0, 20pt);
		
		\draw[thick, draw=redAr!80, <-] (M-1-2.east) ++(5pt, 0) -- node[below, near end] {\textit{1\textsuperscript{a} linha}} +(20pt, 0);
		
	\end{tikzmatrix}
    \centerline{\footnotesize{representação de uma matriz $2 \times 2$}}
    
    \vskip1em\null

\end{multicols}

\paragraph{Nota:} Geralmente usamos alguma \textbf{letra maiúscula} do nosso alfabeto para representar uma matriz $\left( A,B,C,\dots Z \right)$.

\subsection{Representação}
Uma matriz qualquer pode ser representada por:

$$
A_{m \times n} =
\begin{bmatrix}
    a_{11} & a_{12} & \cdots & a_{1n} \\
    a_{21} & a_{22} & \cdots & a_{2n} \\
    \vdots  & \vdots  & \ddots & \vdots  \\
    a_{m1} & a_{m2} & \cdots & a_{mn}
\end{bmatrix}
$$

Repare que para a indicação de matriz podemos usar tanto:
$$
\text{(I) parênteses: } A_{m \times n}=
\begin{pmatrix}
    a_{11} & \cdots & a_{1n} \\
    \vdots & \ddots & \vdots \\
    a_{n1} & \cdots & a_{mn}
\end{pmatrix}
$$

quanto:

$$
\text{(II) colchetes: }A_{m \times n}= \begin{bmatrix}
    a_{11} & \cdots & a_{1n} \\
    \vdots & \ddots & \vdots \\
    a_{n1} & \cdots & a_{mn}
\end{bmatrix}
$$
\chapter{Matrizes Notáveis}
Temos alguns tipos de matrizes com propriedades especiais, as quais vamos destacar nesta seção.

\section{Matriz Quadrada}

Denominamos \textit{quadrada} uma matriz onde o \textbf{número de linhas} é \textbf{igual} ao \textbf{número de colunas}. Nesse caso, ao invés de denotar seu tamanho pelos comprimentos $m \times n$ usamos somente uma de suas dimensões (dizemos então que a matriz é \textit{quadrada de ordem} $m$).

\begin{minipage}{\textwidth}
    $$
    B=
    \begin{bmatrix*}[r]
        a & b\\
        c & d
    \end{bmatrix*}$$
    \centerline{\footnotesize{Uma matriz $2 \times 2$ é quadrada, dizemos então que ela é \textit{quadrada de ordem} $2$.}}
\end{minipage}

Em toda matriz quadrada teremos duas \textbf{diagonais especiais}, chamadas \textit{principal} e \textit{secundária}.

\subsection{Diagonal principal}
A \textit{diagonal principal} será formada pelos elementos $a_{mn}$ onde ${m=n}$.

\begin{tikzmatrix}
        
    \matrix (M) [matstyle]{
    c_{11} \& c_{12} \& c_{13} \\
    c_{21} \& c_{22} \& c_{23} \\
    c_{31} \& c_{32} \& c_{33} \\
    };
    
    \node [left=of M] {$C=$};
    \scoped [on background layer]
    \fill[blue!30,rounded corners] (M-1-1.east) to (M-3-3.north) to (M-3-3.east) to (M-3-3.south) to (M-1-1.west) to (M-1-1.north) -- cycle;
        
\end{tikzmatrix}

\subsection{Diagonal secundária}
A \textit{secundária} será formada pelos elementos $a_{mn}$ tais que ${m+n=\mathcal{O}+1}$ onde $\mathcal{O}$ é a ordem da matriz quadrada.

\begin{tikzmatrix}
    
    \matrix (M) [matstyle]{
    c_{11} \& c_{12} \& c_{13} \\
    c_{21} \& c_{22} \& c_{23} \\
    c_{31} \& c_{32} \& c_{33} \\
    };
    
    \node [left=of M] {$C=$};
    \scoped [on background layer]
    \fill[blue!30,rounded corners] (M-1-3.west) to (M-3-1.north) to (M-3-1.west) to (M-3-1.south) to (M-1-3.east) to (M-1-3.north) -- cycle;
        
\end{tikzmatrix}

\centerline{\footnotesize{Podemos ver a diagonal secundária grifada, onde os elementos obedecem a relação $m+n=3+1 \left(\mathcal{O}=3\right)$}}

\section{Matriz identidade}
As matrizes \textit{identidade} são outro tipo especial que consiste em uma matriz quadrada que possui $1$’s em sua diagonal principal e $0$’s em todas as outras posições. Denominamos \textit{matriz identidade de ordem} $n$ (denotada por $I_n$) uma matriz quadrada dessa ordem que satisfaz essas condições.

Veremos mais a frente que essa matriz será equivalente ao número $1$ na operação de multiplicação de matrizes.

$$
I_2=
\begin{bmatrix*}[r]
    1 & 0 \\
    0 & 1
\end{bmatrix*}
$$
\centerline{\footnotesize{A matriz acima é uma \textit{matriz identidade de ordem} $2$}}

\section{Matriz nula}
Temos também as matrizes denominadas \textit{nulas}, as quais possuem todos os seus elementos igualando $0$ (nulos).

Uma matriz $m\times n$ que satisfaça essa condição é denotada por $0_{m \times n}$.
Caso essa matriz seja quadrada podemos denotá-la por $0_n$, onde $n$ é a ordem da matriz (a qual vamos chamar de \textit{matriz nula de ordem} $n$).

$$
0_{3 \times 2}=
\begin{bmatrix*}[r]
0 & 0 \\
0 & 0 \\
0 & 0
\end{bmatrix*}
$$
\centerline{\footnotesize{Acima temos uma \textit{matriz nula} $3 \times 2$}}

\section{Matriz linha/coluna}

Podemos ter também matrizes \textit{linha} ou \textit{coluna}, as quais são matrizes que se resumem a uma \textbf{linha} ou \textbf{coluna}, respectivamente.

\begin{alignat*}{3}
    A=\begin{bmatrix*}[r] a & b \end{bmatrix*} & \hspace{50pt} & B=\begin{bmatrix}a \\ b\end{bmatrix}
\end{alignat*}

\begin{center}
    \footnotesize{A matriz $A$ é uma \textit{matriz linha}, pois todos os seus elementos se\\ encontram em uma única linha, já a matriz $B$ é uma \textit{matriz coluna}}
\end{center}

\section{Matriz triangular}

Dizemos que uma matriz é \textit{triangular} se esta, além de se quadrada, tiver todos os elementos acima ou abaixo da diagonal principal \textbf{nulos}.

\subsection{Superior}
Uma matriz triangular é dita \textit{superior} se a parte \textbf{abaixo} da diagonal principal for nula.

$$
A_3=\begin{bmatrix*}[c]
a_{11} & a_{12} & a_{13}\\
0 & a_{22} & a_{23}\\
0 & 0 & a_{33}
\end{bmatrix*}
$$

\begin{center}
    \footnotesize{A matriz acima é triangular superior de ordem 3, pois todos os elementos abaixo da diagonal principal são nulos}
\end{center}

\subsection{Inferior}
Uma matriz triangular é dita \textit{inferior} se a parte \textbf{acima} da diagonal principal for nula.


$$
A_3=\begin{bmatrix*}[c]
a_{11} & 0 & 0\\
a_{21} & a_{22} & 0\\
a_{31} & a_{32} & a_{33}
\end{bmatrix*}
$$

\begin{center}
    \footnotesize{A matriz acima é triangular superior de ordem 3, pois todos os elementos abaixo da diagonal principal são nulos}
\end{center}

\section{Matriz diagonal}
Chamamos de \textit{diagonal} uma matriz quadrada onde todos os elementos que não pertencem à diagonal principal são nulos.

$$
A_4=\begin{bmatrix*}[c]
a_{11} & 0 & 0 & 0\\
0 & a_{22} & 0 & 0\\
0 & 0 & a_{33} & 0\\
0 & 0 & 0 & a_{44}
\end{bmatrix*}
$$

\begin{center}
    \footnotesize{A matriz acima é diagonal de ordem 4}
\end{center}

\paragraph{Nota:}
Repare que se uma matriz triangular for \textbf{simultaneamente} superior e inferior, esta será uma matriz chamada \textit{diagonal}.

\section{Matriz de Vandermonde}

Vamos analisar um pequeno exemplo:

$$
\begin{bmatrix*}[r]
1 & 1 & 1\\
3 & 5 & 2\\
9 & 25 & 4
\end{bmatrix*}
$$

Repare como a matriz acima pode ser escrita como:

$$
\begin{bmatrix*}[r]
3^0 & 5^0 & 2^0\\
3^1 & 5^1 & 2^1\\
3^2 & 5^2 & 2^2
\end{bmatrix*}
$$

As matrizes que têm essa propriedade são chamadas \textit{matrizes de Vandermonde}.

\paragraph{Definição:}
A matriz de \textit{Vandermonde} é aquela onde cada \textbf{linha} ou \textbf{coluna} representa um \textbf{termo} de uma \textbf{progressão geométrica} de base $a_n$.

\vspace{-2em}

\begin{multicols}{2}
\null
\noindent A matriz de \textit{Vandermonde} ao lado possui PG's em suas \textbf{linhas}. Podemos ver que os expoentes começam em $0$ e vão até $n-1$, aumentando para a direita.

\vskip0.2em\null

\columnbreak

$$
V_{m \times n} =
\begin{bmatrix}
    a_1^0 & a_1^1 & \cdots & a_1^{n-1}\tikzmark{mv1r1} \\[0.5em]
    a_2^0 & a_2^1 & \cdots & a_2^{n-1}\tikzmark{mv1r2} \\
    \vdots  & \vdots  & \ddots & \vdots  \\
    a_m^0 & a_m^1 & \cdots & a_m^{n-1}\tikzmark{mv1r3}
\end{bmatrix}
$$
\end{multicols}

\begin{multicols}{2}

\null

\noindent A matriz de \textit{Vandermonde} ao lado possui PG's em suas \textbf{colunas}. Podemos ver que os expoentes começam em $0$ e vão até $n-1$, aumentando para baixo.

\vskip1em\null

\columnbreak

\null\vskip1em

$$
V_{n \times m} =
\begin{bmatrix}
    \tikzmark{mv2c1}a_1^0 & \tikzmark{mv2c2}a_2^0 & \cdots & \tikzmark{mv2c3}a_m^0 \\[0.5em]
    a_1^1 & a_2^1 & \cdots & a_m^1 \\
    \vdots  & \vdots  & \ddots & \vdots  \\
    a_1^{n-1} & a_2^{n-1} & \cdots & a_m^{n-1}
\end{bmatrix}
$$

\null

\end{multicols}
\begin{tikzoverlay}[]
    \draw[latex-] (mv1r1) ++(5pt,2pt) -- +(0.5cm, 0) node[anchor=west] {PG de base $a_1$};
    \draw[latex-] (mv1r2) ++(5pt,3pt) -- +(0.5cm, 0) node[anchor=west] {PG de base $a_2$};
    \draw[latex-] (mv1r3) ++(5pt,4pt) -- +(0.5cm, 0) node[anchor=west] {PG de base $a_m$};
    
    \draw[latex-] (mv2c1) ++(5pt, 10pt) -- +(0, 0.5cm) node(a)[]{};
    
    \node at ($(a) +(20pt, -3pt)$) [above left] {PG de base $a_1$};
    
    \draw[latex-] (mv2c2) ++(5pt, 10pt) -- +(0, 1cm) node(b) [] {};
    
    \node at ($(b) +(10pt, -3pt)$) [above] {PG de base $a_2$};
    
    \draw[latex-] (mv2c3) ++(5pt, 10pt) -- +(0, 0.5cm) node(c)[] {};
    
    \node at ($(c) +(-20pt, -3pt)$)[above right]{PG de base $a_m$};
\end{tikzoverlay}
\chapter{Soluções de um sistema linear}

\section{O que são?}

As \textbf{soluções} para um dado sistema linear devem ser escritas organizadamente em uma \textbf{ênupla} (coordenada com $n$ posições, a palavra vem de \textit{n-upla}) \textbf{ordenada} (pois tem uma ordem específica) \textbf{de reais}. Uma solução de um sistema linear consiste em uma série de valores que deverá resolver o sistema de igualdades proposto.

\Example

No sistema

$$
S_1\begin{cases}
2x+y=1\\
2x-y=3
\end{cases}
$$

temos como solução a ênupla $(1, -1)$, pois se substituirmos $x$ e $y$ por esses números respectivamente, teremos as igualdades

$$
S_1\begin{cases}
2\cdot 1 + (-1)=1\\
2\cdot 1-(-1)= 3
\end{cases}
$$

Repare que na ênupla o primeiro número substitui a incógnita $x$ e o segundo, $y$. A posição de cada valor na ênupla nos dirá qual incógnita cada valor deverá substituir (daí vem o termo \textit{ordenada}).
\chapter{Teorema de Cramer}
\label{chap:cram}

\section{O que é?}

Esse teorema nos diz que, dado um sistema linear $S$ que tenha o \textbf{mesmo número} de incógnitas e expressões, se o colocarmos na forma matricial poderemos tomar seu determinante $D$, e, caso esse seja diferente de $0$, poderemos \textbf{determinar} os valores da \textbf{única solução possível} desse sistema.

\section{Passo a passo}

$$
\text{Dado o sistema }S_1\begin{cases}
2x-3y=0\\
x+2y=2
\end{cases}
$$

\begin{enumerate}
    \item Primeiro devemos checar se o número de \textbf{incógnitas} é \textbf{igual} ao número de \textbf{expressões}.
    
    No sistema acima isso é verdadeiro, pois temos $x$ e $y$ (duas incógnitas) e temos duas expressões (duas linhas). \checkmark
    
    \paragraph{Nota:}
    Caso o sistema tenha mais expressões do que incógnitas podemos ignorar algumas das expressões a fim de realizar os cálculos necessários com esse sistema.
    
    \item Agora podemos escrever o sistema em \textit{forma matricial}, para a realização desse teorema basta escrever a \textbf{primeira parte}, formada pelos \textbf{coeficientes}. Chamaremos essa matriz de $A$.
    $$
    A=\begin{bmatrix*}[r]
    2 & -3\\
    1 & 2
    \end{bmatrix*}
    $$

    \item Devemos, então, tomar o \textbf{determinante} desse sistema. 
    
    Chamaremos esse determinante de $D$.
    
    $$
    D=\det A=2\cdot 2-(-3)\cdot 1\Rightarrow D=4+3=7
    $$
    
    \item Se o determinante $D$ for \textbf{diferente} de zero, podemos prosseguir com o teorema.
    
    No sistema que estamos utilizando isso se verifica. \checkmark
\end{enumerate}

\section{Encontrando a única solução possível}

\begin{enumerate}
    \item Para encontrar o \textbf{valor} de uma incógnita $i$ qualquer (valor esse que \textbf{resolverá} o sistema) devemos \textbf{repetir} a matriz dos coeficientes ($A$) e trocar a coluna representada por essa incógnita de escolha por uma coluna feita \textbf{a partir} dos termos independentes. O determinante da nova matriz será chamado $D_i$.
    
    Continuando com o mesmo exemplo, vamos resolver para $x$:
    
    $$
    D_x=\begin{vmatrix*}[r]
    \color{green}0 & -3\\ \color{green}2 & 2
    \end{vmatrix*} \Rightarrow \\
    D_x=0 \cdot 2 -(-3)\cdot 2 = 6
    $$
    
    Repare que substituímos os termos da coluna que representa os coeficientes de $x$ por uma coluna formada pelos termos independentes (em \textcolor{green}{verde}), que devem estar na \textbf{mesma ordem} do sistema.
    
    \item O valor solução $\alpha_i$ da incógnita escolhida ($i$) será encontrado pela equação:
    
    $$
    \alpha_i=\frac{D_i}{D}
    $$
    
    Seguindo com o exemplo, fazemos:
    
    $$
    \text{Substituíndo } \alpha_x = \frac{D_x}{D} \text{ pelos valores que possuímos, encontramos } \alpha_x=\frac{6}{7}
    $$
    
    \item Agora devemos repetir o processo para as outras incógnitas.
    
    Fazendo para $y$ temos:
    
    \begin{gather*}
        D_y=\begin{vmatrix}
        2 & 0\\ 1 & 2
        \end{vmatrix} \Rightarrow D_y= 2\cdot 2-0\cdot 2=4\\
        \alpha_y=\frac{D_y}{D}\Rightarrow\alpha_y=\frac{4}{7}
    \end{gather*}
    
    \item Por fim, escrevemos a solução encontrada em forma de ênupla.
    
    $$
    \left(\sfrac{6}{7},~\sfrac{4}{7}\right)
    $$
    
    \item \textbf{(Extra)} Caso ache necessário, basta substituir os valores da ênupla nas expressões do sistema para \textbf{testar} a solução.
    
    $$
    S_1\begin{cases}
    2\cdot \dfrac{6}{7}-3\cdot\dfrac{4}{7}=0 \\[1em]
    \dfrac{6}{7}+2\cdot\dfrac{4}{7}=2
    \end{cases}\Rightarrow 
    S_1\begin{cases}
    \dfrac{12-12}{7}=0\\[1em]
    \dfrac{6+8}{7}=2
    \end{cases}\Rightarrow 
    S_1\begin{cases}
    \dfrac{0}{7}=0\\[1em]
    \dfrac{14}{7}=2
    \end{cases}
    $$

\end{enumerate}
\chapter{Escalonamento}
\label{chap:esc}

\section{Introdução}
Para resolver e diferenciar os sistemas lineares dispomos da técnica chamada \mbox{\textit{escalonamento}}. Essa técnica consiste em um \textbf{algoritmo} (outra palavra para \textit{receita}) onde vamos, através de soma e multiplicação, \textbf{zerar} coeficientes das expressões de um sistema linear, um por um.

\section{Algoritmo (ou receita)}
\begin{enumerate}
    \item Devemos escolher alguma das expressões de um sistema para usar como \textbf{referência}. É boa prática escolher uma expressão onde os coeficientes são \textbf{menores}.
    \Example
    $$
    S_1\begin{cases}
    \color{red}x+y-z=2\\
    3x-4y+2z=1\\
    5x+2y-z=7
    \end{cases}
    $$
    
    Vamos usar a primeira expressão (em \textcolor{red}{vermelho}). Vamos chamá-la de \ref{eq:a}.
    
    \item Agora vamos \textbf{escolher} alguma incógnita com o objetivo de \textbf{zerar} seu \textbf{coeficiente}. Nessa etapa podemos organizar as expressões colocando a escolhida \textbf{em cima} das outras.
    
    \paragraph{Continuação:}
    
    \begin{empheq}[left=S_1\empheqlbrace]{align*}
    &x+y-z=2 \label{eq:a}\tag{$S_a$}\\
    &3x-4y+2z=1 \label{eq:b}\tag{$S_b$}\\
    &5x+2y-z=7 \label{eq:c}\tag{$S_c$}
    \end{empheq}
    
    A expressão referência se encontra \textbf{acima} das que pretendemos mexer (\ref{eq:a} acima de \ref{eq:b} e \ref{eq:c}).
    
    \item Através de equação podemos encontrar valores que, \textbf{multiplicados} à primeira equação, vão \textbf{zerar} o \textbf{coeficiente} da incógnita de escolha nas outras.
    
    A expressão \ref{eq:a} multiplicada por uma constante $k_1$ e somada à expressão \ref{eq:b} deverá zerar o coeficiente que multiplica $x$:
    $$
    S_a\times k_1 + S_b=0
    $$
    
    Porém, como é \textbf{absurdo} fazer contas para $n$ valores ao mesmo tempo, vamos focar no que nos interessa, o valor de $x$ nas expressões \ref{eq:a} e \ref{eq:b}. Para isso basta \textbf{substituir as expressões} \ref{eq:a} e \ref{eq:b} por seus \textbf{respectivos coeficientes} de $x$:
    $$
    1\cdot k_1 + 3=0
    $$
    Resolvendo, temos:
    $$
    1\cdot k_1 + 3=0 \Rightarrow k=-3
    $$
    
    Fazendo o mesmo para \ref{eq:a} e \ref{eq:c}, temos:
    $$
    S_a\times k_2+S_c \Rightarrow 1\cdot k_2+5=0 \Rightarrow k=-5
    $$
    
    \paragraph{Nota:}
    O valor $k_1$ (que zera o coeficiente de $x$ em \ref{eq:b}) \textbf{não necessariamente} é o mesmo que $k_2$ (que zera o coeficiente de $x$ em \ref{eq:c}).
    
    \item Agora basta \textbf{multiplicar} a expressão de \textbf{referência} pelo valor encontrado e \textbf{somar} às outras expressões. 
    
    Como os valores que zeram cada uma são \textbf{diferentes}, executamos esse passo \textbf{independentemente}.
    
    Sabemos que $k_1=-3$ zera a expressão \ref{eq:b}, então faremos:
    \begin{gather*}
        (x+y-z=2)\times (-3) +S_b\Rightarrow\\
        (-3x-3y+3z=-6) + S_b\Rightarrow\\
        \begin{aligned}
        -3x-3y+3z&=-6\\
        \mathbf{+}~~3x-4y+2z&=2\\\midrule
         0x-7y+5z&=-4
        \end{aligned}
    \end{gather*}
    
    Colocando a nova expressão no lugar de \ref{eq:b}, ficamos com:
    $$
    S_1\begin{cases}
    x+y-z=2\\
    0x-7y+5z=-4\\
    5x+2y-z=7
    \end{cases}
    $$
    
    \paragraph{Nota:} 
    a expressão de referência (\ref{eq:a}) \textbf{não se altera}, e nem a expressão \ref{eq:c}. No escalonamento só alteramos \textbf{uma expressão de cada vez}.
    
    Vamos fazer a mesma coisa para \ref{eq:a} e \ref{eq:c}. Sabemos que $k=-5$ zera a expressão \ref{eq:c}, então fazemos:
    \begin{gather*}
        (x+y-z=2) \times (-5) + S_c\Rightarrow\\
        (-5x-5y+5z=-10) + S_c\Rightarrow\\
        \begin{aligned}
        -5x-5y+5z&=-10\\
        \mathbf{+}~~5x+2y-z&=7\\\midrule
        0x-3y+4z&=-3
        \end{aligned}
    \end{gather*}
    
    Colocando a nova expressão no lugar de \ref{eq:c}, temos:
    $$
    S_1\begin{cases}
    x+y-z=2\\
    0x-7y+5z=-4\\
    0x-3y+4z=-3
    \end{cases}
    $$
    
    \item Por fim, devemos repetir os passos $1$ a $4$ escolhendo \textbf{outra incógnita} para "zerar" e outra expressão como referência, \textbf{sem mexer} na referência anterior.
    
    Usando \ref{eq:b} como referência para zerar o coeficiente $y$ em \ref{eq:c} temos:
    $$
    S_b\times k_3 +S_c=0\Rightarrow -7\cdot k + (-3) =0\Rightarrow k_3= \sfrac{3}{7}
    $$
    
    Como não devemos alterar as referências anteriores (expressão \ref{eq:a}), basta encontrar o valor $k_3$ para zerar o coeficiente $y$ em \ref{eq:c}.
    
    Fazendo a conta ficamos com:
    \begin{gather*}
        (-7+5z=-4) \times \left(\sfrac{-3}{7}\right)+S_c\Rightarrow\\
        \left(3y-\frac{15}{7}=-\frac{12}{7}\right) + S_c\Rightarrow\\
        \begin{aligned}
        3y-\frac{15}{7}&=-\frac{12}{7}\\
        \mathbf{+}~~-3y+4z&=-3\\\midrule
        0y+\frac{13}{7}z&=-\frac{33}{7}
        \end{aligned}
    \end{gather*}

    Colocando a nova expressão no lugar de \ref{eq:c} temos:
    $$
    S_1\begin{cases}
    x+y-z=2\\
    0x-7y+5z=-4\\
    0x+0y+\dfrac{13}{7}=-\dfrac{33}{7}
    \end{cases}
    $$
    
    Podemos ver que a cada vez que \textbf{repetimos} o algoritmo zeramos os coeficientes de \textbf{outra incógnita}.
    
    Agora que não há mais expressões para mexer, pois devemos ter pelo menos uma nova referência e uma outra expressão para mexer (e no momento só teríamos uma referência e mais nada), podemos, então, interromper o algoritmo.


\end{enumerate}
\chapter{Casos interessantes}

\section{Matriz transposta}
O determinante de uma matriz quadrada qualquer $A_n$ será \textbf{igual} ao determinante de sua \textbf{transposta}.
$$
\det A_n = \det A_n^t
$$

\section{Fila nula}
Caso haja qualquer \textbf{linha} ou \textbf{coluna nula} em uma matriz seu determinante será \textbf{zero}.
$$
\text{Dada a matriz }A_n=\begin{bmatrix}
    a_{11} & a_{12} & 0 & \cdots & a_{1n} \\
    a_{21} & a_{22} & 0 & \cdots & a_{2n} \\
    \vdots  & \vdots & \vdots & \ddots & \vdots  \\
    a_{m1} & a_{m2} & 0 & \cdots & a_{mn}
\end{bmatrix} \text{, } \det A_n=0 \text{ pois sua 3\textsuperscript{a} coluna é nula.}
$$

\section{Multiplicação de uma fila por uma constante}
Se toda uma \textbf{linha} ou \textbf{coluna} de uma matriz quadrada qualquer $A_n$ for \textbf{multiplicada} por um valor, podemos \textbf{reescrever} o determinante como sendo multiplicado por aquele valor (e retirá-lo da matriz).

\Example
$$
\text{Dado o }\det A_n= 
\begin{vmatrix}
    a_{11} & a_{12} & k\cdot a_{13} & \cdots & a_{1n} \\
    a_{21} & a_{22} & k\cdot a_{23} & \cdots & a_{2n} \\
    \vdots  & \vdots & \vdots &  \ddots & \vdots  \\
    a_{n1} & a_{n2} & k\cdot a_{n3} & \cdots & a_{nn}
\end{vmatrix}
$$
podemos reescrever esse determinante como
$$
k \cdot \det A_n = k \cdot 
\begin{vmatrix}
    a_{11} & a_{12} & a_{13} & \cdots & a_{1n} \\
    a_{21} & a_{22} & a_{23} & \cdots & a_{2n} \\
    \vdots  & \vdots & \vdots & \ddots & \vdots  \\
    a_{n1} & a_{n2} & a_{n3} & \cdots & a_{nn}
\end{vmatrix}
$$

\Example

\begin{gather*}
    \text{Dada a matriz }A=\begin{bmatrix*}[r]
        1 & 3 & 8\\
        7 & -9 & 1\\
        -2 & 4 & -6
    \end{bmatrix*} \text{, que pode ser reescrita como:}\\
    A=\begin{bmatrix*}[r]
        1 & 3 & 8\\
        7 & -9 & 1\\
        -1\cdot2 & 2\cdot2 & -3\cdot2
    \end{bmatrix*} 
    \text{ temos que }\det A= 2\cdot 
    \begin{vmatrix*}[r]
        1 & 3 & 8\\
        7 & -9 & 1\\
        -1 & 2 & -3
    \end{vmatrix*}
\end{gather*}

\section{Multiplicação da matriz inteira por uma constante}
Se uma matriz quadrada $A_n$ \textbf{inteira} for \textbf{multiplicada} por uma constante $k$ podemos \textbf{retirá-la} da matriz e seu determinante será \textbf{igual} a constante \textbf{elevada} a ordem da matriz ($k^n$) \textbf{multiplicado} pelo determinante da matriz \textbf{sem a constante}.

\begin{gather*}
    \text{Se } k\cdot A_n=\begin{bmatrix*}[c]
        k\cdot a_{11} & k\cdot a_{12} & \cdots & k\cdot a_{1n} \\
        k\cdot a_{21} & k\cdot a_{22} & \cdots & k\cdot a_{2n} \\
        \vdots  & \vdots & \ddots & \vdots  \\
        k\cdot a_{m1} & k\cdot a_{m2} & \cdots & k\cdot a_{mn}
    \end{bmatrix*} 
    \text{, então, segue que} \\[0.5em]
    \det (k \cdot A_n) = k^n \cdot \det A_n
\end{gather*}

\Example

\begin{gather*}
    \text{Dada a matriz }A=
    \begin{bmatrix*}[r]
        3 & -6 & -3\\
        12 & 15 & 0\\
        9 & -9 & 2
    \end{bmatrix*} 
    \text{, que pode ser reescrita como:}\\
    A=\begin{bmatrix*}[r]
        1\cdot3 & -2\cdot3 & -1\cdot3\\
        4\cdot3 & 5\cdot3 & 0\cdot3\\
        3\cdot3 & -3\cdot3 & \sfrac{2}{3}\cdot3
    \end{bmatrix*}
    =3\cdot \begin{bmatrix*}[r]
    1 & -2 & -1\\
    4 & 5 & 0\\
    3 & -3 & \sfrac{2}{3}
    \end{bmatrix*} 
    \text{ temos que } 
    \det A= 3^3 \cdot \begin{vmatrix*}[r]
    1 & -2 & -1\\
    4 & 5 & 0\\
    3 & -3 & \sfrac{2}{3}
    \end{vmatrix*}
\end{gather*}

\section{Troca de filas paralelas}
Caso \textbf{troquemos} duas \textbf{linhas} ou duas \textbf{colunas distintas} de uma matriz quadrada $A_n$, criamos uma nova matriz $B_n$ tal que: $$\det B_n=-\det A_n$$
\Example

$$
\text{Dada a matriz }A=\begin{bmatrix*}[r]
    a & b \\
    c & d
\end{bmatrix*}
$$
trocando a 1\textsuperscript{a} coluna pela 2\textsuperscript{a} coluna temos uma nova matriz $B$ tal que:
$$
B=\begin{bmatrix*}[r]
    b & a\\
    d & c
\end{bmatrix*} \text{ onde }\det B =-\det A
$$

\section{Filas paralelas iguais ou proporcionais}
Caso duas \textbf{linhas} ou \textbf{colunas distintas} sejam \textbf{múltiplas} uma da outra em uma matriz quadrada $A_n$ seu determinante será igual a $0$.

\Example
$$
\text{Dada a matriz }A=\begin{bmatrix*}[r]
1 & -4 & 1\\
2 & 5 & 2\\
3 & 12 & 3
\end{bmatrix*}
$$

Como a primeira coluna é múltipla da terceira (multiplicada por $1$):

$$
\det A=0
$$

\Example
$$
\text{Dada a matriz }A=\begin{bmatrix*}[r]
2 & 1 & -5\\
7 & 0 & 0\\
0 & -3 & 15
\end{bmatrix*}
$$

Como a segunda coluna é múltipla da terceira (multiplicada por $-5$):

$$
\det A=0
$$

\section{Matriz inversa}
Dada uma matriz quadrada $A_n$, o determinante de sua \textbf{inversa} obedece à relação:

$$
\det A_n^{-1}=\frac{1}{\det A_n}
$$

\section{Matriz triangular}

O determinante de uma matriz triangular $A_n$ onde $n$ é a ordem dessa matriz será o produto dos elementos de sua diagonal principal. Isso é válido para todo $n$.

\Example

$$
\det A=\begin{vmatrix}
a_{11} & 0 & 0\\
a_{21} & a_{22} & 0\\
a_{31} & a_{32} & a_{33}
\end{vmatrix}= a_{11} \cdot a_{22} \cdot a_{33}
$$

\section{Matriz identidade}

O determinante de uma matriz identidade $I_n$ onde $n$ é a ordem dessa matriz será igual a $1$ para qualquer valor possível de $n$.

$$
\det I_n=1
$$

\paragraph{Nota:}
Repare que a matriz identidade é meramente um caso específico de uma matriz triangular, e, portanto, como o determinante de uma matriz triangular qualquer será o produto de sua diagonal principal, segue que $\det I_n=1\cdot 1\cdot1\dots1=1$ sempre.

\section{Multiplicação de matrizes}

O determinante de uma matriz $C=A\cdot B$ vai ser igual ao produto dos determinantes das matrizes $A$ e $B$.

$$
\det C=\det (A\cdot B) = \det A \cdot \det B
$$

\section{Matriz de Vandermonde}

O determinante de uma matriz de \textit{Vandermonde} quadrada $V_n$ de ordem $n$ será dado pelo produto de todas as diferenças possíveis entre os elementos da linha/coluna onde estão as bases das PG's da matriz. As diferenças devem ser de um elemento qualquer (que não será o primeiro da linha/coluna) subtraído de algum que vem antes dele (e que não será o último da linha/coluna).

\Example


$$
\text{Dada a matriz de \textit{Vandermonde} }V=\begin{bmatrix*}[r]
1 & 4 & 16\\
1 & 9 & 81\\
1 & 5 & 25
\end{bmatrix*}
$$
vamos, primeiro, isolar a coluna dessa matriz onde estão as bases das PG's:
$$
\begin{bmatrix*}[r]
4\\9\\5
\end{bmatrix*}
$$
as diferenças possíveis entre dois elementos dessa coluna que respeitam a condição de o \textit{minuendo} (termo que vem primeiro, da qual se subtrai) estar em uma posição posterior ao \textit{subtraendo} (termo que vem depois, que é subtraído) são:
\begin{gather*}
    9-4\\
    5-4\\
    5-9
\end{gather*}
podemos calcular o determinante de $V$ fazendo:
$$
\det V=(5-9)\cdot(9-4)\cdot(5-4)=(-4)\cdot5\cdot1=-20
$$

\end{document}
 
