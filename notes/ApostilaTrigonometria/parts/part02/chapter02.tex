\chapter{Ciclo trigonométrico}

O ciclo trigonométrico é uma ferramenta muito útil para o estudo da trigonometria. Consiste em uma circunferência de raio $1$ montada na origem de um plano cartesiano.

\begin{tikzscale}[4]
	\draw[step=0.5, gray] (-1.4,-1.4) grid (1.4,1.4);

	\draw[->] (-1.5cm,0cm) -- (1.5cm,0cm) node[right,fill=white] {$x$};
	\draw[->] (0cm,-1.5cm) -- (0cm,1.5cm) node[above,fill=white] {$y$};

	% draw the unit circle
	\draw[thick] (0cm,0cm) circle(1cm);

	\tkzDefPoints{0/0/O,1/0/R}

	\tkzDrawSegment[white, thick](O,R)
	\tkzDrawSegment[red](O,R)

	\tkzLabelSegment[above=1pt, red, fill=white](O,R) {$r=1$}

	\draw (-1.2cm,0cm) node[above=1pt] {$(-1,0)$}
	(1.15cm,0cm)  node[above=1pt] {$(1,0)$}
	(0cm,-1.1cm) node[fill=white] {$(0,-1)$}
	(0cm,1.1cm)  node[fill=white] {$(0,1)$};

	\foreach \num in {0, 90, 180, 270}
	\filldraw[black] (\num:1cm) circle(0.4pt);

	\filldraw[black] (0,0) circle (0.4pt) node[below=2pt, xshift=7pt, fill=white] {$(0,0)$};

\end{tikzscale}

\begin{multi}
	\noindent Se escolhermos um ponto qualquer $A$ na circunferência e traçarmos o raio até esse ponto vamos ter um ângulo $\theta$ sendo formado entre o novo segmento $\overline{OA}$ e o antigo segmento $\overline{OR}$.

	\noindent Uma idéia interessante é imaginar o segmento $\displaystyle\overline{OR}$ "rodando" ~em torno da circunferência ,e, conforme aumenta, $\theta$ também aumenta.

	\nextcol

	\begin{tikzscale}[2.5]
		\draw[step=0.5, gray] (-1.4,-1.4) grid (1.4,1.4);

		\draw[->] (-1.5cm,0cm) -- (1.5cm,0cm) node[right,fill=white] {$x$};
		\draw[->] (0cm,-1.5cm) -- (0cm,1.5cm) node[above,fill=white] {$y$};

		\tkzDefPoints{0/0/O,1/0/R}

		\tkzDefPoint(30:1){A}

		\tkzDrawSegment[white, thick](O,R)
		\tkzDrawSegment[red](O,R)

		\tkzDrawSegment[red, dashed](O,A)

		\draw ($(A) +(4pt, 1pt)$) node[] {$A$}
		(0.56cm, 0.15cm) node[red] {$\theta$};

		\tkzDrawArc[color=black, thick](O,A)(R)
		\draw[->, red, >=stealth, thick] (1cm, 0) arc (0:29:1cm);

		\filldraw[black] (0:1cm) circle(0.4pt) node[yshift=-5, xshift=5] {$R$};
		\filldraw[black] (0,0) circle (0.4pt) node[yshift=-5, xshift=5] {$O$};

		\filldraw[black] (A) circle (0.4pt);

		\draw[->, red, >=stealth] (0.5cm, 0) arc (0:30:0.5cm);

	\end{tikzscale}
\end{multi}

\begin{multi}
	Podemos notar que esse existe um triângulo retângulo de hipotenusa igual ao segmento $\overline{OA}$.

	\nextcol

	\begin{tikzscale}[2.5]
		\draw[step=0.5, gray] (-1.4,-1.4) grid (1.4,1.4);

		\draw[->] (-1.5cm,0cm) -- (1.5cm,0cm) node[right,fill=white] {$x$};
		\draw[->] (0cm,-1.5cm) -- (0cm,1.5cm) node[above,fill=white] {$y$};

		\draw (0,0) circle (1cm);

		\tkzDefPoints{0/0/O,1/0/R}

		\tkzDefPoint(30:1){A}

		\coordinate(I) at (30:1cm |- 0,0);

		\begin{scope}[scale=0.3]
			\tkzMarkRightAngle[line width=0.3pt](O,I,A)
			%\tkzLabelAngle[pos=0.15](O,I,A){\small$\cdot$}

		\end{scope}

		\draw[white] (0,0) -- (I);
		\tkzDrawSegments[red](O,A O,I A,I)

		\draw ($(A) +(4pt, 1pt)$) node[] {$A$}
		(0.36cm, 0.1cm) node[] {$\theta$};

		%\filldraw[black] (0:1cm) circle(0.4pt) node[yshift=-5, xshift=5] {$R$};
		%\filldraw[black] (0,0) circle (0.4pt) node[yshift=-5, xshift=5] {$O$};

		%\filldraw[black] (A) circle (0.4pt);

		\draw[black] (0.3cm, 0) arc (0:30:0.3cm);

	\end{tikzscale}
\end{multi}