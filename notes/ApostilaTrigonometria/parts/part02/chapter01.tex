\chapter{Arcos e ângulos}

\section{Radianos}

\begin{multi}

	\noindent Denotamos por $\arc{AXB}$ o arco formado pelo ângulo $B\Hat{O}A$ passando pelo ponto $X$. Podemos medir arcos usando graus. O arco do exemplo mede $\SI{60}{\degree}$.

	\nextcol

	\begin{tikzscale}[0.8]
		\tkzDefPoint(0,0){O}
		\tkzDefPoint(3,0){R}
		\begin{scope}[shift={(O)}]
			\tkzDefPoint(30:3){X}
			\tkzDefPoint(60:3){A}
		\end{scope}

		\tkzDrawArc[color=red, thick](O,R)(A)
		\tkzDrawArc[color=black](O,A)(R)

		\tkzDrawSegments(O,R O,A)

		\tkzMarkAngle[size=1](R,O,A)

		\tkzDrawPoints[fill=red!50, color=red!80!black](X)

		\tkzLabelPoint[yshift=10pt](A){$A$}
		\tkzLabelPoint[yshift=10pt](X){$X$}
		\tkzLabelPoint[xshift=-17pt, yshift=5pt](O){$O$}
		\tkzLabelPoint[yshift=5pt, xshift=0pt](R){$B$}

		\tkzLabelPoint[xshift=19pt, yshift=21pt](O){$\SI{60}{\degree}$}

	\end{tikzscale}
\end{multi}


\begin{multi}

	\noindent Podemos medir um arco usando uma outra medida muito prática chamada \textbf{radiano} (denotada $\si{\radian}$).

	\noindent O radiano é definido como a razão entre o comprimento do arco e o raio do círculo que o descreve:

	$$\alpha~\si{\radian}=\frac{l}{r}$$

	\nextcol

	\begin{tikzscale}[0.8]
		\tkzDefPoints{0/0/O,3/0/R}

		\begin{scope}
			\tkzDefPoint(55:3){A}
		\end{scope}

		\tkzDrawArc[color=red, thick](O,R)(A)

		\tkzDrawArc[color=black](O,A)(R)

		\tkzDrawSegments(O,R O,A)

		\tkzMarkAngle[size=1](R,O,A)

		\curlybrace[thick]{0}{55}{3.15}

		\tkzLabelPoint[xshift=25, yshift=-15](A){$l$}

		\tkzLabelPoint[xshift=-15pt, yshift=3pt](O){$O$}
		\tkzLabelPoint[xshift=0pt, yshift=3pt](R){$R$}

		\tkzLabelSegment[below](O,R){$r$}

		\tkzLabelPoint[xshift=19pt, yshift=21pt](O){$\alpha~\si{\radian}$}
	\end{tikzscale}
\end{multi}

\begin{multi}

	\noindent A medida de $\SI{1}{\radian}$ na circunferência é mostrada ao lado.

	\noindent Repare que, independentemente do raio da circunferência, $1~\si{\radian}$ equivale a um arco com a medida de um raio, pois só teremos $\SI{1}{\radian}$ quando $l=r$, já que $\si{\radian}=\sfrac{l}{r}$.

	\nextcol

	\begin{tikzscale}[0.8]
		\tkzDefPoints{0/0/O,3/0/R,3/3/RU}

		\tkzDefPointBy[rotation in rad= center O angle 1.05](R) \tkzGetPoint{RU'}

		\tkzDrawArc[color=red, thick, dashed](O,R)(RU')

		\tkzDrawArc[color=black](O,RU')(R)

		\tkzDrawArc[dashed, color=black](R,RU)(RU')

		\tkzDrawSegments(O,R R,RU O,RU')

		\tkzMarkAngle[size=1](R,O,RU')

		\tkzLabelPoint[xshift=-15pt, yshift=3pt](O){$O$}
		\tkzLabelPoint[xshift=0pt, yshift=3pt](R){$R$}

		\tkzLabelSegment[below](O,R){$r$}

		\tkzLabelSegment[right](R,RU){$r$}

		\tkzLabelPoint[xshift=19pt, yshift=21pt](O){$\SI{1}{\radian}$}

	\end{tikzscale}
\end{multi}

\section{Medida da circunferência}

Sabemos que a circunferência mede $\SI{360}{\degree}$, no total. Mas qual é esse valor em $\si{\radian}$?

Primeiro, vamos nos lembrar que $\pi=\dfrac{c}{d}$ onde $c$ é o comprimento da circunferência e $d$ é seu diâmetro. Repare que a definição de $\pi$ se assemelha à definição de $\si{\radian}$, a única diferença sendo que $\pi$ usa o diâmetro ao invés do raio.

Como $d=2r$, podemos fazer:

$$
	\pi=\frac{c}{d}=\frac{c}{2r}\Rightarrow 2\pi=\frac{c}{r}
$$

Sabendo que queremos achar o valor de $\dfrac{c}{r}$ em $\si{\radian}$, podemos fazer:

$$
	\alpha=\frac{l}{r}=\frac{c}{r}\Rightarrow\alpha=2\pi~\si{\radian}
$$
