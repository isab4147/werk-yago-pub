\chapter{Introdução}

\section{Um pouco de contexto}
A trigonometria (do grego \textit{trigōnon} "triângulo" + \textit{metron} "medida") é o ramo da matemática que se dedica ao estudo de relações envolvendo comprimentos e ângulos de triângulos, e é o foco dessa apostila.

Nosso estudo se inicia nos triângulos, na verdade, em um caso bem específico de triângulo, chamado \textit{retângulo}, os triângulos chamados de retângulo são aqueles que têm um ângulo de $\SI{90}{\degree}$.

Nessa apostila vamos adotar a convenção de que:

\begin{enumerate}[label=(\Roman*), align=Center]
	\item pontos serão representados por letras maiúsculas ($A$, $B$, ..., $Z$)
	\item segmentos serão representados pelos pontos que os determinam ($\overline{AB}$, $\overline{BC}$, ..., $\overline{XZ}$) ou por letras minúsculas ($a$, $b$, ...,$z$).
	\item ângulos serão representados pelo arco que determinam ($\angle ABC$, $A\Hat{B}C$ ou, simplesmente $\Hat{B}$) ou letras gregas ($\alpha$, $\beta$, ..., $\theta$).
\end{enumerate}

\paragraph{Exemplo:}

\begin{tikzscale}[0.8]
	\tkzDefPoints{0/0/A,4/0/B,4/3/C}
	\tkzDrawTriangle[pythagore](A,B)

	\tkzLabelSegment[below=2pt, xshift=3pt](A,B){$b$}
	\tkzLabelSegment[right=2pt, yshift=-2pt](B,C){$a$}
	\tkzLabelSegment[above left](A,C){$c$}

	\tkzMarkRightAngle[line width=0.3pt](A,B,C)
	\tkzLabelAngle[pos=0.15](A,B,C){\small$\cdot$}
	%\tkzLabelPoint[xshift=-0.5cm, yshift=16pt](B){$\theta$}

	\tkzMarkAngle[size=1](B,A,C)
	\tkzLabelPoint[xshift=0.71cm, yshift=14pt](A){$\alpha$}

	\tkzLabelPoint[xshift=-12pt](A){$A$}
	\tkzLabelPoint[xshift=-2pt](B){$C$}
	\tkzLabelPoint[yshift=12pt, xshift=-2pt](C){$B$}

	\tkzMarkAngle[size=0.8](A,C,B)
	\tkzLabelPoint[xshift=-18pt, yshift=-15pt](C){$\beta$}
\end{tikzscale}
