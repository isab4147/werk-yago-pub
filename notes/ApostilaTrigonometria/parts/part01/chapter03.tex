\chapter{Ângulos notáveis}

Existem ângulos cujos valores de seno e cosseno valem ser lembrados, pois são recorrentes e simples.

\section{Razões trigonométricas para o ângulo de \texorpdfstring{$\SI{45}{\degree}$}{45º}}

Um triângulo retângulo que possui um ângulo de $\SI{45}{\degree}$ é automaticamente isósceles (se $\alpha$ é o terceiro ângulo desse triângulo, segue que $\alpha + \SI{45}{\degree}=\SI{90}{\degree}\Rightarrow\alpha=\SI{45}{\degree}$)

\paragraph{Nota:}
Um triângulo isósceles é um triângulo que tem dois lados e dois ângulos iguais.

Para encontrar os valores desejados vamos usar um triângulo isósceles com os lados iguais de valor $l=1$.

\begin{multicols}{2}
	Pelo teorema de pitágoras temos:
	$$
		l^2+l^2=h^2\Rightarrow1^2+1^2=h^2\Rightarrow2=h^2\Rightarrow h=\sqrt{2}
	$$
	Daí tiramos que:
	\begin{gather*}
		\sen\SI{45}{\degree}=\frac{1}{\sqrt{2}}=\frac{1}{\sqrt{2}}\cdot\frac{\sqrt{2}}{\sqrt{2}}=\frac{\sqrt{2}}{\left(\sqrt{2}\right)^2}=\frac{\sqrt{2}}{2}\\
		\cos\SI{45}{\degree}=\frac{1}{\sqrt{2}}=\frac{\sqrt{2}}{2}=\sen\SI{45}{\degree}
	\end{gather*}

	\begin{tikzscale}[1]
	\tkzDefPoints{0/0/A,4/0/B}
	\tkzDefPoint(45:5){C}
	\tkzDefLine[perpendicular= through B](A,B) \tkzGetPoint{b}

	\pgfresetboundingbox

	\tkzInterLL(A,C)(B,b) \tkzGetPoint{C'}

	\tkzMarkRightAngle[line width=0.3pt](A,B,C')
	\tkzLabelAngle[pos=0.15](A,B,C'){\small$\cdot$}

	\tkzLabelSegment[below=2pt, xshift=3pt](A,B){$l=1$}
	\tkzLabelSegment[right=2pt, yshift=-2pt](B,C'){$l=1$}
	\tkzLabelSegment[above, rotate=45](A,C'){$h=\sqrt{2}$}

	\tkzMarkAngle[size=1](B,A,C')
	\tkzLabelPoint[xshift=0.85cm, yshift=19pt](A){$\SI{45}{\degree}$}

	\tkzMarkAngle[size=1](A,C',B)
	\tkzLabelPoint[xshift=-25pt, yshift=-25pt](C'){$\SI{45}{\degree}$}

	\tkzDrawPolygon(A,B,C')

\end{tikzscale}

\end{multicols}

\section{Razões trigonométricas para o ângulo de \texorpdfstring{$\SI{30}{\degree}$}{30º}}

Vamos começar nossa demonstração com um triângulo equilátero, isso é, um triângulo que possui os três lados iguais (e, consequentemente, seus três ângulos também).

Os ângulos do nosso triângulo vão medir $\alpha=\sfrac{\SI{180}{\degree}}{3}=\SI{60}{\degree}$ e seus lados vão medir $l=2$.

\paragraph{Nota:}
Repare que os tracinhos nos ângulos e nos lados do triângulo mostram que são iguais (mesma quantidade de traços equivale a ter o mesmo tamanho).

\begin{minipage}{\textwidth}
	\begin{multicols}{2}
		Por pitágoras temos:
		\begin{gather*}
			h^2+c^2=l^2\Rightarrow h^2+1^2=2^2 \Rightarrow h^2=4-1\\
			h^2=3 \Rightarrow h=\sqrt{3}
		\end{gather*}
		Daí segue que:
		\begin{gather*}
			\sen\SI{30}{\degree}=\frac{1}{2}\\
			\cos\SI{30}{\degree}=\frac{\sqrt{3}}{2}
		\end{gather*}
		\vskip3em
		\hskip5em
		\begin{tikzscale}[1]
	\tkzDefPoints{0/0/A,4/0/B}
	\tkzDefPoint(60:4){C}
	\tkzDefMidPoint(A,B) \tkzGetPoint{mAB}

	\tkzDrawSegment(mAB, C)

	\tkzDrawPolygon(A,B,C)

	\tkzMarkRightAngle[line width=0.3pt](B,mAB,C)
	\tkzLabelAngle[pos=0.15](B,mAB,C){\small$\cdot$}

	\tkzLabelSegment[below=2pt](mAB,B){$c=1$}
	\tkzLabelSegment[above=2pt, rotate=-60](B,C){$l=2$}

	\tkzMarkAngle[size=0.8,mark=|](B,A,C)
	\tkzLabelPoint[xshift=23pt, yshift=20pt](A){$\SI{60}{\degree}$}

	\tkzMarkAngle[size=0.8, mark=|](C,B,A)

	\tkzMarkAngle[size=0.8,mark=||](A,C,mAB)
	\tkzLabelPoint[xshift=-18pt, yshift=-24pt](C){$\SI{30}{\degree}$}

	\tkzMarkAngle[size=0.8,mark=||](mAB,C,B)

	\tkzLabelSegment[yshift=-10pt,rotate=90](mAB,C){$h=\sqrt{3}$}

	\tkzMarkSegment[mark=|](A,C)
	\tkzMarkSegment[mark=|](C,B)
	\tkzMarkSegment[mark=||](A,mAB)
	\tkzMarkSegment[mark=||](mAB,B)

\end{tikzscale}

	\end{multicols}
\end{minipage}

\section{Razões trigonométricas para o ângulo de \texorpdfstring{$\SI{60}{\degree}$}{60º}}

Como $\SI{30}{\degree}$ e $\SI{60}{\degree}$ são ângulos complementares, vale que:

\begin{gather*}
	\sen\SI{30}{\degree}=\frac{1}{2}=\cos\SI{60}{\degree}\\[0.5em]
	\cos\SI{30}{\degree}=\frac{\sqrt{3}}{2}=\sen\SI{60}{\degree}
\end{gather*}

\section{Outras razões trigonométricas para os ângulos notáveis}

Vamos, agora, encontrar as relações básicas dos ângulos notáveis explorados anteriormente.

Para o ângulo de $\SI{45}{\degree}$ temos que:

\begin{gather*}
	\tg\SI{45}{\degree}=\frac{\sen\SI{45}{\degree}}{\cos\SI{45}{\degree}}=\frac{\frac{\sqrt{2}}{2}}{\frac{\sqrt{2}}{2}}=\frac{\cancel{\frac{\sqrt{2}}{2}}}{\cancel{\frac{\sqrt{2}}{2}}}=1\\[0.5em]
	\cotg\SI{45}{\degree}=\frac{1}{\tg\SI{45}{\degree}}=\frac{1}{1}=1=\tg\SI{45}{\degree}
\end{gather*}

Para os ângulos de $\SI{30}{\degree}$ e $\SI{60}{\degree}$ temos que:

\begin{gather*}
	\tg\SI{30}{\degree}=\frac{\sen\SI{30}{\degree}}{\cos\SI{30}{\degree}}=\frac{\frac{1}{2}}{\frac{\sqrt{3}}{2}}=\frac{1}{2}\div\frac{\sqrt{3}}{2}=\frac{1}{2}\cdot\frac{2}{\sqrt{3}}=\frac{1\cdot\cancel{2}}{\cancel{2}\cdot\sqrt{3}}=\frac{1}{\sqrt{3}}\cdot\frac{\sqrt{3}}{\sqrt{3}}=\frac{\sqrt{3}}{\left(\sqrt{3}\right)^2}=\frac{\sqrt{3}}{3}\\[0.5em]
	\cotg\SI{30}{\degree}=\frac{1}{\tg\SI{30}{\degree}}=\frac{1}{\frac{\sqrt{3}}{3}}=1\div\frac{\sqrt{3}}{3}=1\cdot\frac{3}{\sqrt{3}}=\frac{3}{\sqrt{3}}\cdot\frac{\sqrt{3}}{\sqrt{3}}=\frac{3\cdot\sqrt{3}}{\left(\sqrt{3}\right)^2}=\frac{\cancel{3}\sqrt{3}}{\cancel{3}}=\sqrt{3}
\end{gather*}

Como $\tg\SI{30}{\degree}=\cotg\SI{60}{\degree}$ e vice-versa, temos:

\begin{gather*}
	\tg\SI{60}{\degree}=\cotg\SI{30}{\degree}=\sqrt{3}\\
	\cotg\SI{60}{\degree}=\tg\SI{30}{\degree}=\frac{\sqrt{3}}{3}
\end{gather*}

Abaixo, segue uma tabela para refêrencia dos valores que encontramos.

\begin{table}[H]
	\centering
	\begin{tabular}{*{4}{c}}
		\toprule
		Razão   & \multicolumn{3}{c}{Ângulo}                                                 \\
		\cmidrule(lr){2-4}
		        & $\SI{30}{\degree}$         & $\SI{45}{\degree}$    & $\SI{60}{\degree}$    \\ \midrule
		$\sen$  & $\dfrac{1}{2}$             & $\dfrac{\sqrt{2}}{2}$ & $\dfrac{\sqrt{3}}{2}$ \\[5pt]
		$\cos$  & $\dfrac{\sqrt{3}}{2}$      & $\dfrac{\sqrt{2}}{2}$ & $\dfrac{1}{2}$        \\[5pt]
		$\tg$   & $\dfrac{\sqrt{3}}{3}$      & $1$                   & $\sqrt{3}$            \\[5pt]
		$\cotg$ & $\sqrt{3}$                 & $1$                   & $\dfrac{\sqrt{3}}{3}$ \\[5pt]
		\bottomrule
	\end{tabular}
	\caption{Tabela dos valores dos ângulos notáveis}
\end{table}


