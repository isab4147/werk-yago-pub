\chapter{Operações matemáticas}

\section{Introdução}

Nosso estudo se inicia com as operações elementares da matemática, nomeadamente:

\begin{RomanList}
	\item Soma;
	\item Subtração;
	\item Multiplicação, e;
	\item Divisão.
\end{RomanList}

Veremos, também, que as operações de subtração e divisão são simplesmente soma e subtração repetidas vezes.

\section{Contagem}

A contagem é, de certa forma, a operação mais essencial de todas, consiste em contar coisas e agrupá-las.

\subsection{Contando elementos}

Elementos são coisas, em geral, e serão o objeto de análise da contagem. Contamos usando os números naturais, isto é, $ 1,2,3,4,\ldots $. A contagem deve ser realizada na mesma ordem dos números naturais e é consistente, ou seja, toda vez que contarmos $ 5 $ de alguma coisa, se a quantidade dessa coisa não mudar, contaremos $ 5 $ se contarmos novamente. A contagem tampouco muda dependendo da ordem, por exemplo: se arrumarmos cartas de um baralho em um montinho podemos contar da cima, de baixo, ou de qualquer outra forma, contanto que contemos todas as cartas do montinho uma vez e uma vez somente, a contagem nos dará o mesmo resultado.

\Example
Podemos contar da esquerda pra direita...
\begin{xkcdenv}[1]
	
	\draw[thick, -{Latex[length=8pt]}] (0.5,0.5) -- (3.5,0.5);
	
	\foreach \x/\y/\w in {0/0/below,1/1/above,2/0/below,3/1/above,4/0/below}{
		\path (\x, \y) node[circle, xkcd, inner sep=0.3cm, draw] {} node[\w=15pt] {\pgfmathparse{int(\x + 1)}\pgfmathresult};
	}
	
\end{xkcdenv}
ou da direita pra esquerda...
\begin{xkcdenv}[1, xscale=-1]
	
	\draw[thick, -{Latex[length=8pt]}] (0.5,0.5) -- (3.5,0.5);
	
	\foreach \x/\y/\w in {0/0/below,1/1/above,2/0/below,3/1/above,4/0/below}{
		\path (\x, \y) node[circle, xkcd, inner sep=0.3cm, draw] {} node[\w=15pt] {\pgfmathparse{int(\x + 1)}\pgfmathresult};
	}
	
\end{xkcdenv}
o resultado será o mesmo, independentemente.

\subsection{Agrupando elementos}

O agrupamento de elementos é essencial para que entendamos o nosso sistema de representação de números, tópico que será abordado no final desse material, e consiste em formar grupos (daí o nome \textit{agrupamento} $=$ \textit{a} $+$ \textit{grupo} $+$ \textit{mento}) de elementos.

Um grupo de elementos pode ser formado juntando-os em quantidades regulares, por exemplo:

\Example
Se juntarmos $ 10 $ elementos em grupos de $ 5 $, teremos $ 2 $ grupos.
\begin{xkcdenv}[0.5]
	
	\draw (-1,-2) rectangle node[above=1.4cm] {grupo 1} (5, 3);
	\draw (6,-2) rectangle node[above=1.4cm] {grupo 2} (12, 3);
	
	\foreach \x/\y/\w in {0/0/below,1/1/above,2/0/below,3/1/above,4/0/below}{
		\path (\x, \y) node[circle, xkcd, inner sep=0.2cm, draw] {} node[\w=10pt] {\pgfmathparse{int(\x + 1)}\pgfmathresult};
	}

	\foreach \x/\y/\w in {7/0/below,8/1/above,9/0/below,10/1/above,11/0/below}{
		\path (\x, \y) node[circle, xkcd, inner sep=0.2cm, draw] {} node[\w=10pt] {\pgfmathparse{int(\x - 6)}\pgfmathresult};
	}

\end{xkcdenv}

\subsection{Zero}
O zero (representado por $ 0 $) é um conceito importante na contagem, mais recente na história devido a sua complexidade, ele representa a ausência de elementos. Quando não há alguma coisa, podemos dizer que "há zero" de algo. Quando iniciamos a contagem de algo, podemos não nos dar conta mas, na realidade, começamos a contar do zero, dessa forma, se nós contamos $ 10 $ cartas vermelhas e agora queremos contar somente cartas pretas, antes de começar a contar as pretas voltamos ao zero para começar novamente.

\Example

\begin{xkcdenv}[1]
	
	\draw[thick, -{Latex[length=8pt]}] (-0.5,-0.35) -- (9.4,-0.35);
	
	\path (-0.5, 0.9) node[above=1] {0};
	\path (-0.5, -1.6) node[below=1] {0};
	
	\foreach \x in {0,1,...,9}{
		\draw (\x,0) rectangle node[above=1] {\pgfmathparse{int(\x +1)}\pgfmathresult} +(0.9, 1.8);
		\draw[fill=black] (\x,0) ++(0.2,0.2) rectangle +(0.5, 1.4);
		
		\draw (\x,-2.5) rectangle node[below=1] {\pgfmathparse{int(\x +1)}\pgfmathresult} +(0.9, 1.8);
		\draw[fill=red] (\x,-2.5) ++(0.2,0.2) rectangle +(0.5, 1.4);
	}

\end{xkcdenv}

\subsection{Exercícios}

\Question Existem quantas cartas abaixo?

\begin{xkcdenv}[1]
	
	\foreach \x in {0,1,...,11}{
		\draw (\x,0) rectangle +(0.9, 1.8);
		\draw[fill=black] (\x,0) ++(0.2,0.2) rectangle +(0.5, 1.4);
		
		\draw (\x,-2) rectangle +(0.9, 1.8);
		\draw[fill=black] (\x,-2) ++(0.2,0.2) rectangle +(0.5, 1.4);
	}
	
\end{xkcdenv}

\section{Soma}

A soma é a operação aritmética mais essencial de todas, e consiste, de certa forma, em contar dois ou mais grupos de coisas sem voltar ao zero. O símbolo da soma é o $ + $. Também podemos chamá-la de \textit{adição}.

Como exemplo, vamos somar um grupo de $ 2 $ com um grupo de $ 7 $:

\Example
Usando objetos, podemos fazer da seguinte forma:
\begin{xkcdenv}[0.5]
	
	\draw (-1,-2) rectangle node[above=1.2cm] {2 objetos} (2.5, 2.8);
	\draw (5.5,-2) rectangle node[above=1.2cm] {7 objetos} (14.6, 2.8);
	
	\foreach \x/\y/\w/\z in {0/0/below/1,1.5/0.6/above/2}{
		\path (\x, \y) node[circle, xkcd, inner sep=0.2cm, draw] {} node[\w=10pt] {\z};
	}

	\path (4,0.5) node {\LARGE +};
	
	\foreach \x/\y/\w/\z in {4.5/0.5/below/1,6/0.2/above/2,7.1/0.3/below/3,8.2/0.8/above/4,9.3/0/below/5,10.5/0.3/above/6,11.6/0.5/below/7}{
		\path ($(\x, \y) +(2,0)$) node[circle, xkcd, inner sep=0.2cm, draw] {} node[\w=10pt] {\z};
	}

	\path (16.1,0.5) node {\LARGE =};
	
	\draw (1.6,-8.8) rectangle node[above=1.2cm] {9 objetos} (13.2, -4);
	
	\foreach \x/\y/\w/\z in {2.2/0/below/1,3.3/0.6/above/2,4.5/0.5/below/3,6/0.2/above/4,7.1/0.3/below/5,8.2/0.8/above/6,9.3/0/below/7,10.5/0.3/above/8,11.6/0.5/below/9}{
		\path ($(\x, \y) +(0.6,-6.8)$) node[circle, xkcd, inner sep=0.2cm, draw] {} node[\w=10pt] {\z};
	}
	
\end{xkcdenv}
Aqui nós juntamos os dois grupos em um só e contamos "tudo junto".

Uma ferramenta muito útil pra compreender a soma é a reta numérica. Um exemplo de reta numérica que usamos diariamente é a régua:

\Example

\begin{xkcdenv}[1, xscale=0.5]
	
	\draw (0,0) rectangle (20, 1);
	
	\foreach \x in {1,2,...,19}{
		\draw (\x,0) node[above=0.15] {\footnotesize\x} -- +(0,0.2);
	}
	\foreach \x in {0.25,0.75,...,19.75}{
		\draw (\x,0) -- +(0,0.1);
	}
	
	\foreach \x in {0.5,1.5,...,19.5}{
		\draw (\x,0) -- +(0,0.14);
	}
\end{xkcdenv}

Podemos somar distâncias usando a régua:

\Example
\begin{xkcdenv}[1]
	
	\draw[dashed] (0,1.7) -- (0,1) (2,1.7) -- (2,0.6) (9,1.7) -- (9,0.6);
	\draw[decorate,decoration={brace,amplitude=10pt}] (0,1.8) -- node[midway, above=0.5] {2 cm} +(2,0);
	
	\path (3.25,2.55) node {\LARGE +};
	
	\draw[decorate,decoration={brace,amplitude=10pt}] (2,1.8) -- node[midway, above=0.5] {5 cm} +(7,0);
	
	\path (9.5,2.55) node {\LARGE =};
	
	\draw (0,0) rectangle (10, 1);
	
	\foreach \x in {1,2,...,9}{
		\draw (\x,0) node[above=0.15] {\footnotesize\x} -- +(0,0.2);
	}
	\foreach \x in {0.25,0.75,...,9.75}{
		\draw (\x,0) -- +(0,0.1);
	}
	
	\foreach \x in {0.5,1.5,...,9.5}{
		\draw (\x,0) -- +(0,0.14);
	}
	
	\draw (0,-0.1) -- +(0,-0.2) (2,-0.1) -- +(0,-0.2) (9,-0.1) -- +(0,-0.2);
	
	\draw[decorate,decoration={brace,amplitude=10pt}] (9,-0.4) -- node[midway, below=0.5] {9 cm} (0,-0.4);

\end{xkcdenv}

Repare que nos dois exemplos de soma nós simplesmente contamos $ 1, 2 $ e, depois, sem voltar do zero, continuamos com $ 3, 4, 5, 6, 7, 8 $ e $ 9 $. Dessa maneira realizamos a operação $ 2+7=9 $. Na régua, basta colocarmos as duas distâncias lado a lado e medirmos o comprimento das duas juntas.

\paragraph{Nota:} As "retas numéricas" abordadas nesse capítulo não são de fato retas numéricas! Entenderemos isso mais à frente no material, mas, por praticidade, usarei o termo por enquanto. 

\subsection{Exercícios}

\Question
Realize a soma de:
\begin{Subquest}

	\item $ 2+5 $
	\item $ 3+2 $
	\item $ 2+3 $
	\item $ 22+17 $

\end{Subquest}

\Question
Há pouco vimos um exemplo de reta numérica, imagine uma escada gigante (ela pode ter quantos degraus você desejar!) e responda:
\begin{Subquest}
	
	\item Como poderíamos usar a escada para contar até $ 4 $?
	\item Como poderíamos usar a mesma escada para somar $ 4+3 $?
	\item Essa escada é uma reta numérica? Explique.
	
\end{Subquest}






