\chapter*{Introdução}

Existem poucas matérias mais essenciais e mais mal compreendidas do que a matemática, que é a alma dos estudos de física, química e das ciências exatas em geral. Essas ciências trazem consigo vários aspectos de raciocínio lógico que se debruçam, em peso, nos fundamentos da lógica (esta, por sua vez, vem da filosofia) e das operações básicas de matemática, e é entendendo estas que estaremos prontos para lidar com a matemática mais complexa e abstrata, exigida a partir do ensino médio e em muitos cursos de faculdade.

É com o pesar de saber que muitos estudantes, já estando no ensino médio brasileiro, compreendem muito pouco do que idealmente deveriam para ter a performance exigida pelo sistema de educação, que redigi essa apostila, feita com cuidado e sob medida, para atender às demandas de alunos, jovens e mais velhos, de um sistema injusto e precário de educação que presta pouca preocupação às necessidades dos que dependem dele.

Abordarei, nesse material, os tópicos de \textbf{aritmética}, \textbf{álgebra}, \textbf{geometria} e \textbf{raciocínio matemático}, essenciais para todo o estudo subsequente da matemática e ciências exatas. Cada parte conta com uma introdução dos conteúdos abordados, mostrando sua utilidade e os conceitos que serão tratados, contando com exercícios para construção de intuição no final de cada capítulo, e também ajudando o leitor a exercitar sua interpretação.